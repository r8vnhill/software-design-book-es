%!TEX root = C:\Users\Ignacio\Documents\Escuela\CC3002 - Metodologías de Diseño y Programación\apunte-y-ejercicios\src\latex\Apunte.tex
\section{\textit{JARs}}
  En esta sección veremos lo que son los \textit{Java Archives} más conocidos como 
  \textit{JARs} y las herramientas de las que disponemos para crear y utilizarlos.

  Hasta ahora hemos utilizado el \textit{IDE} para ejecutar nuestros programas, esto es bueno cuando
  estamos desarrollando pero no cuando queremos entregar un producto que sea utilizable por un 
  cliente (no podemos esperar que el cliente instale un \textit{IDE} solamente para correr el 
  programa).

  La manera más básica de ejecutar un programa en \textit{Java} es mediante la línea de comandos 
  (\textit{CLI}), finalmente lo que hace \textit{IntelliJ} es brindar una interfaz más amigable para
  que nosotros no tengamos que manejar el proceso de compilación manualmente.
  Antes de ver lo que es un archivo \textit{JAR} veamos un poco más en detalle lo que hace el 
  compilador de \textit{Java}.

  Si recuerdan, para compilar un programa desde la terminal se debían ejecutar dos instrucciones,
  primero \texttt{javac} y luego \texttt{java}.
  ¿Pero por qué dos comandos en lugar de simplemente uno?
  Una de las principales razones por las que \textit{Java} fue tan ampliamente adoptado en la 
  industria es porque fue uno de los primeros lenguajes de programación que era independiente del
  sistema operativo (i.e. que el mismo código funcionaba en \textit{Windows}, \textit{UNIX}, etc.).
  Para lograr esto el proceso de ejecución de un programa de \textit{Java} se dividió en dos grandes
  pasos, compilación y ejecución.

  El comando \textit{javac} invoca al compilador crea archivos (\texttt{.class}) con 
  \textit{bytecode} ejecutable por la \textit{JVM}.
  Luego, el comando \texttt{java} toma los archivos generados por el comando anterior para 
  convertirlos en binarios nativos del sistema operativo y así poder ejecutarlos.
  El beneficio de esto es que los archivos generados por el primer paso de la compilación serán 
  independientes del sistema operativo ya que no serán directamente interpretados por éste, sino que
  por la máquina virtual.

  Un \textit{JAR} es una forma de \enquote{enfrascar} en un archivo un programa junto con sus 
  metadatos y archivos adicionales que pueda necesitar\footnote{Se suele referir a estos archivos 
  adicionales como recursos del programa.} y así generar un programa más simple de distribuir y 
  ejecutar.

  \subimport{jars/}{command_line.tex}

  TODO:
  \begin{itemize}
    \item JARs
    \item JIT
    \item CLI
    \item Gradle
  \end{itemize}