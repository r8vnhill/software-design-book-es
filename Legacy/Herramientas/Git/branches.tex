% !TEX root = C:\Users\Ignacio\Documents\Escuela\CC3002 - Metodologías de Diseño y Programación\apunte-y-ejercicios\src\latex\Apunte.tex
\section{Ramas}
  Las ramas (o \textit{branches}) son uno de los aspectos más importantes al trabajar con 
  \textit{Git}, estas cumplen varios objetivos, pero el caso de uso más común es cuando se trabaja
  en equipo.
  Cuando utilizamos un repositorio remoto, varias personas pueden clonar y hacer \textit{push} al 
  mismo \textit{repo}, esto puede generar problemas cuando las versiones no son compatibles.

  Retrocedamos un poco para explicar el funcionamiento de \textit{Git}.
  La manera en la que \textit{Git} maneja el historial de versiones es manteniendo un grafo con los
  \textit{commits} realizados, este grafo tiene una rama principal que se crea al iniciar un nuevo
  repositorio y por defecto se llama \textit{master}.
  Lo recomendado es que en la rama principal solamente se encuentren las versiones del programa que
  están listas para ser utilizadas y se usen otras ramas para mantener las versiones durante el 
  proceso de desarrollo.

  Para crear una nueva rama en un repositorio se usa el comando \texttt{branch} seguido del nombre
  que se le desea dar a la rama.
  Luego, si queremos cambiar de rama utilizaremos el comando \texttt{checkout}.
  Por ejemplo:
  \begin{minted}{bash}
    git branch ramitas-saladas
    git checkout ramitas-saladas
  \end{minted}

  Al momento de cambiarse de rama, todos los \textit{commits} que se realicen se harán en esa rama.
  Lo siguiente que querríamos hacer es pasar nuestro trabajo a otra rama (como \textit{master}), 
  para esto utilizaremos la instrucción \texttt{merge} seguido del nombre de la rama que queremos 
  mezclar con la nuestra.
  Al igual que como siempre deben hacer \textit{pull} antes de \textit{push}, es importante que 
  \textbf{siempre realicen \textit{merge} en ambas direcciones}, primero pasamos los cambios de la 
  rama de destino a la nuestra, y luego los de la nuestra a la rama de destino, por ejemplo, si 
  quisiéramos actualizar \textit{master} con los cambios que hicimos en nuestra rama haríamos:
  \begin{minted}{bash}
    # Desde ramitas-saladas
    git merge master
    git checkout master
    git merge ramitas-saladas
  \end{minted}
