% !TEX root = D:\Users\Ignacio\Documentos\Escuela\CC3002 - Metodologías de Diseño y Programación\apunte-y-ejercicios\src\latex\Apunte.tex
\section{Repositorios}
  Un repositorio (generalmente llamado \textit{repo}) se puede entender como un proyecto
  con un sistema de versionado integrado.

  En el caso de \textit{Git}, uno de los principales beneficios es que puede utilizarse
  de manera local o remota utilizando un servidor o un \textit{host} de repositorios 
  (vea la sección \ref{sec:github}).

  Para utilizar \textit{Git} primero se debe crear un repositorio, para esto se utiliza el comando
  \texttt{init}.
  Comencemos entonces por crear un \textit{repo}, para esto abran una terminal y ejecuten:

  \begin{minted}{bash}
    mkdir my-repo && cd my-repo # Crea y se ubica en una carpeta
    git init
  \end{minted}

  Esto creará un repositorio dentro de la carpeta \texttt{my-repo}.

  En cualquier momento podemos revisar el estado del repositorio con la instrucción 
  \texttt{status}, luego:

  \begin{minted}{bash}
    git status
  \end{minted}

  Debiera entregar como resultado:

  \begin{minted}{text}
    On branch master

    No commits yet

    nothing to commit (create/copy files and use "git add" to track)
  \end{minted}

  Veamos ahora como agregar archivos al sistema de versionado.
  Para esto creen un archivo \texttt{README.md} y déjenlo en blanco, con el archivo creado corran
  nuevamente \texttt{git status}.

  Hemos creado un archivo, pero este no se agrega inmediatamente al sistema de versionado, esto es
  bueno ya que, como veremos más adelante, no queremos que todos nuestros archivos se agreguen a 
  \textit{Git}.
  Para agregar un archivo, esto se hace con el comando \texttt{add} y luego el nombre del archivo
  que se quiera agregar o \enquote{\texttt{.}} para agregar todos los archivos.
  Entonces, si queremos agregar el archivo que creamos lo haríamos como:

  \begin{minted}{bash}
    git add README.md 
    # O equivalentemente 
    git add .
  \end{minted}

  Sin embargo, lo que acabamos de hacer no es suficiente para que los archivos aparezcan en el 
  historial del \textit{VCS}. Lo que hicimos se conoce como \textit{staging}, y es una etapa previa
  a agregar los archivos a \textit{Git}.
  Prueben hacer \texttt{git status} de nuevo y vean el resultado.

  A cada \enquote{versión} almacenada en el historial de \textit{Git} se le llama \textit{commit}.
  Para incluir los archivos de \textit{staging} al sistema de versionado se utiliza el comando 
  \texttt{commit}, este comando debe ir acompañado además de un comentario para que el usuario pueda
  identificar el \textit{commit}.
  Vamos al ejemplo:

  \begin{minted}{bash}
    git commit -m "My first commit :D"
  \end{minted}

  Prueben hacer \texttt{git status} de nuevo.

  Ahora, hay una razón por la cuál hacer \textit{stage} y \textit{commit} son dos procesos 
  separados: \textbf{no es posible deshacer un \textit{commit}}.
  Esto tiene como consecuencia que cualquier archivo o cambio que se haga al repositorio permanecerá
  en el historial y no podrá borrarse de ahí, incluso si el archivo se borra.
  Esto es de especial importancia cuando se guarda información de carácter privado en el directorio
  del \textit{repo}.

  Primero, creemos un archivo \texttt{secret.txt} y digamos que no queremos que ese archivo se 
  agregue a \textit{Git}.
  Veamos dos maneras de evitar que este archivo se agregue al historial.

  La primera es hacer \textit{unstage} del cambio, esto se puede hacer cuando se hace \texttt{add} a
  algún archivo incorrecto, para nuestro ejemplo sería de la siguiente forma:

  \begin{minted}{bash}
    git add .
  \end{minted}

  Esto va a hacer \textit{stage} de todos los archivos.
  Si hacen \texttt{git status} verán que el archivo está esperando a que se le haga \textit{commit}.
  Ahora, para hacer \textit{unstage} haríamos:

  \begin{minted}{bash}
    git rm --cached secret.txt
  \end{minted}

  La otra opción es indicarle a \textit{Git} que ignore ciertos archivos al momento de hacer 
  \textit{stage} (si el archivo ya había sido agregado con \texttt{add} entonces tienen que hacer lo
  indicado antes).
  Para definir los archivos que serán ignorados \textit{Git} utiliza un archivo especial llamado 
  \texttt{.gitignore}, este archivo contiene los nombres de los archivos (o patrones de nombres) que
  se quiere ignorar.
  Para ilustrar esto, primero creen el archivo \texttt{.gitignore} en el repositorio con el 
  siguiente contenido:

  \begin{minted}{bash}
    # .gitignore
    secret.txt
  \end{minted}

  Esto le dice a \textit{Git} que no queremos que se mantengan versiones del archivo 
  \texttt{secret.txt}.
  Ahora, esto no es práctico cuando tenemos muchos archivos, para eso podemos usar patrones de 
  nombres, supongamos que quisiéramos ignorar todos los archivos que contengan la palabra 
  \textit{secret}, esto lo podemos hacer como:

  \begin{minted}{bash}
    # .gitignore
    *secret*
  \end{minted}

  Para detalles sobre como utilizar \textit{gitignore} refiéranse a la documentación oficial de 
  \textit{Git}\autocite{gitignore}.
  Adicionalmente, existen herramientas para generar archivos \texttt{.gitignore} a partir de 
  \textit{templates}, en particular una de las más utilizadas y completas es 
  \url{https://www.gitignore.io}.
%