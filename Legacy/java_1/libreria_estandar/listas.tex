\subsection{Listas}
  Hasta ahora hemos utilizado arreglos para almacenar información, los arreglos tienen la ventaja
  de ser eficientes en cuanto a velocidad y memoria pero la poca flexibilidad que presentan 
  resulta ser poco práctica para la mayoría de los casos.

  El problema de la poca flexibilidad de los arreglos puede solucionarse utilizando una estructura
  de lista.
  Existen muchas formas de implementar listas, una de las implementaciones más comunes de lista es
  la basada en arreglos.
  En \textit{Java} podemos definir esta estructura de la siguiente manera:

  \begin{minted}{java}
    public class ArrayList {
      private Object[] elements = new Object[8];
      private int size = 0;

      public Object get(int i) {
        return elements[i];
      }

      public void add(Object o) {
        if (++size == elements.length) {
          grow();
        }
        elements[size - 1] = o;
      }

      private void grow() {
        Object[] newArray = new Object[elements.length * 2];
        for (int i = 0; i < elements.length; i++) {
          newArray[i] = elements[i];
        }
        elements = newArray;
      }
    }
  \end{minted}

  Esta es una implementación estándar de una estructura de este tipo, ahora, si bien nuestra 
  implementación es correcta en general queremos evitar \enquote{reinventar la rueda} ya que es un
  gasto de tiempo innecesario y es propenso a cometer errores.
  Para evitar esto \textit{Java} implementa una gran cantidad de las estructuras y algoritmos más
  comunes.

  Volviendo al ejemplo de recién, no es necesario que implementemos la clase \texttt{ArrayList} ya
  que es parte de la librería estándar junto con implementaciones de otros tipos de listas.
  
  Las listas en \textit{Java} implementan la interfaz \mintinline{java}{List}.
  Algunos de los métodos que expone esta interfaz son:
  \begin{minted}{java}
    // Retorna el tamaño de la lista
    int size();
    // Pregunta si la lista está vacía
    boolean isEmpty();
    // Pregunta si un elemento está en la lista
    boolean contains(Object o); 
    // Agrega un elemento a la lista
    boolean add(E e);
    // Remueve un elemento de la lista
    boolean remove(Object o);
    // Ordena la lista de acuerdo a su orden natural
    // Por defecto todas las implementaciones de List utilizan MergeSort
    void sort(Comparator<? super E> c);
    // Vacía la lista
    void clear();
    // Obtiene el elemento en la posición index
    E get(int index);
    // Cambia el elemento en la posición index por element
    E set(int index, E element);
    // Retorna el índice del elemento o
    int indexOf(Object o);
  \end{minted}

  Ahora, si queremos crear una lista de enteros podríamos hacerlo como:
  \begin{minted}{java}
    List<Integer> ints = new ArrayList<>();
    for (int i = 0; i < 50; i++) {
      ints.add(i);
    }
  \end{minted}

  Luego, si queremos obtener los valores de la lista lo podemos hacer así:
  \begin{minted}{java}
    for (int i : ints) {
      System.out.println(i)
    }
  \end{minted}

  Noten que al momento de definir el tipo ocupamos la sintaxis \texttt{<Integer>}, esto quiere 
  decir que la lista estará formada por elementos de tipo \texttt{Integer}.\footnote{Las 
  listas en \textit{Java} son homogéneas.}
  El parámetro que va entre \texttt{<...>} debe ser el nombre de una clase o interfaz (\textbf{no 
  puede ser un tipo primitivo}), esto es lo que se conoce como \textit{tipo genérico} y lo 
  revisaremos en más detalle en la sección \ref{sec:generics}.

  Algunas implementaciones de la interfaz de listas que vale la pena mencionar son:
  \begin{itemize}
    \item \texttt{ArrayList}: Lista implementada con arreglos nativos de tamaño dinámico.
    \item \texttt{LinkedList}: Implementación como una lista doblemente enlazada.
  \end{itemize}

  En general lo más común es utilizar un \texttt{ArrayList}, pero dependiendo del caso puede ser
  más conveniente una lista enlazada.\autocite{so-linked-vs-array}