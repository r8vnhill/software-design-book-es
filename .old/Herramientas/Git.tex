% !TEX root = C:\Users\Ignacio\Documents\Escuela\CC3002 - Metodologías de Diseño y Programación\apunte-y-ejercicios\src\latex\Apunte.tex
\chapter{\textit{Git}}
  \textit{Git} es un sistema de control de versiones (\textit{VCS}) distribuido para 
  mantener un historial de los cambios que se realizan en los archivos durante el 
  desarrollo de una aplicación.

  Se creó con el objetivo de manejar las versiones del \textit{kernel} de \textit{Linux}.
  Es un proyecto de código abierto y fue adquiriendo popularidad con los años (según un 
  estudio realizado por \textit{StackOverflow}, \textit{Git} es el sistema de 
  versionado utilizado por el 70 \% de sus usuarios).
  
  Existen muchos otros sistemas de versionado que también se utilizan en la industria,
  en particular \textit{Mercurial}, \textit{Perforce} y \textit{Subversion} son algunos de 
  los más importantes.
  
  \section{Instalación}
    Como \textit{Git} es un proyecto \textit{open-source} existen diversas maneras de 
    instalarlo, en particular aquí se ejemplificarán algunas.

    \subimport{Git/}{Linux.tex}
    \subimport{Git/}{Windows.tex}
    \subimport{Git/}{MacOS.tex}
  %

  \section{Configuración}
    Primero, para comprobar que se haya instalado correctamente, deben ejecutar el 
    comando:

    \begin{minted}{bash}
      git --version
    \end{minted}

    El resultado que debiera retornar este comando es algo del estilo (para el caso de 
    \textit{Windows} instalado con \textit{Chocolatey}):

    \begin{minted}{text}
      git version 2.21.0.windows.1
    \end{minted}

    Dependiendo de la manera en que hayan instalado \textit{Git} es posible que 
    necesiten configurar las credenciales, para esto deben ejecutar los comandos:
    
    \begin{minted}{bash}
      git config --global user.name "Xen-Tao"
      git config --global user.email xentao@depa.na
    \end{minted}

    Reemplazando los datos con su nombre y correo.
    Luego, haciendo \mintinline{bash}{git config -l} verifiquen que su usuario y correo se
    hayan registrado correctamente.
  %

  \subimport{Git/}{Repositorios.tex}
  \subimport{Git/}{github.tex}  
  \subimport{Git/}{branches.tex}
  \subimport{Git/}{markdown.tex}

  %

  \section{Ejercicios}
  %
  \nocite{*}
  \printbibliography[keyword=git]
%