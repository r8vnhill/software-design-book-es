\chapter{Recomendación: \textit{Cmder} para \textit{Windows}}
  La consola por defecto de \textit{Powershell}, si bien es funcional, carece de muchas 
  herramientas para trabajar cómodamente que sí ofrecen las terminales de sistemas 
  \textit{UNIX}.
  Por esta razón recomendamos instalar \textit{Cmder}.

  \textit{Cmder} es un \textit{emulador de consola} que permite correr múltiples 
  \textit{shells} en una misma interfaz, incluyendo \textit{Powershell}, \textit{CMD} y 
  \textit{WSL} entre otras.

  Para instalar \textit{Cmder} utilizaremos \textit{Chocolatey}.
  En una terminal de \textit{Powershell} con permisos de administrador ejecuten:

  \begin{minted}{powershell}
    cinst cmder -y
  \end{minted}

  Con esto basta para tener \textit{Cmder} instalado, pero una de las principales ventajas
  de utilizar este emulador de consola es la capacidad de personalizarlo.

  Lo primero que haremos será descargar las fuentes de \textit{Powerline} que soportan más
  caracteres que los que trae \textit{Powershell} por defecto.
  Para esto ejecuten (puede ser en la misma terminal de \textit{Powershell} o en 
  \textit{Cmder}):

  \begin{minted}{powershell}
    git clone "https://github.com/powerline/fonts.git"
    Set-Location fonts
    .\install.ps1
  \end{minted}

  Ahora, desde \textit{Cmder} vayan a la configuración con \texttt{Win+Alt+P}.
  Una vez ahí, en la pestaña \texttt{General > Fonts} cambien \texttt{Main console font} y 
  \texttt{Alternative font} por alguna de las que provee \textit{Powerline} (por ejemplo 
  \texttt{DejaVu Sans Mono for Powerline}).

  Luego, en la pestaña \texttt{Startup} de las configuraciones, seleccionen en la opción
  \texttt{Specified named task} la terminal por defecto que se ejecutará al abrir 
  \textit{Cmder}.

  Lo siguiente será instalar herramientas que mejorarán la interacción de 
  \textit{Powershell} con \textit{Git} y entregar de mejor forma la información al momento 
  de usar la consola.
  Para esto, deberán ejecutar los siguientes comandos:

  \begin{minted}{powershell}
    Install-PackageProvider NuGet -MinimumVersion '2.8.5.201' -Force
    Set-PSRepository -Name PSGallery -InstallationPolicy Trusted
    Install-Module -Name 'posh-git'
    Install-Module -Name 'oh-my-posh'
    Install-Module -Name 'Get-ChildItemColor'
  \end{minted}

  Lo último es configurar el perfil de \textit{Powershell}, esto se hace en un archivo que
  es el equivalente a \texttt{.bashrc} de los sistemas \textit{Linux}.
  Para abrir este archivo ejecuten:

  \begin{minted}{powershell}
    ise $PROFILE
  \end{minted}

  Esto abrirá el entorno de \textit{scripting} de \textit{Powershell} (si nunca han 
  configurado la consola, entonces debiera estar vacío).
  Como último paso, escriban lo siguiente en el archivo de configuración y guarden los 
  cambios.

  \begin{minted}{powershell}
    # Helper function to set location to the User Profile directory
    function cuserprofile { Set-Location ~ }
    Set-Alias ~ cuserprofile -Option AllScope

    Import-Module 'posh-git'
    Import-Module 'oh-my-posh' -DisableNameChecking

    # CHOCOLATEY PROFILE
    $ChocolateyProfile = "$env:ChocolateyInstall\helpers\chocolateyProfile.psm1"
    if (Test-Path($ChocolateyProfile)) {
      Import-Module "$ChocolateyProfile"
    }

    remove-item alias:cd -force
    function cd($target) {
      if ((Test-Path "$target.lnk") -or $target.EndsWith(".lnk")) {
        if (-not $target.EndsWith(".lnk")) {
          $target = "$target.lnk"
        }
        $sh = New-Object -com wscript.shell
        $fullpath = Resolve-Path $target
        $targetpath = $sh.CreateShortcut($fullpath).TargetPath
        Set-Location $targetpath
      }
      else {
        Set-Location $target
      }
      <#
        .SYNOPSIS
          Moves to the a target directory.
        .DESCRIPTION
          Sets the current location to the given target folder or the folder pointed by its
          symlink.
        .PARAMETER target
          The target directory
      #>
    }

    function U([int]$Code) {
      if ((0 -le $Code) -and ($Code -le 0xFFFF)) {
        return [char] $Code
      }
    
      if ((0x10000 -le $Code) -and ($Code -le 0x10FFFF)) {
        return [char]::ConvertFromUtf32($Code)
      }
    
      throw "Invalid character code $Code"
      <#
      .DESCRIPTION
          Helper function to show Unicode characters
      #>
    }

    Set-PSReadlineOption -BellStyle None
    Set-Theme Honukai
  \end{minted}

  La última línea solamente define el \textit{tema} de la consola, pueden ver una lista de
  \textit{temas} disponibles en el 
  \href{https://github.com/JanDeDobbeleer/oh-my-posh#themes}{repositorio de \textit{Oh-My-Posh}}
%