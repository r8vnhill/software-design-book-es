% !TEX root = D:\Users\Ignacio\Documentos\Escuela\CC3002 - Metodologías de Diseño y Programación\apunte-y-ejercicios\src\latex\Apunte.tex
\subsection{Windows}
  De nuevo, existen muchas maneras de instalar \textit{Git}, a continuación se 
  muestran algunas:

  \subsubsection{Opción 1: \textit{Chocolatey} (Recomendado)}
    Asumiendo que se haya instalado \textit{Chocolatey} como se mostró en la sección
    \ref{sec:java-choco}

    \begin{minted}{powershell}
      cinst git.install -y
    \end{minted}
  %

  \subsubsection{Opción 2: \textit{Git for Windows}}
    \textit{Git for Windows} es un conjunto de herramientas que incluye \textit{Git 
    BASH} (una interfaz de consola que emula la terminal de un sistema \textit{UNIX} 
    que viene con \textit{Git} instalado), \textit{Git GUI} (una interfaz gráfica para 
    manejar \textit{Git}) e integración con \textit{Windows Explorer} (esto significa 
    que pueden hacer \textit{clic} derecho en una carpeta y abrirla desde \textit{Git 
    BASH} o \textit{Git GUI}).
    
    Para instalarla deben descargar el cliente desde el 
    \href{https://gitforwindows.org}{sitio oficial} de \textit{Git for Windows} y 
    seguir las instrucciones del instalador.
  %
%