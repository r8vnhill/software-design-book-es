\section{Repositorios remotos}
  \label{sec:github}

  Como mencionamos antes, uno de los principales beneficios es poder utilizar \textit{Git} de manera
  remota.
  Para esto se necesita utilizar un servidor, y si bien es posible tener el sistema de versionado
  montado en un servidor propio existen proveedores que ofrecen este servicio y facilitan el 
  trabajo.

  Existen varias opciones\footnote{Además de \textit{GitHub} están \textit{GitLab}, 
  \textit{BitBucket}, \textit{Azure} y otros.} cuando se quiere utilizar un servidor, pero la más 
  utilizada es por lejos \textit{GitHub}.\autocite{vcs-providers}

  \textit{GitHub} provee muchas funcionalidades pero para este curso utilizaremos solamente las más
  básicas.

  Para detalles sobre cómo utilizar \textit{GitHub} revisen las bibliografías incluidas al final de
  este capítulo.

  Sin importar el proveedor de \textit{Git} que se use los comandos para manejar los repositorios 
  remotos serán siempre los mismos.

  Primero veamos cómo \enquote{descargar} un repositorio desde el servidor.
  A la acción de crear una copia del repositorio remoto para utilizarlo como un repositorio local se
  le conoce como \textit{clonar}.
  Para esto se utiliza el comando \texttt{clone} como se muestra a continuación:
  \begin{minted}{bash}
    git clone https://github.com/octocat/Spoon-Knife.git
  \end{minted}

  Ejecutar la instrucción anterior creará una carpeta \texttt{Spoon-Knife} que contendrá todos los
  contenidos del repositorio (incluyendo todas las versiones del repositorio remoto, esto quiere 
  decir que no necesitan hacer \texttt{git init}).

  Una vez que clonaron el repositorio es posible que se hagan cambios en el repositorio remoto y 
  necesiten actualizar la versión local, para hacer esto se utiliza el comando \texttt{pull}.
  Por ejemplo, siguiendo el ejemplo anterior haríamos:
  \begin{minted}{bash}
    cd Spoon-Knife
    git pull
  \end{minted}

  El último comando que será importante al momento de trabajar con repositorios remotos será 
  \texttt{push}, esta instrucción subirá los \textit{commits} realizados en su repositorio local al
  repositorio remoto.
  Es importante \textbf{siempre hacer \texttt{pull}} antes de hacer \texttt{push} para evitar 
  conflictos de versiones.
  La sintaxis es la misma que la de \texttt{pull}.
