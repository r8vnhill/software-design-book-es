%!TEX root = C:\Users\Ignacio\Documents\Escuela\CC3002 - Metodologías de Diseño y Programación\apunte-y-ejercicios\src\latex\Apunte.tex
\section{\textit{Markdown}}
  \textit{Markdown} es un estándar liviano que apunta a ser fácil de usar para crear documentos con
  formato básico.
  Es similar a \textit{HTML} pero más simple y conciso.
  En su mayoría, los archivos \textit{Markdown} son documentos de texto plano con algunos caracteres
  especiales para definir el estilo del texto.

  En \textit{GitHub} (y varios otros servidores de \textit{Git}) \textit{Markdown} es el formato 
  estándar de documentación.

  Para crear un archivo \textit{Markdown} basta con que el nombre del archivo termine en 
  \texttt{.markdown} o \texttt{.md}.

  Para una referencia de cómo usar \textit{Markdown} refiérase a la documentación de 
  \textit{GitHub}\autocite{gh-markdown}.
    