% !TEX root = D:\Users\Ignacio\Documentos\Escuela\CC3002 - Metodologías de Diseño y Programación\apunte-y-ejercicios\src\latex\Apunte.tex
\section{Linux (x64)}
  \subsection{Opción 1: \textit{Open JDK} (Recomendado)}
    Para cualquier distribución de \textit{Linux x64}, desde la terminal:

    \begin{minted}{bash}
      wget https://bit.ly/344NYw9
      tar xvf openjdk-14*_bin.tar.gz
      sudo mv jdk-14 /usr/lib/jdk-14
    \end{minted}

    Luego, para verificar que el binario se haya extraído correctamente:

    \begin{minted}{bash}
      export PATH=$PATH:/usr/lib/jdk-14/bin
      java -version
    \end{minted}

    Si el binario se instaló correctamente, este comando debiera retornar algo como:

    \begin{minted}{text}
      openjdk version "14" 2020-03-17
      OpenJDK Runtime Environment (build 14+36-1461)
      OpenJDK 64-Bit Server VM (build 14+36-1461, mixed mode, sharing)
    \end{minted}

    En caso de que no haya resultado, intenten realizar la instalación nuevamente o 
    procedan a la siguiente opción.

    Si se instaló correctamente, entonces el último paso es agregar el \textit{JDK} a 
    las variables de entorno del usuario, para esto primero deben saber qué 
    \textit{shell} están ejecutando, pueden ver esto con:

    \begin{minted}{bash}
      echo $SHELL
    \end{minted}

    En mi caso, esto retorna:

    \begin{minted}{text}
      /usr/bin/zsh
    \end{minted}

    Luego, deben editar el archivo de configuración de su \textit{shell}, en mi caso ese
    sería \mintinline{bash}{~/.zshrc} (en \textit{bash} sería \texttt{.bashrc}) y 
    agregar al final del archivo la línea:
    
    \begin{minted}{bash}
      export PATH=$PATH:/usr/lib/jdk-14/bin
    \end{minted}
  %

  \subsection{Opción 2: \textit{Oracle JDK}}
    Primero deben descargar el \textit{JDK} desde el
    \href{https://www.oracle.com/java/technologies/javase-jdk14-downloads.html}{sitio de 
    \textit{Oracle}} (asumiremos que descargaron la versión \texttt{.tar.gz}).
    Luego, desde el directorio donde descargaron el archivo:

    \begin{minted}{bash}
      tar zxvf jdk-14.interim.update.patch_linux-x64_bin.tar.gz
      sudo mv jdk-14.interim.update.patch /usr/lib/jdk-14.interim.update.patch
    \end{minted}

    Después, de la misma forma que se hizo con la opción 1:
    
    \begin{minted}{bash}
      export PATH=$PATH:/usr/lib/jdk-14.interim.update.patch/bin
      java -version
    \end{minted}

    Si este comando funciona, entonces deberán modificar el archivo de configuración de su
    \textit{shell} para incluir la línea:

    \begin{minted}{bash}
      export PATH=$PATH:/usr/lib/jdk-14.interim.update.patch/bin
    \end{minted}
  %
%