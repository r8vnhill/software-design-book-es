%!TEX root = C:\Users\smfla\Documents\Escuela\CC3002 - Metodologías de Diseño y Programación\apunte-y-ejercicios\src\latex\Apunte.tex
\section{Modelo-Vista-Controlador}
  Para explicar lo que es el \textit{Modelo-Vista-Controlador} (o \textit{MVC}) primero debemos 
  familiarizarnos con los conceptos \textit{arquitectura de software} y \textit{patrón 
  arquitectónico}.

  \textit{Arquitectura de software} es un concepto complejo.
  Definir la arquitectura de una aplicación es uno de los procesos más importantes y difíciles 
  dentro del desarrollo de un proyecto, y no existe un acuerdo respecto a \textbf{qué entendemos por 
  arquitectura}.
  En palabras de \textit{Martin Fowler} (citando a \textit{Ralph Johnson}), la arquitectura son 
  \enquote{las decisiones que desearíamos haber tomado bien al principio de un 
  proyecto}.\autocite{fowler-software-architecture}
  Esta definición es algo esotérica pero es debido a la dimensión del problema de definir la 
  arquitectura de nuestro programa, intentando simplificar, podríamos decir que la arquitectura de
  software es la forma en la que se estructuran, comunican y relacionan todas las componentes de 
  nuestro programa.