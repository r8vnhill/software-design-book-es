% !TEX root = C:\Users\Ignacio\Documents\Escuela\CC3002 - Metodologías de Diseño y Programación\apunte-y-ejercicios\src\latex\Apunte.tex
\section{La librería estándar}
  En el curso de algoritmos y estructuras de datos vieron cómo implementar varias estructuras de 
  datos para resolver problemas comunes, tener una idea general de cómo se implementan y los 
  beneficios y costos de éstas es crucial para el ámbito de programación.
  Dicho esto, la mayoría de los lenguajes de programación tienen implementadas las estructuras más
  usuales dentro de sus librerías estándar, y \textit{Java} no es la excepción\footnote{\textit{C} 
  es un ejemplo de lenguaje que no tiene implementadas estas estructuras.}.

  La librería estándar de \textit{Java} es demasiado amplia como para cubrirla en este apunte por lo
  que sólo veremos los contenidos que nos serán de utilidad, en particular listas y diccionarios.
  
  \subimport{libreria_estandar/}{listas.tex}
  \subimport{libreria_estandar/}{maps.tex}

  TODO:
  \begin{itemize}
    \item Sets
    \item Colas
  \end{itemize}