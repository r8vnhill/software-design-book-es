% !TEX root = C:\Users\smfla\Documents\Escuela\CC3002 - Metodologías de Diseño y Programación\apunte-y-ejercicios\src\latex\Apunte.tex
\subsection{Constructores y herencia}
  Ya hemos mencionado anteriormente cómo la herencia es una de las propiedades más poderosas de la
  programación orientada a objetos, entonces una pregunta natural sería cómo funcionan los 
  constructores cuando hay herencia.

  Tomemos como punto de partida el ejemplo de vectores que vimos en el capítulo 
  \ref{ch:python-java}.

  Primero modifiquemos un poco la clase \texttt{VectorND} para integrar los modificadores de 
  privacidad que vimos en la sección anterior:

  \begin{minted}{java}
    public class VectorND {
      protected double[] tail;

      public VectorND(double[] tail) {
        this.tail = tail;
      }

      public VectorND add(VectorND otherVector) {
        // Igual que antes
      }

      public void print() {
        // Igual que antes
      }
    }
  \end{minted}
  
  Ahora, también habíamos definido una clase \texttt{Vector2D} que heredaba de \texttt{VectorND},
  modifiquemos también esta clase para restringir su visibilidad.

  \begin{minted}{java}
    public class Vector2D extends VectorND {
      public Vector2D(double x, double y) {
        super(new double[]{x, y});
      }
    }
  \end{minted}

  Noten que el constructor de \texttt{Vector2D} hace una llamada a \mintinline{java}{super(...)}, 
  es importante que no confundan este mensaje con la pseudo-variable \mintinline{java}{super}.
  Lo que hace este mensaje es enviar un mensaje a la superclase para que ésta ejecute el constructor
  que reciba los parámetros que se le pasaron (en este caso \mintinline{java}{VectorND(double[])}).

  Una pregunta que podría surgirnos ahora es la necesidad de llamar explícitamente al constructor
  de la clase padre.
  ¿No podríamos simplemente hacer lo siguiente?

  \begin{minted}{java}
    public Vector2D(double x, double y) {
      this.tail = new double[]{x, y};
    }
  \end{minted}
  
  Si intentan correr el código anterior se van a dar cuenta que no compila, la razón de esto es 
  similar a lo que sucedía si no declarábamos explícitamente un constructor en la clase, por defecto
  todos los constructores hacen una llamada implícita al constructor de su clase padre sin 
  argumentos a no ser que se llame explícitamente a \mintinline{java}{super(...)}.
  Esto quiere decir que el código del ejemplo anterior es equivalente a:

  \begin{minted}{java}
    public Vector2D(double x, double y) {
      super();
      this.tail = new double[]{x, y};
    }
  \end{minted}

  Luego, como la clase \texttt{VectorND} no tiene ningún constructor que no reciba argumentos 
  entonces el código no compila.

  Otra pregunta interesante es si los constructores se heredan.
  Para explicar esto partamos por un ejemplo.
  La velocidad se puede definir como un vector dentro de un espacio euclidiano de \(n\) dimensiones,
  luego, la velocidad es equivalente a la clase \texttt{VectorND} que ya definimos.
  Podríamos definir una clase para la velocidad que sea una especialización de un vector.
  
  \begin{minted}{java}
    public class Velocity extends VectorND {
    }
  \end{minted}

  Si los constructores se heredan entonces podríamos crear un objeto velocidad utilizando el 
  constructor de la superclase de la forma:

  \begin{minted}{java}
    Velocity v = new Velocity(new double[]{0, 1, 2, 3});
  \end{minted}
  
  Nuevamente, si intentan correr ese código el programa no va a compilar.
  Esto se debe a que en \textit{Java}\footnote{Hay lenguajes que \textbf{sí tienen} herencia de 
  constructores (e.g. \textit{Python}).} los constructores \textbf{no se heredan}.
  Es muy importante tener esto en cuenta ya que un error común es confundir la llamada implícita a
  \mintinline{java}{super()} con herencia de constructores.

  \begin{exercise}
    ¿Se puede enviar el mensaje \mintinline{java}{super(...)} desde fuera del constructor de la 
    clase?
  \end{exercise}

  \begin{exercise}
    ¿Se le ocurre una razón por la cuál en \textit{Java} no se tiene herencia de constructores?
    \textit{Hint:} Piense en la definición de constructor que se dio en la sección 
    \ref{sec:default-constructor}.
  \end{exercise}