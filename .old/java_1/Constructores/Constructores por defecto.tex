\subsection{Constructores vacíos y por defecto}
  \label{sec:default-constructor}
  Anteriormente, cuando hablamos de constructores (refiérase a la sección 
  \ref{sec:constructors-1}) dijimos que eran métodos especiales que se llamaban al crear una 
  instancia de una clase, esto es parcialmente cierto.
  En \textit{Java} todos los métodos retornan algo, incluso los marcados como 
  \mintinline{java}{void} retornan implícitamente un valor que no se puede utilizar y por lo tanto
  es desechado (el detalle del manejo de memoria en \textit{Java} se verá en la sección 
  \ref{sec:memory}), para ejemplificar esto, noten que los siguientes métodos son equivalentes:
  
  \begin{minted}{java}
    public void method1() {}
    public void method2() {
      return;
    }  
  \end{minted}

  ¿Qué es entonces un constructor?
  Un constructor es un \textbf{bloque de código} que se ejecuta al momento de crear una nueva 
  instancia de una clase.
  ¿Por qué es importante hacer notar esta diferencia?
  Volveremos a esto un poco más adelante.

  Primero, veamos cómo se define un constructor.
  La firma de un constructor es de la forma \texttt{[modifier] ClassName([parameters]) \{[body]\}}
  Vayamos a un ejemplo, volvamos a la clase \texttt{Bakemon} que habíamos creado anteriormente y 
  agreguémosle un constructor:
  
  \begin{minted}{java}
    public class Bakemon {
      public Bakemon() {
      }
    }
  \end{minted}

  Nuestro constructor está vacío, eso significa que lo único que hace el constructor es reservar la 
  memoria necesaria para las instancias de ésta clase.
  Pero ya habíamos visto antes que podíamos crear instancias de la clase sin necesidad de definir un
  constructor, esto se debe a que \textit{Java} crea un constructor por defecto si no existe ninguno
  en la clase.
  Lo anterior se traduce en que el código que hicimos sea equivalente a:

  \begin{minted}{java}
    public class Bakemon {
    }
  \end{minted}

  Agreguemos algo de funcionalidad al constructor.
  Consideren que queremos agregarle un tipo a nuestros \textit{Bakemon}, y digamos por ahora que 
  comenzarán sin tipo (y representemos esto como el \textit{String} \mintinline{java}{"NONE"}).

  \begin{minted}{java}
    public class Bakemon {
      private String type;

      public Bakemon() {
        type = "NONE";
      }
    }
  \end{minted}

  Como estamos definiendo siempre la propiedad \texttt{type} con el mismo valor \textit{Java} provee
  una forma más compacta de escribir esto mismo como:

  \begin{minted}{java}
    public class Bakemon {
      private String type = "NONE";
    }
  \end{minted}