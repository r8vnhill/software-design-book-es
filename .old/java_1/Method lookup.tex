\section{\textit{Method lookup}}
  \label{sec:lookup}

  Cuando un objeto recibe un mensaje, éste debe ser interpretado de alguna forma, pero para esto se 
  necesita alguna forma de saber cómo responder al mensaje.
  El proceso de buscar la respuesta adecuada para un mensaje se conoce como \textit{method lookup}.

  En \textit{Java} (y el resto de los lenguajes de tipado estático) todos los mensajes 
  \textbf{deben} tener una forma de ser interpretados para que el programa compile, pero en el caso
  de los lenguajes de tipado dinámico como \textit{Python} esto no necesariamente es así.

  Al recibir un mensaje, el objeto que lo recibe revisa si tiene algún método para responderlo; si
  encuentra el método entonces lo ejecuta, sino lo busca en su clase padre.
  En caso de que no encuentre el método entonces no puede responder el mensaje y por lo tanto arroja
  un error, ya sea al compilarse o en tiempo de ejecución dependiendo del lenguaje.

  Tomemos como ejemplo las clases \texttt{Vector2D} y \texttt{VectorND} que vimos en las secciones
  anteriores.
  \texttt{Vector2D} solamente cuenta con un constructor y no tiene ningún método propio, por lo que
  cualquier mensaje que reciba no podrá ser interpretado directamente.
  Ahora, si enviamos el mensaje \texttt{add(VectorND)} a \texttt{Vector2D} lo va a buscar en dicha 
  clase y no lo va a encontrar, como no lo encuentra se hace delegación\autocite{wp-delegation} para
  que la clase padre interprete el mensaje.
  En este caso, el método se encuentra en la clase \texttt{VectorND} así que se responde
  exitosamente el mensaje.
  Por otro lado, si enviáramos el mensaje \texttt{substract(VectorND)}, buscaría en 
  \texttt{Vector2D} sin encontrarlo, luego buscaría en \texttt{VectorND} y, por último, en la clase
  \texttt{Object}; al no encontrar el método no se puede responder al mensaje y ocurre un error de
  compilación.

  \begin{exercise}
    ¿Qué ocurre si enviamos el mensaje \texttt{equals(VectorND)} a un objeto de la clase 
    \texttt{Vector2D}?
  \end{exercise}


  %% region : this/super
  \subsection{\mintinline{java}{this} y \mintinline{java}{super}}
    Todos los objetos de \textit{Java} poseen dos \textit{pseudo-variables} para referenciar al 
    mismo objeto.
  % endregion