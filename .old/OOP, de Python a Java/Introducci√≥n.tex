% !TEX root = D:\Users\Ignacio\Documentos\Escuela\CC3002 - Metodologías de Diseño y Programación\apunte-y-ejercicios\src\latex\Apunte.tex
\section{Lo básico}
  Consideremos un ejemplo básico para comenzar: queremos imprimir un mensaje en consola.

  Podríamos hacer el clásico \textit{Hello world!}, pero que fome, en cambio imprimamos
  otro mensaje.

  Creen un archivo \texttt{ejemplo\_basico.py} y ábranlo con 
  Para imprimir un texto en consola en \textit{Python} haríamos:

  \begin{minted}{python}
    print("My name is Giorno Giovanna, and I have a dream.")      
  \end{minted}

  O lo que sería más correcto de acuerdo a las convenciones de \textit{Python}:

  \begin{minted}{python}
    if __name__ == "__main__":
        print("My name is Giorno Giovanna, and I have a dream.")
  \end{minted}

  Luego, si queremos ejecutar el script haríamos:

  \begin{minted}{bash}
    python3 ejemplo_basico.py
  \end{minted}

  o en caso de utilizar \textit{Windows}:

  \begin{minted}{powershell}
    py -3 ejemplo_basico.py
  \end{minted}

  En \textit{Java} reproducir este mismo ejemplo es un tanto más complicado ya que 
  necesitaremos escribir más líneas de código.
  Para crear un programa equivalente en \textit{Java}, primero crearemos un archivo 
  \texttt{EjemploBasico.java}, luego en el editor de texto que prefieran escriban el 
  código:
  \begin{listing}[ht!]
    \begin{minted}{java}
      public class EjemploBasico {

        public static void main(String[] args) {
          System.out.println("My name is Giorno Giovanna, and I have a dream.");
        }
      }
    \end{minted}
  \end{listing}

  Veremos en detalle las diferencias entre la sintaxis de ambos ejemplos, pero primero
  veamos como ejecutar el programa para ver que efectivamente hace lo mismo que el de 
  \textit{Python}, para esto deben ejecutar en consola:

  \begin{minted}{bash}
    javac EjemploBasico.java
    java EjemploBasico
  \end{minted}

  El primer comando creará un archivo \texttt{EjemploBasico.class}, este es un archivo
  pre-compilado (es importante que en las tareas \textbf{NO ENTREGUEN} los archivos 
  \texttt{.class}, ya que no los podemos revisar), luego el segundo comando compila y 
  ejecuta el programa.
  Esto se explicará en el capítulo \ref{ch:java-2}.

  Ahora, veamos las diferencias entre ambos programas.
  El código en \textit{Python} es bastante fácil de seguir.
  ¿Pero por qué en \textit{Java} hay que definir tantas cosas solamente para imprimir un
  mensaje en consola?

  Vamos por partes, lo primero que deben notar es que la línea con el llamado a 
  \texttt{println(...)} termina con un \texttt{;}, esto debe ser así para todas las 
  instrucciones, esto puede no parecer tan importante a primera vista, pero marca una
  diferencia enorme respecto a \textit{Python}, ya que a diferencia de \textit{Python} 
  la indentación no es importante.
  Cuando programamos en \textit{Python} la indentación es lo que define dónde comienza
  y termina una definición, en \textit{Java} en cambio, esto se define entre llaves, 
  donde la apertura de una marca el inicio y el cierre el fin.

  Luego, tenemos la definición \mintinline{java}{public static void main(String[] args)}
  este es un método especial que será el punto de entrada del programa, por lo que al 
  ejecutar el código se ejecutarán todas las instrucciones definidas dentro del método.
  
  Por último, tenemos que todo esto va dentro de la definición de una clase 
  \texttt{EjemploBasico}, esto es necesario ya que \textit{Java} es un lenguaje 
  (casi) totalmente orientado a objetos \textit{fuertemente tipado}.
  De momento basta que sepan que los programas siguen esa sintaxis, en el capítulo 
  \ref{ch:java} veremos más en detalle el funcionamiento de \textit{Java} y 
  profundizaremos en este tema.
%