% !TEX root = D:\Users\Ignacio\Documentos\Escuela\CC3002 - Metodologías de Diseño y Programación\apunte-y-ejercicios\src\latex\Apunte.tex
\section{Tipos}
  Tanto \textit{Python} como \textit{Java} son lenguajes orientados a objetos, esto quiere
  decir que todas las variables que se utilicen en un programa son objetos o tipos 
  primitivos.
  La diferencia está en que, al ser fuertemente tipado, en \textit{Java} es necesario 
  definir explícitamente el tipo de todas las variables.
  Esto hace que en \textit{Java} todo deba definirse dentro de una clase.

  \subsection{Tipos primitivos}
    Los \textit{tipos primitivos} son datos que ocupan una cantidad fija de espacio en la
    memoria.
    Esto hace que sean más eficientes de utilizar ya que una vez que se les asigna un 
    lugar en la memoria difícilmente tendrán que moverse a otra dirección (algo que 
    sucederá mucho con los objetos).
    
    La tabla \ref{tab:primitive} muestra los tipos primitivos de \textit{Java} (en 
    \textit{Python} la definición de estos es más complicada, así que se omitirá).

    \begin{table}[ht!]
      \centering
      \begin{tabular}{| c | m{5em} | m{6em} | m{12em} | c |}
        \hline
        Tipo                      & Uso de memoria  & Rango 
          & Uso                                               
          & Valor por defecto           \\
        \hline\hline
        \mintinline{java}{byte}     & 8-bits          & [-128, 127]  
          & Representar arreglos masivos de números pequeños  
          & \mintinline{java}{0}        \\\hline
        \mintinline{java}{short}    & 16-bits         & [-32.768, 32.767]
          & Representar arreglos grandes de números pequeños  
          & \mintinline{java}{0}        \\\hline
        \mintinline{java}{char}     & 16-bits         & [0, 65.535]
          & Caracteres del estándar \textit{ASCII}            
          & \mintinline{java}{'\u0000'} \\\hline
        \mintinline{java}{int}      & 32-bits         & \([-2^{31}, 2^{31} - 1]\)
          & Estándar para representar enteros               
          & \mintinline{java}{0}        \\\hline
        \mintinline{java}{long}     & 64-bit          & \([-2^{63}, 2^{63} - 1]\)
          & Representar enteros que no quepan en 32-bits
          & \mintinline{java}{0L}       \\\hline
        \mintinline{java}{float}    & 32-bit          & Estándar \textit{IEEE 754x32}
          & Representar números reales cuando la precisión no es importante
          & \mintinline{java}{0.0f}     \\\hline
        \mintinline{java}{double}   & 64-bit          & Estándar \textit{IEEE 754x64}
          & Representar números reales con mediana precisión
          & \mintinline{java}{0.0}      \\\hline
        \mintinline{java}{boolean}  & No definido     & \mintinline{java}{true| o |false}
          & Valores binarios
          & \mintinline{java}{false}    \\
        \hline
      \end{tabular}
      \caption{
        Tipos primitivos en \textit{Java} (los valores por defecto son los que toma la
        variable si no se le entrega un valor explícitamente \textbf{y es un campo} de una
        clase)
      }
      \label{tab:primitive}
    \end{table}

    Todos los tipos primitivos en \textit{Java} tienen un objeto \enquote{equivalente} 
    para brindar operaciones que no se pueden realizar directamente con los tipos 
    primitivos, e.g. utilizarlos como tipos genéricos (esto se verá en la última parte del 
    apunte).

    Noten que la sintaxis en \textit{Java} para los valores \textit{boolean} es 
    \mintinline{java}{true| y |false}, mientras que en \textit{Python} es 
    \mintinline{python}{True| y |False}.

    \begin{important}
      Los \texttt{String} no son tipos primitivos.
    \end{important}
  %

  \subsection{Objetos y clases}
    Como mencionamos anteriormente, un objeto es una instancia de una clase, por lo que para
    definir un objeto primero debe definirse la clase a la que pertenece.
    En ambos lenguajes, las clases se definen con la \textit{keyword} 
    \mintinline{java}{class}.
    Para ver la sintaxis en ambos casos comencemos con un ejemplo simple, crear una clase 
    que represente un punto en 2 dimensiones.

    En \textit{Python} esto se puede lograr de la siguiente forma:

    \begin{listing}[ht!]
      \begin{minted}{python}
        class Vector2D:
            def __init__(self, x, y):
                self.x = x
                self.y = y
      \end{minted}
    \end{listing}

    Luego, se pueden crear y utilizar instancias de esta clase como en el siguiente 
    ejemplo:

    \begin{listing}[ht!]
      \begin{minted}{python}
        point = Vector2D(1, 3)
        print(point.x)
        print(point.y)
      \end{minted}
    \end{listing}

    Y en \textit{Java}:
    \begin{listing}[ht!]
      \begin{minted}{java}
        class Vector2D {
          double x, y;

          Vector2D(double x, double y) {
            this.x = x;
            this.y = y;
          }
        }
      \end{minted}
    \end{listing}

    Para crear instancias de una clase en \textit{Java} se utiliza la \textit{keyword} 
    \mintinline{java}{new}, de esta forma se puede hacer:

    \begin{minted}{java}
      Vector2D point = new Vector2D(1, 3);
      System.out.println(point.x);
      System.out.println(point.y);
    \end{minted}

    Tanto \mintinline{python}{__init__} como \mintinline{java}{Vector2D(double, double)}
    son los constructores de la clase y se explicarán en detalle en la siguiente sección.

    \subsubsection{Métodos}
      \label{sec:methods}
      En el capítulo anterior se explicó como, además de almacenar datos, los objetos 
      pueden realizar acciones mediante métodos, en \textit{Python} un método se define 
      como \mintinline{python}{def method_name(param1, param2, ...): ...}, en 
      \textit{Java} la sintaxis es un poco más tediosa ya que se debe explicitar el tipo 
      de los parámetros que recibe el método y el tipo del dato que retorna de la forma
      \mintinline{java}{returnType methodName(param1Type param1, param2Type param2, ...) {...}}.

      Agreguemos 2 métodos a la clase \texttt{Vector2D}, uno que sume las coordenadas de 2
      puntos retornando un nuevo punto, y otro que imprima un punto en pantalla como 
      \texttt{(x, y)}.

      En \textit{Python}:
      \begin{listing}[ht!]
        \begin{minted}{python}
          # Dentro de la clase Vector2D
          def add(self, other_point):
              return Vector2D(self.x + other_point.x,
                             self.y + other_point.y)
          
          def print(self):
              print(f"({self.x}, {self.y})")
        \end{minted}
      \end{listing}

      Y se pueden utilizar estos métodos de la forma:

      \begin{listing}[ht!]
        \begin{minted}{python}
          point = Vector2D(1, 3)
          new_point = point.add(Vector2D(-1, 2))
          new_point.print()
        \end{minted}
      \end{listing}

      En el caso de \textit{Java} debemos especificar el tipo de retorno o, en el caso de 
      que no retorne nada, \mintinline{java}{void}.
      Entonces:

      \begin{listing}[ht!]
        \begin{minted}{java}
          // Dentro de la clase Vector2D
          Vector2D add(Vector2D otherPoint) {
            return new Vector2D(x + otherPoint.x, y + otherPoint.y);
          }

          void print() {
            System.out.println("(" + x + ", " + y + ")");
          }
        \end{minted}
      \end{listing}

      Y luego puede utilizarse de la forma:

      \begin{listing}[ht!]
        \begin{minted}{java}
          Vector2D point = new Vector2D(1, 3);
          Vector2D newPoint = point.add(new Vector2D(-1, 2));
          newPoint.print();
        \end{minted}
      \end{listing}

      \begin{note}
        En el caso de \textit{Python}, los métodos reciben un parámetro 
        \mintinline{python}{self} que se le entrega implícitamente al momento de invocar 
        la función (en el curso \textit{CC4101 - Lenguajes de programación} se explica la 
        razón detrás de esto).

        \textit{Java} por otro lado no necesita de ese parámetro, por lo que basta que no
        se declaren parámetros en la definición del método para indicar que éste no recibe
        parámetros.
        
        Un caso en particular a tener en cuenta es el lenguaje \textit{C}, donde si no se 
        especifican los parámetros que recibe una función se interpreta como que puede 
        recibir cualquier número y tipo de parámetros.
        Si se quiere que efectivamente no se reciban parámetros se debe especificar 
        explícitamente de la forma \mintinline{c}{returnType function(void) {...}}.
      \end{note}

      \begin{exercise}
        Cree un método \textit{printPolar()} que imprima el punto en coordenadas polares 
        de la forma \texttt{(r, t)}, donde:
        \[
          \begin{aligned}
            r &= \sqrt{x^2 + y^2} \\
            t &= \begin{cases}
              \arctan\left(\frac{y}{x}\right)       
                & \mbox{si } x > 0                      \\
              \arctan\left(\frac{y}{x}\right) + \pi 
                & \mbox{si } x < 0 \wedge y \ge 0 \\
              \arctan\left(\frac{y}{x}\right) - \pi 
                & \mbox{si } x < 0 \wedge y < 0   \\
              \frac{\pi}{2}                         
                & \mbox{si } x = 0 \wedge y > 0   \\
              -\frac{\pi}{2}                         
                & \mbox{si } x = 0 \wedge y < 0   \\
              \text{indefinido}                     
                & \mbox{si } x = 0 \wedge y = 0
            \end{cases}
          \end{aligned}
        \]

        Para esto, ocupe los métodos 
        \mintinline{java}{Math.sqrt(double)| y | Math.atan(double)}, y el valor
        \mintinline{java}{Math.PI}.
      \end{exercise}
    %
  %
%