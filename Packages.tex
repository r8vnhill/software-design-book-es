\usepackage{fontspec}
\usepackage{amsmath}
\usepackage{amsthm}
\usepackage{amssymb}
\usepackage[spanish]{babel}
\usepackage{authblk}
\usepackage{fullpage}
\usepackage[hyphens]{url}
\usepackage{hyperref}
\usepackage{graphicx}
% | The package starts from the basic facilities of the color package, and provides easy 
% | driver-independent access to several kinds of color tints, shades, tones, and mixes of arbitrary 
% | colors. 
% | It allows a user to select a document-wide target color model and offers complete tools for 
% | conversion between eight color models. 
% | Additionally, there is a command for alternating row colors plus repeated non-aligned material 
% | (like horizontal lines) in tables. Colors can be mixed like \color{red!30!green!40!blue}.
\usepackage[dvipsnames]{xcolor}
\usepackage{minted}
\usepackage{parskip}
\usepackage[stable]{footmisc}
\usepackage{import}
\usepackage{exercise}
\usepackage{array}
\usepackage{csquotes}
\usepackage[style=reading, backend=biber, refsection=chapter, citereset=chapter]{biblatex}
% | The epigraph package can be used to typeset a relevant quotation
% | or saying as an epigraph, usually just after a sectional heading.
% | Various handles are provided to tweak the appearance.
\usepackage{epigraph}
% | Extended List Environments
\usepackage{paralist}
\usepackage{imakeidx}
% | This package provides an environment for coloured and framed text boxes with a heading line. 
% | Optionally, such a box may be split in an upper and a lower part; thus the package may be used 
% | for the setting of LATEX examples where one part of the box displays the source code and the 
% | other part shows the output. 
% | Another common use case is the setting of theorems. 
% | The package supports saving and reuse of source code and text parts.
\usepackage[skins, minted, breakable]{tcolorbox}
