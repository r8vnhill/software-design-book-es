\section{\textit{Markdown}}
  \textit{Markdown} es un estándar liviano que apunta a ser fácil de usar para crear documentos con
  formato básico.
  Es similar a \textit{HTML} pero más simple y conciso.
  En su mayoría, los archivos \textit{Markdown} son documentos de texto plano con algunos caracteres
  especiales para definir el estilo del texto.

  En \textit{GitHub} (y varios otros servidores de \textit{Git}) \textit{Markdown} es el formato 
  estándar de documentación.

  Para crear un archivo \textit{Markdown} basta con que el nombre del archivo termine en 
  \texttt{.markdown} o \texttt{.md}.

  \subsection{Sintaxis}
    \paragraph{Encabezados}
      \begin{minted}{md}
        # This is an <h1> tag
        ## This is an <h2> tag
        ###### This is an <h6> tag
      \end{minted}