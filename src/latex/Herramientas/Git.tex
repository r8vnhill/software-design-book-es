\chapter{\textit{Git}}
  \textit{Git} es un sistema de control de versiones (\textit{VCS}) distribuido para 
  mantener un historial de los cambios que se realizan en los archivos durante el 
  desarrollo de una aplicación.

  Se creó con el objetivo de manejar las versiones del \textit{kernel} de \textit{Linux}.
  Es un proyecto de código abierto y fue adquiriendo popularidad con los años (según un 
  estudio realizado por \textit{StackOverflow}, \textit{Git} es el sistema de 
  versionado utilizado por el 70 \% de sus usuarios).
  
  Existen muchos otros sistemas de versionado que también se utilizan en la industria,
  en particular \textit{Mercurial}, \textit{Perforce} y \textit{Subversion} son algunos de 
  los más importantes.
  
  \section{Instalación}
    Como \textit{Git} es un proyecto \textit{open-source} existen diversas maneras de 
    instalarlo, en particular aquí se ejemplificarán algunas.

    \subimport{Git/}{Linux.tex}
    \subimport{Git/}{Windows.tex}
    \subimport{Git/}{MacOS.tex}
  %

  \section{Configuración}
    Primero, para comprobar que se haya instalado correctamente, deben ejecutar el 
    comando:

    \begin{minted}{bash}
      git --version
    \end{minted}

    El resultado que debiera retornar este comando es algo del estilo (para el caso de 
    \textit{Windows} instalado con \textit{Chocolatey}):

    \begin{minted}{text}
      git version 2.21.0.windows.1
    \end{minted}

    Dependiendo de la manera en que hayan instalado \textit{Git} es posible que 
    necesiten configurar las credenciales, para esto deben ejecutar los comandos:
    
    \begin{minted}{bash}
      git config --global user.name "Xen-Tao"
      git config --global user.email xentao@depa.na
    \end{minted}

    Reemplazando los datos con su nombre y correo.
    Luego, haciendo \mintinline{bash}{git config -l} verifiquen que su usuario y correo se
    hayan registrado correctamente.
  %

  \subimport{Git/}{Repositorios.tex}

  \section{\textit{GitHub}}
    \label{sec:github}
  %

  \section{Ejercicios}
  %
  \section{Conceptos clave}
    \begin{itemize}
      \item \textbf{\textit{Git}:} Sistema de control de versiones distribuido
        Comandos importantes:
        \begin{itemize}
          \item \mintinline{bash}{git init}: Inicia un repositorio.
        \end{itemize}
      \item \textbf{Conceptos adicionales:}
        \begin{itemize}
          \item \textbf{Sistema distribuido:\label{kw:distr-sist}}
            Sistema en el que sus componentes están ubicados en varios computadores 
            conectados entre sí.
            Pueden ver este concepto más en profundidad en el curso \textit{CC5212 - 
            Procesamiento Masivo de Datos}
        \end{itemize} 
    \end{itemize}
  %
  \section{Material adicional}
    Los siguientes enlaces contienen explicaciones alternativas y/o más detalladas de
    las herramientas disponibles para utilizar \textit{Git}:
    
    \begin{itemize}
      \item \href{https://guides.github.com/activities/hello-world/}{
        \textbf{Introducción a \textit{GitHub}}}: tutorial básico de cómo crear y 
        manejar un repositorio en \textit{GitHub}. 
      \item \href{https://guides.github.com/introduction/git-handbook/}{
        \textbf{Integración de \textit{Git} y \textit{GitHub}}}: guía de los comandos 
        básicos de \textit{Git} y cómo se relacionan con \textit{GitHub}.
      \item \href{https://guides.github.com/introduction/flow/}{\textbf{
        \textit{GitHub} flow:}} la modalidad (flujo) de trabajo recomendada por 
        \textit{GitHub} para trabajar en equipos.
        Les recomendamos encarecidamente que utilicen esta modalidad al trabajar 
        individualmente y, en especial, en las tareas de este ramo.
      \item \href{https://guides.github.com/features/mastering-markdown/}{
        \textbf{\textit{Markdown}:}} es la sintaxis utilizada para crear documentación
        en \textit{GitHub}.
        Parecida a \textit{HTML} pero más concisa (en realidad es un \textit{superset}
        de \textit{HTML}). 
      \item \href{https://guides.github.com/features/issues/}{
        \textbf{\textit{Issues}:}} es una forma de llevar una lista de \textit{tareas}
        o \textit{problemas} y \textit{metas} en un proyecto.
        Suelen utilizarse mucho cuando un usuario encuentra un \textit{bug} o solicita
        la implementación de un \textit{feature} nuevo en algún proyecto.
      \item \href{https://guides.github.com/activities/forking/}{
        \textbf{\textit{Fork}:}} un \textit{fork} es como clonar un repositorio de 
        otra persona para trabajar paralelamente en el desarrollo de alguna 
        funcionalidad sin cambiar el repo original.
        Esto es común de hacer cuando se quiere ayudar en el desarrollo de proyectos 
        \textit{open-source}.
      \item \href{https://guides.github.com/features/wikis/}{
        \textbf{\textit{GitHub wikis}:}} herramienta avanzada para documentar 
        proyectos.
        La misma que se usa en la wiki del curso. 
      \item \href{https://www.youtube.com/githubguides}{\textbf{Canal de 
        \textit{YouTube} de \textit{GitHub}}}.
      \item \href{https://www.youtube.com/watch?v=uUzRMOCBorg}{\textbf{Usar 
        \textit{IntelliJ} para manejar \textit{Git}}}.
      \item  \href{https://code.visualstudio.com/docs/editor/versioncontrol}{
        \textbf{Usar \textit{VSCode} para manejar \textit{Git}}}.
      \item \href{https://www.gitkraken.com}{\textbf{\textit{GitKraken}:}} interfaz 
        gráfica para manejar repositorios de \textit{Git}.
        Pueden obtener una licencia gratis postulando a los beneficios de 
        \textit{GitHub Education}.
      \item \href{https://www.mercurial-scm.org}{\textbf{\textit{Mercurial}:}} 
        alternativa a \textit{Git} que también es ampliamente usada.
        \textit{SourceForge} es uno de los ejemplos más importantes de \textit{host} 
        de repositorios de \textit{Mercurial} (el equivalente a \textit{GitHub}).
    \end{itemize}
  %
  \nocite{*}
  \printbibliography[keyword=git]
%