% !TEX root = D:\Users\Ignacio\Documentos\Escuela\CC3002 - Metodologías de Diseño y Programación\apunte-y-ejercicios\src\latex\Apunte.tex
\section{Herencia}

  En las secciones anteriores dimos una definición para un punto en 2 dimensiones, pero un 
  punto en un plano es algo bastante limitado.
  En esta sección tomaremos la definición inicial de nuestro punto y la generalizaremos 
  para luego, en base a nuestra nueva definición, crear casos más específicos.

  Consideren un vector como una línea que va desde el origen del sistema de coordenadas 
  hasta un punto.
  Un vector en un espacio euclídeo de dimensión \(n\) es una n-tupla de números reales.
  Dada esta definición podemos definir una clase \texttt{VectorND} de la siguiente forma:
  
  \begin{listing}[ht!]
    \begin{minted}{java}
      class VectorND {
        double[] tail;
        
        VectorND(double[] tail) {
          this.tail = tail;
        }
      }
    \end{minted}
  \end{listing}

  Ahora, una característica de los vectores es que pueden sumarse entre ellos.
  La suma de dos vectores es sólo la suma de sus coordenadas, teniendo en cuenta las 
  dimensiones de estos (un vector de \(m\) dimensiones, con \(m \leq n\) es un vector de 
  \(n\) dimensiones en el que todas las componentes \(\mathbf{v}_i\) para \(i > m\) son 
  0).

  \begin{minted}{java}
    VectorND add(VectorND otherVector) {
      double[] bigger, smaller;
      if (tail.length > otherVector.tail.length) {
        bigger = tail;
        smaller = otherVector.tail;
      } else {
        bigger = otherVector.tail;
        smaller = tail;
      }
      double[] components = new double[bigger.length];
      for (int i = 0; i < smaller.length; i++) {
        components[i] = bigger[i] + smaller[i];
      }
      for (int i = smaller.length; i < bigger.length; i++) {
        components[i] = bigger[i];
      }
      return new VectorND(components);
    }
  \end{minted}
  
  Ahora, es poco común trabajar con vectores de \(n\) dimensiones, en general trabajaremos
  con vectores de dos o tres dimensiones, así que sería una buena idea tener clases 
  específicas para dichos tipos.
  Ahora, ¿Por qué es una buena idea?

  Como mencionamos en el capítulo \ref{ch:oop}, el objetivo de la herencia es la 
  \textbf{especialización} de una clase, y eso es precisamente lo que queremos hacer aquí.
  Tomar un vector de \(n\) dimensiones y crear casos específicos para otros con 
  dimensiones fijas (que tendrán propiedades propias de acuerdo a sus dimensiones).

  Cambiemos el nombre de nuestra clase \texttt{Point2D} por \texttt{Vector2D} y hagamos 
  que sea una subclase de \texttt{VectorND}.
%