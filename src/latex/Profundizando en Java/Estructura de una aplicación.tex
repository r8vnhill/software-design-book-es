% !TEX root = D:\Users\Ignacio\Documentos\Escuela\CC3002 - Metodologías de Diseño y Programación\apunte-y-ejercicios\src\latex\Apunte.tex
\section{Estructura de una aplicación}
  Como se explicó en el capítulo anterior en \textit{Java} (casi) todo son objetos, por 
  esto la aplicaciones se componen de clases que se referencian entre sí.
  Por este motivo, una aplicación en \textit{Java} es finalmente una clase con un método
  \texttt{main} que indica que es ejecutable (este método \texttt{main} puede estar 
  implícito en algunos contextos, un ejemplo de este caso se verá cuando se revise 
  \textit{testing} en el capítulo \ref{ch:tdd}).
  La firma\footnote{Refiérase a la sección \ref{sec:lookup}}\ de este método 
  \textbf{siempre} debe ser \mintinline{java}{public static void main(String[] args)}
  (o lo que es equivalente \mintinline{java}{public static void main(String... args)}), 
  esto quiere decir que si tenemos una clase con el método:

  \begin{minted}{java}
    public void main(String[] args) {
      System.out.println("I am gravely mistaken");
    }
  \end{minted}

  esta clase no será ejecutable.

  \subsection{Estructura de un proyecto}
    Entenderemos por \textbf{proyecto} a cualquier conjunto de archivos y clases que 
    componen una aplicación.
    
    Si es la primera vez que se abre \textit{IntelliJ}, entonces se mostrará el 
    \textit{landing view} del \textit{IDE}, ahí pueden crear un nuevo proyecto 
    directamente haciendo clic en la opción \texttt{Create New Project} (vean la figura 
    \ref{fig:intellij-lv}).

    \begin{figure}[ht!]
      \centering
      \includegraphics[width=0.7\textwidth]{img/Profundizando en Java/IntelliJ Landing.png}
      \caption{\textit{Landing view} de \textit{IntelliJ}}
      \label{fig:intellij-lv}
    \end{figure}

    Si ya abrieron algún proyecto con \textit{IntelliJ}, entonces lo más probable es que
    el \textit{IDE} vuelva a abrir el último proyecto en el que trabajaron.
    En ese caso, para crear un nuevo proyecto deben ir a \texttt{File > New... > Project} 
    (como en la figura \ref{fig:intellij-project-1}).

    \begin{figure}[ht!]
      \centering
      \includegraphics[
        width=0.7\textwidth
      ]{img/Profundizando en Java/IntelliJ Main Menu New Project.png}
      \caption{Crear un nuevo proyecto desde el menú principal de \textit{IntelliJ}}
      \label{fig:intellij-project-1}
    \end{figure}

    \newpage
    Una vez que hayan hecho lo anterior, entrarán al menú de creación de proyectos.
    Aquí se les presentaran muchas opciones (figura \ref{fig:intellij-project-2}), aquí 
    deben seleccionar el \textit{SDK} y las herramientas a utilizar en el proyecto, en el 
    ejemplo el \textit{SDK} es el 14 y no utilizaremos ninguna herramienta adicional.

    \begin{figure}[ht!]
      \centering
      \includegraphics[
        width=0.7\textwidth
      ]{img/Profundizando en Java/IntelliJ Project Type.png}
      \caption{Menú de creación de proyecto: tipo de proyecto}
      \label{fig:intellij-project-2}
    \end{figure}

    Con estas opciones seleccionadas, den clic a \texttt{Next} para continuar al paso 
    siguiente de la creación del proyecto.

    La pestaña siguiente les dejará seleccionar un \textit{template} para comenzar un 
    proyecto con código predefinido.
    En este caso no utilizaremos ninguno, así que pueden saltarse esa vista.

    \begin{figure}[ht!]
      \centering
      \includegraphics[
        width=0.7\textwidth
      ]{img/Profundizando en Java/IntelliJ Project Name.png}
      \caption{Menú de creación de proyecto: datos del proyecto}
      \label{fig:intellij-project-3}
    \end{figure}
  %
%