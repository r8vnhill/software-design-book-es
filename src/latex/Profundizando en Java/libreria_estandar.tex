% !TEX root = C:\Users\smfla\Documents\Escuela\CC3002 - Metodologías de Diseño y Programación\apunte-y-ejercicios\src\latex\Apunte.tex
\section{La librería estándar}
  En el curso de algoritmos y estructuras de datos vieron cómo implementar varias estructuras de 
  datos para resolver problemas comunes, tener una idea general de cómo se implementan y los 
  beneficios y costos de éstas es crucial para el ámbito de programación.
  Dicho esto, la mayoría de los lenguajes de programación tienen implementadas las estructuras más
  usuales dentro de sus librerías estándar, y \textit{Java} no es la excepción.\footnote{\textit{C} 
  es un ejemplo de lenguaje que no tiene implementadas estas estructuras.}

  La librería estándar de \textit{Java} es demasiado amplia como para cubrirla en este apunte por lo
  que sólo veremos los contenidos que nos serán de utilidad, en particular listas y diccionarios.

  %% region : Listas
  \subsection{Listas}
    Hasta ahora hemos utilizado arreglos para almacenar información, los arreglos tienen la ventaja
    de ser eficientes en cuanto a velocidad y memoria pero la poca flexibilidad que presentan resulta
  % endregion
  %% region : Diccionarios
  % endregion