% !TEX root = D:\Users\Ignacio\Documentos\Escuela\CC3002 - Metodologías de Diseño y Programación\apunte-y-ejercicios\src\latex\Apunte.tex
\subsection{Creando instancias de clases}
  Ahora, veamos un poco más en detalle qué es lo que está ocurriendo cuando se crea un objeto.
  Los objetos se crean con la keyword \mintinline{java}{new}, eso ya lo habíamos visto. 
  ¿Pero qué es exactamente lo que hace la instrucción \mintinline{java}{new}?

  Cuando creamos un objeto, lo que estamos haciendo es enviar el mensaje \mintinline{java}{new} 
  \textbf{a la clase} de la que queremos crear la instancia.
  Al recibir el mensaje, la clase invoca a su constructor y retorna una instancia de ella (un 
  objeto).
  Podemos hacer un símil de esto con los objetos (estructuras) de \textit{C}.
  Consideren el siguiente código:

  \begin{minted}{c}
    deftype struct {
      char* type;
    } Bakemon;

    Bakemon createBakemon() {
      Bakemon* bakemon = (Bakemon*) malloc(sizeof(Bakemon));
      bakemon->type = "NONE";
      return *bakemon;
    }
  \end{minted}

  Los constructores, al igual que los métodos, pueden recibir parámetros, entonces, si quisieramos 
  que nuestro \textit{Bakemon} tenga un nombre y una cantidad de \textit{hit points}. 
  Esto lo podemos lograr con:

  \begin{minted}{java}
    public class Bakemon {
      private String name;
      private int hitPoints;
      private String type = "NONE";

      public Bakemon(String name, int hitPoints) {
        this.name = name;
        this.hitPoints = hitPoints;
      }
    }
  \end{minted}

  Ahora, con este nuevo constructor podemos crear instancias de la clase como:
  \begin{minted}{java}
    Bakemon glamorant = new Bakemon("Glamorant", 10);
    Bakemon italicore = new Bamekon("Italicore", 15);
  \end{minted}

  ¿Qué pasa ahora si intentamos hacer \mintinline{java}{new Bakemon()}?
  Nuestro código no compilaría ya que no existe un constructor sin parámetros ya que, como 
  declaramos un constructor explícitamente, \textit{Java} no crea un constructor por defecto como
  habíamos visto en la sección anterior.
  Si quisiéramos lograr el mismo comportamiento anterior tendríamos que declarar un constructor que
  no reciba parámetros.

  En \textit{Java} (a diferencia de \textit{Python} por ejemplo) se pueden declarar múltiples 
  constructores para una clase.
  Luego, podemos añadir dos constructores adicionales a nuestra clase de la forma

  \begin{minted}{java}
    public class Bakemon {
      private String name;
      private int hitPoints;
      private String type = "NONE";

      public Bakemon() {
      }

      public Bakemon(String name, int hitPoints) {
        this.name = name;
        this.hitPoints = hitPoints;
      }

      public Bakemon(String name, int hitPoints, String type) {
        this.name = name;
        this.hitPoints = hitPoints;
        this.type = type;
      }
    }
  \end{minted}

  Noten que tanto en \mintinline{java}{Bakemon(String, int)} y 
  \mintinline{java}{Bakemon(String, int, String)} asignamos las variables de instancia \texttt{name}
  y \texttt{hitPoints}, esto puede llevar a problemas y, como veremos en más detalles en la 
  siguiente parte del apunte, en general querremos evitar la duplicación de código.
  Para evitar esto podemos reutilizar constructores de la misma clase con el mensaje 
  \mintinline{java}{this(...)} que llamará al constructor que tenga los mismos tipos de parámetros
  que el mensaje que se envía.
  Es importante notar que \mintinline{java}{this} y \mintinline{java}{this(...)} son cosas 
  distintas, la primera es una pseudo-variable (véase sección \ref{sec:lookup}) mientras que la 
  segunda es un mensaje.
  Veamos como mejorar los constructores.

  \begin{minted}{java}
    public class Bakemon {
      private String name;
      private int hitPoints;
      private String type = "NONE";

      public Bakemon() {
      }

      public Bakemon(String name, int hitPoints) {
        this.name = name;
        this.hitPoints = hitPoints;
      }

      public Bakemon(String name, int hitPoints, String type) {
        this(name, hitPoints);
        this.type = type;
      }
    }
  \end{minted}

  \begin{exercise}
    Suponga que agregamos a la clase el método
    \begin{minted}{java}
      public void setHitPoints(int hitPoints) {
        this.hitPoints = hitPoints;
      }
    \end{minted}

    ¿Puedo llamar al método \mintinline{java}{setHitPoints(int)} desde el constructor?
    Por ejemplo:
    \begin{minted}{java}
      public Bakemon(String name, int hitPoints) {
        this.name = name;
        setHitPoints(hitPoints);
      }
    \end{minted}
  \end{exercise}

  \begin{exercise}
    ¿Puedo enviar el mensaje \mintinline{java}{this(...)} desde un método?
    ¿Por qué razón?
  \end{exercise}

  \begin{exercise}
    Suponga que modificamos el constructor de la clase por
    \begin{minted}{java}
      public Bakemon(String name, int hitPoints, String type) {
        this.type = type;
        this(name, hitPoints);
      }
    \end{minted}
    ¿Cambia esto el comportamiento de la clase?
    ¿Qué sentido tiene ese comportamiento?
  \end{exercise}

  \begin{exercise}
    ¿Pueden haber constructores privados en una clase?
    ¿Qué utilidad tendría esto?  
  \end{exercise}

  \begin{exercise}
    ¿Puede una clase tener sólo constructores privados?
    ¿En qué contexto podría resultar útil esto?
  \end{exercise}

