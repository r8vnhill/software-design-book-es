%!TEX root = C:\Users\Ignacio\Documents\Escuela\CC3002 - Metodologías de Diseño y Programación\apunte-y-ejercicios\src\latex\Apunte.tex
\subsection{Diccionarios}
  Otra estructura sumamente importante son los diccionarios, estos nos permiten asociar 
  \textit{llaves} con \textit{valores} y así poder acceder a ellos de manera eficiente.
  Al igual que con las listas existen muchas implementaciones distintas de diccionarios pero una
  forma simple de verlos es pensarlos como una tabla con dos columnas, donde la primera contiene las
  llaves y la segunda los valores.

  En \textit{Java} los diccionarios se agrupan dentro de la interfaz \mintinline{java}{Map<K, V>},
  donde \texttt{K} es el tipo de la llave y \texttt{V} el del valor (al igual que como sucede con 
  las listas, las llaves y valores son homogéneos).

  Algunos métodos útiles que define la interfaz \texttt{Map} son:
  \begin{minted}{java}
    // Entrega la cantidad de pares llave-valor del diccionario
    int size();
    // Retorna true si el diccionario está vacío
    boolean isEmpty();
    // Verifica si una llave existe en el diccionario
    boolean containsKey(Object key);
    // Análogo al anterior pero para valores
    boolean containsValue(Object value);
    // Retorna el valor asociado a la llave key
    V get(Object key);
    // Asocia el valor value con la llave key
    V put(K key, V value);
    // Elimina la llave key y su valor asociado y retorna el valor.
    V remove(Object key);
  \end{minted}

  Noten que dada la naturaleza de los diccionarios una llave puede estar asociada a un solo valor.

  A diferencia de las listas, existen bastantes más tipos interesantes de diccionarios, algunos que
  vale la pena mencionar:
  \begin{itemize}
    \item \texttt{HashMap}: Diccionario implementado con una función de \textit{hashing}.
    \item \texttt{ConcurrentHashMap}: Equivalente a \texttt{HashMap} pero para operaciones con 
      múltiples \textit{threads}.
    \item \texttt{LinkedHashMap}: Implementación de un diccionario con una función de 
      \textit{hashing} que adicionalmente guarda una lista para conservar el orden en el que se 
      insertan los elementos.
    \item \texttt{EnumMap}: Implementación especial del \texttt{HashMap} que usa 
      enumeraciones\footnote{Refiérase a la sección \ref{sec:enum}} como llaves y tiene una función 
      de \textit{hashing optimizada}.
      Este tipo de diccionarios es más limitado pero asegura una eficiencia mayor a la de un 
      \texttt{HashMap} tradicional.
    \item \texttt{TreeMap}: Diccionario implementado con un árbol rojo-negro.
      Es menos eficiente que un \texttt{HashMap} pero ordena los elementos por su llave.
  \end{itemize}