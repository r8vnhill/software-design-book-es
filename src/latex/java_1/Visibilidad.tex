% !TEX root = C:\Users\Ignacio\Documents\Escuela\CC3002 - Metodologías de Diseño y Programación\apunte-y-ejercicios\src\latex\Apunte.tex
\section{Visibilidad}
  \label{sec:access-mod}

  Como se mencionó en los capítulos anteriores, una de las propiedades importantes de los objetos es
  la encapsulación, o sea, que los elementos internos de un objeto no puedan ser manipulados desde
  afuera.
  Para lograr esto existe el concepto de \textit{visibilidad} (a veces llamada 
  \textit{privacidad} o \textit{acceso}) que define qué elementos de un objeto van a poder verse 
  desde afuera de éste.

  Noten que hacer que un elemento sea visible desde fuera de un objeto no rompe la encapsulación, ya
  que no se están manejando directamente los elementos internos del objeto, sino que es el mismo 
  objeto el que hace visibles ciertas componentes y funcionalidades suyas, y responde a un 
  \textit{mensaje}\footnote{El concepto de mensaje se verá en más detalle en la sección siguiente} 
  enviado por el otro objeto con algún resultado.

  En \textit{Java} existen cuatro modificadores de visibilidad: \mintinline{java}{public}, 
  \mintinline{java}{protected}, \mintinline{java}{private} y \textit{package-private}.
  Los modificadores de acceso pueden utilizarse para definir clases, métodos y variables, a 
  excepción de las variables locales de un método que siempre serán internas al método e 
  inaccesibles desde fuera de este.

  El modificador \mintinline{java}{public} indica que el elemento definido va a ser visible desde 
  cualquier otra clase.
  \mintinline{java}{protected} se usa para definir un elemento que solamente sea accesible desde las
  clases que heredan de la clase en la que se definió el elemento.
  La keyword \mintinline{java}{private} por otro lado define que el elemento será visible solamente 
  desde dentro de la clase en la que se definió.

  El modificador \textit{package-private}, que es con el que se definen todos los elementos si no se
  define explícitamente un modificador de acceso, define al elemento como solo visible dentro del 
  paquete en el que se encuentra la clase.

  TODO:
  \begin{enumerate}
    \item package-private malo
    \item scope de los modificadores
    \item getters/setters
  \end{enumerate}
%