\chapter{Git}
      \section{Material adicional}
        Los siguientes enlaces contienen explicaciones alternativas y/o más detalladas de
        las herramientas disponibles para utilizar \textit{Git}:
        
        \begin{itemize}
          \item \href{https://git-scm.com}{\textbf{Documentación oficial de 
            \textit{Git}}}.
          \item \href{https://rogerdudler.github.io/git-guide/}{\textbf{Comandos básicos 
            de \textit{Git}}}.
          \item \href{https://guides.github.com/activities/hello-world/}{
            \textbf{Introducción a \textit{GitHub}}}: tutorial básico de cómo crear y 
            manejar un repositorio en \textit{GitHub}. 
          \item \href{https://guides.github.com/introduction/git-handbook/}{
            \textbf{Integración de \textit{Git} y \textit{GitHub}}}: guía de los comandos 
            básicos de \textit{Git} y cómo se relacionan con \textit{GitHub}.
          \item \href{https://guides.github.com/introduction/flow/}{\textbf{
            \textit{GitHub} flow:}} la modalidad (flujo) de trabajo recomendada por 
            \textit{GitHub} para trabajar en equipos.
            Les recomendamos encarecidamente que utilicen esta modalidad al trabajar 
            individualmente y, en especial, en las tareas de este ramo.
          \item \href{https://guides.github.com/features/mastering-markdown/}{
            \textbf{\textit{Markdown}:}} es la sintaxis utilizada para crear documentación
            en \textit{GitHub}.
            Parecida a \textit{HTML} pero más concisa (en realidad es un \textit{superset}
            de \textit{HTML}). 
          \item \href{https://guides.github.com/features/issues/}{
            \textbf{\textit{Issues}:}} es una forma de llevar una lista de \textit{tareas}
            o \textit{problemas} y \textit{metas} en un proyecto.
            Suelen utilizarse mucho cuando un usuario encuentra un \textit{bug} o solicita
            la implementación de un \textit{feature} nuevo en algún proyecto.
          \item \href{https://guides.github.com/activities/forking/}{
            \textbf{\textit{Fork}:}} un \textit{fork} es como clonar un repositorio de 
            otra persona para trabajar paralelamente en el desarrollo de alguna 
            funcionalidad sin cambiar el repo original.
            Esto es común de hacer cuando se quiere ayudar en el desarrollo de proyectos 
            \textit{open-source}.
          \item \href{https://guides.github.com/features/wikis/}{
            \textbf{\textit{GitHub wikis}:}} herramienta avanzada para documentar 
            proyectos.
            La misma que se usa en la wiki del curso. 
          \item \href{https://www.youtube.com/githubguides}{\textbf{Canal de 
            \textit{YouTube} de \textit{GitHub}}}.
          \item \href{https://www.youtube.com/watch?v=uUzRMOCBorg}{\textbf{Usar 
            \textit{IntelliJ} para manejar \textit{Git}}}.
          \item  \href{https://code.visualstudio.com/docs/editor/versioncontrol}{
            \textbf{Usar \textit{VSCode} para menejar \textit{Git}}}.
          \item \href{https://www.gitkraken.com}{\textbf{\textit{GitKraken}:}} interfaz 
            gráfica para manejar repositorios de \textit{Git}.
            Pueden obtener una licencia gratis postulando a los beneficios de 
            \textit{GitHub Education}.
          \item \href{https://www.mercurial-scm.org}{\textbf{\textit{Mercurial}:}} 
            alternativa a \textit{Git} que también es ampliamente usada.
            \textit{SourceForge} es uno de los ejemplos más importantes de \textit{host} 
            de repositorios de \textit{Mercurial} (el equivalente a \textit{GitHub}).
        \end{itemize}