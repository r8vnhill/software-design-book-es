% !TEX root = D:\Users\Ignacio\Documentos\Escuela\CC3002 - Metodologías de Diseño y Programación\apunte-y-ejercicios\src\latex\Apunte.tex
\chapter{OOP: de \textit{Python} a \textit{Java}}
  En este capítulo veremos cómo implementar los conceptos vistos en el capítulo anterior 
  en \textit{Java} partiendo desde ejemplos en \textit{Python} para facilitar la 
  transición de uno al otro.

  \subimport{OOP, de Python a Java/}{Introducción.tex}
  \subimport{OOP, de Python a Java/}{Control de flujo.tex}
  \subimport{OOP, de Python a Java/}{Input.tex}
  \subimport{OOP, de Python a Java/}{Tipos.tex}

  \section{Constructores}
    Supongamos ahora que queremos representar un punto con 3 dimensiones.
    Esto podemos representarlo con una clase \texttt{Point3D}, de la siguiente forma:

    \begin{minted}{python}
      class Point3D:
        
    \end{minted}
  %
%