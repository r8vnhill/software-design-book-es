% !TEX root = C:\Users\smfla\Documents\Escuela\CC3002 - Metodologías de Diseño y Programación\apunte-y-ejercicios\src\latex\Apunte.tex
\chapter{OOP de \textit{Python} a \textit{Java}}
  En este capítulo veremos como implementar los conceptos vistos en el capítulo anterior 
  en \textit{Java} partiendo desde ejemplos en \textit{Python} para facilitar la 
  transición de uno al otro.

  \section{Introducción}
    Consideremos un ejemplo básico para comenzar: queremos imprimir un mensaje en consola.

    Podríamos hacer el clásico \textit{Hello world!}, pero que fome, en cambio imprimamos
    otro mensaje.

    Creen un archivo \texttt{ejemplo\_basico.py} y ábranlo con 
    Para imprimir un texto en consola en \textit{Python} haríamos:

    \begin{minted}{python}
      print("My name is Giorno Giovanna, and I have a dream.")      
    \end{minted}

    O lo que sería más correcto de acuerdo a las convenciones de \textit{Python}:

    \begin{minted}{python}
      if __name__ == "__main__":
          print("My name is Giorno Giovanna, and I have a dream.")
    \end{minted}

    Luego, si queremos ejecutar el script haríamos:

    \begin{minted}{bash}
      python3 ejemplo_basico.py
    \end{minted}

    o en caso de utilizar \textit{Windows}:

    \begin{minted}{powershell}
      py -3 ejemplo_basico.py
    \end{minted}

    En \textit{Java} reproducir este mismo ejemplo es un tanto más complicado ya que 
    necesitaremos escribir más líneas de código.
    Para crear un programa equivalente en \textit{Java}, primero crearemos un archivo 
    \texttt{EjemploBasico.java}, luego en el editor de texto que prefieran escriban el 
    código:

    \begin{minted}{java}
      public class EjemploBasico {

        public static void main(String[] args) {
          System.out.println("My name is Giorno Giovanna, and I have a dream.");
        }
      }
    \end{minted}

    Veremos en detalle las diferencias entre la sintaxis de ambos ejemplos, pero primero
    veamos como ejecutar el programa para ver que efectivamente hace lo mismo que el de 
    \textit{Python}, para esto eben ejecutar en consola:

    \begin{minted}{bash}
      javac EjemploBasico.java
      java EjemploBasico
    \end{minted}

    El primer comando creará un archivo \texttt{EjemploBasico.class}, este es un archivo
    pre-compilado (es importante que en las tareas \textbf{NO ENTREGUEN} los archivos 
    \texttt{.class}, ya que no los podemos revisar), luego el segundo comando compila y 
    ejecuta el programa.
    Esto se explicará en el capítulo \ref{ch:java}.

    Ahora, veamos las diferencias entre ambos programas.
    El código en \textit{Python} es bastante fácil de seguir.
    ¿Pero por qué en \textit{Java} hay que definir tantas cosas solamente para imprimir un
    mensaje en consola?

    Vamos por partes, lo primero que deben notar es que la línea con el llamado a 
    \texttt{println(...)} termina con un \texttt{;}, esto debe ser así para todas las 
    instrucciones, esto puede no parecer tan importante a primera vista, pero marca una
    diferencia enorme respecto a \textit{Python}, ya que a diferencia de \textit{Python} 
    la indentación no es importante.
    Cuando programamos en \textit{Python} la indentación es lo que define donde compienza
    y termina una definición, en \textit{Java} en cambio, esto se define entre llaves, 
    donde la apertura de una marca el inicio y el cierre el fin.

    Luego, tenemos la definición \mintinline{java}{public static void main(String[] args)}
    este es un método especial que será el punto de entrada del programa, por lo que al 
    ejecutar el código se ejecutarán todas las instrucciones definidas dentro del método.
    
    Por último, tenemos que todo esto va dentro de la definición de una clase 
    \texttt{EjemploBasico}, esto es necesario ya que \textit{Java} es un lenguaje 
    (casi) totalmente orientado a objetos \textit{fuertemente tipado}.
    De momento basta que sepan que los programas siguen esa sintaxis, en el capítulo 
    \ref{ch:java} veremos más en detalle el funcionamiento de \textit{Java} y 
    profundizaremos en este tema.
  %

  \section{\textit{Input}}
    En la sección anterior mostramos un programa que era capaz de imprimir un mensaje en
    pantalla, pero siempre que lo ejecutemos hará lo mismo.
    ¿Qué hacemos para que el programa reciba \textit{input} de un usuario?
    Para esto tendrémos dos opciones:

    \subsection{Opción 1: Argumentos por consola}
      La primera opción es la que usan la mayoría de aplicaciones de consola que vienen 
      integradas en los sistemas operativos, i.e. recibir los parámetros como argumentos
      entregados al momento de ejecutar el programa, por ejemplo:

      \begin{minted}{bash}
        cd path/to/dir
      \end{minted}

      Modifiquemos un poco el ejemplo anterior para que reciba parámetros desde consola,
      en el caso de \textit{Python}, para recibir los argumentos se debe hacer una llamada
      al sistema, el siguiente ejemplo muestra como hacer esto:

      \begin{minted}{python}
        import sys

        if __name__ == "__main__":
            # Tomamos todos los argumentos menos el primero porque en Python el primer 
            # argumento siempre es el nombre del archivo.
            name = " ".join(sys.argv[1:])
            print(f"My name is {name}, and I have a dream.")
      \end{minted}

      Y luego podemos ejecutarlo como:

      \begin{minted}{bash}
        python3 ejemplo_basico.py Ignacio Slater
      \end{minted}

      Por otro lado, en \textit{Java} tendríamos:

      \begin{minted}{java}
        public class EjemploBasico {

          public static void main(String[] args) {
            System.out.println("My name is " + String.join(" ", args) + ", and I have a dream.");
          }
        }
      \end{minted}

      Como pueden ver, en este caso no hay necesidad de hacer una llamada explícita al 
      sistema para obtener los argumentos, ya que el método \texttt{main} lo hace por 
      defecto.
      De manera similar a lo que hicimos para ejecutar el programa en \textit{Python}, 
      ahora ejecutaremos éste como:

      \begin{minted}{bash}
        javac EjemploBasico.java
        java EjemploBasico Ignacio Slater
      \end{minted}      

      En ambos casos el resultado debiera ser el mismo.
    %

    \subsection{Opción 2: Pedir parámetros interactivamente}
      La otra opción es hacer que el usuario ingrese parámetros durante la ejecución del 
      programa.

      En \textit{Python}, si quisieramos hacer eso, tendríamos que modificar el programa 
      anterior como:

      \begin{minted}{python}
        if __name__ == "__main__":
            name = input("Write your name: ")
            print(f"My name is {name}, and I have a dream.")
      \end{minted}

      Nuevamente, en \textit{Java} el código sería más extenso:

      \begin{minted}{java}
        import java.util.Scanner;

        public class EjemploBasico {

          public static void main(String[] args) {
            System.out.println("Write your name: ");
            String name = new Scanner(System.in).nextLine();
            System.out.println("My name is " + name + ", and I have a dream.");
          }
        }
      \end{minted}

      No entraremos en más detalles dentro de estos conceptos ya que no serán utilizados 
      durante el curso.
    %
  %

  \section{Objetos y clases}
    

  \section{Constructores}
    Supongamos ahora que queremos representar un punto con 3 dimensiones.
    Esto podemos representarlo con una clase \texttt{Point3D}, de la siguiente forma:

    \begin{minted}{python}
      class Point3D:
        
    \end{minted}
  %
%