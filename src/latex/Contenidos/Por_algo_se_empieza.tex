% !TEX root = C:\Users\Ignacio\Documents\Escuela\CC3002 - Metodologías de Diseño y Programación\apunte-y-ejercicios\src\latex\METODOLOG_AS_DE_DISE_O_Y_PROGRAMACI_N_ORIENTADA_A_OBJETOS.tex
\epigraphhead[471]{
  Este libro no partía así, y si alguien leyó una versión anterior se dará cuenta.
  Solía comenzar con una descripción e instrucciones para instalar las herramientas necesarias para
  seguir este libro y eso no estaba mal, pero no me agradaba comenzar así, simplemente presentando
  las herramientas sin ningún contexto de por qué ni para qué las ibamos a utilizar.

  Puede parecer ridículo cambiar todo lo que ya había escrito solamente por eso, pero cuando nos 
  enfrentamos a problemas del mundo real esto comienza a cobrar más sentido.
  Reescribo estos capítulos por una razón simple pero sumamente importante y que será una de las
  principales motivaciones para las decisiones de diseño que tomaremos a medida que avancemos, en
  el desarrollo de software \textbf{lo único constante es el 
  \textit{cambio}}.\autocite{head-first-intro}
  Una aplicación que no puede adaptarse a los cambios, sin importar que tan bien funcione, está
  destinada a morir.

  ¿Qué sucede entonces con las herramientas que vamos a utilizar?
  Las vamos a introducir, no podemos sacar una parte tan importante, pero no las vamos a presentar
  todas a la vez, en su lugar las iremos explicando a medida que las vayamos necesitando.
}
\part{Por algo se empieza}

  \chapter{¿Qué es un Java?}
    \label{ch:java}
    \textit{Java} es uno de los lenguajes de programación más utilizados en el mundo (de ahí la 
    necesidad de enseñarles éste y no otro lenguaje), se caracteriza por ser un lenguaje basado en 
    clases, orientado a objetos, estática y fuertemente tipado, y (casi totalmente) independiente del sistema 
    operativo.
    
    \begin{center}
      \textit{¿Qué?}
    \end{center}

    Tranquilos, vamos a ir de a poco.
    Comencemos por uno de los puntos que hizo que \textit{Java} fuera adoptado tan ampliamente en la
    industria, la independencia del sistema operativo.
    Cuando \textit{Sun Microsystems}\footnote{Actualmente \textit{Java} es propiedad de 
    \textit{Oracle Corporation}} publicó la primera versión de \textit{Java} (en 1996), los 
    lenguajes de programación predominantes eran \textit{C} y \textit{C++} (y en menor medida 
    \textit{Visual Basic} y \textit{Perl}).
    Estos lenguajes tenían en común que interactuaban directamente con la API del sistema operativo,
    lo que implicaba que un programa escrito para un sistema \textit{Windows} no funcionaría de la
    misma manera en un sistema \textit{UNIX}.
    \textit{Java} por su parte planteó una alternativa distinta, delegando la tarea de compilar y 
    ejecutar los programas a una máquina virtual (más adelante veremos en más detalle parte del 
    funcionamiento de la \textit{JVM} para entender sus beneficios y desventajas).
    Esto último hizo que, en vez de cambiar el código del programa para crear una aplicación para 
    uno u otro sistema operativo, lo que cambiaba era la versión de la \textit{JVM} permitiendo así
    que un mismo código funcionara de la misma forma en cualquier plataforma capaz de correr la 
    máquina virtual.\footnote{Actualmente casi todos los sistemas operativos son capaces de usar la 
    \textit{JVM}, en particular el sistema \textit{Android} está implementado casi en su totalidad 
    para usar esta máquina virtual.}
    Pasarían varios años antes de que surgieran otros lenguajes que compartieran esa característica
    (destacando entre ellos \textit{Python 2}, publicado el año 2000).

    No tiene mucho sentido seguir hablando de \textit{Java} si no podemos poner en práctica lo que 
    vayamos aprendiendo, así que es un buen momento para instalarlo.
    
    \subimport{Por_algo_se_empieza/}{Instalando_el_JDK.tex}
    \subimport{Por_algo_se_empieza/}{Tipos_en_Java.tex}
    \subimport{Por_algo_se_empieza/}{Primeros_pasos.tex}
  
  \printbibliography[keyword=Por_algo_se_empieza]
  
  \chapter{¡Kotlin!}
    Esta sección busca presentar \textit{Kotlin} como una alternativa a \textit{Java}, no es 
    necesario, pero se recomienda leer el capítulo \ref{ch:java} antes de continuar este ya que 
    muchos de los principios que se explican ahí también se cumplen para \textit{Kotlin}.
    
    \textit{Kotlin} es un lenguaje de programación \textit{multiplataforma}, estática y fuertemente 
    tipado y con una inferencia de tipos más avanzada que la de \textit{Java}.
    % TODO  Liskov HFDP p.5 override vacío
    % TODO  Ppio. de diseño HFDP p.9 Identify the aspects of your application that vary and separate 
    %       them from what stays the same.
  \nocite{*}
