\subsection{Primeros pasos}
  Si programar fuera como salir a trotar (afortunadamente no lo es), podríamos pensar que todo hasta
  este punto fue la preparación previa, buscar las zapatillas, la ropa, poner su playlist más 
  motivante.
  Pero todavía no podemos comenzar a hacer ejercicio, primero debemos calentar; eso es esta parte, 
  un calentamiento para soltar las manos con \textit{Java}.

  Para los siguientes ejemplos puede serles útil tener dos terminales abiertas: una con 
  \textit{jshell} y otra con la consola interactiva de \textit{Python}.\footnote{Los comandos para 
  abrirlos son \texttt{jshell} y \texttt{python} respectivamente.}

  \subsubsection{Funciones}
    Lo más básico que nos gustaría poder hacer en un programa es crear una función.
    Partamos por algo simple, imprimir un mensaje en pantalla, en \textit{Python} esto se hace con la 
    función \mintinline{python}{print(str)} mientras que en \textit{Java} tenemos que usar 
    \mintinline{java}{System.out.println(String)}, vemos inmediatamente que el segundo es bastante más
    complejo, por ahora ignoren la parte \mintinline{java}{System.out} y asuman que esa instrucción 
    escribe en la salida estándar.\footnote{Si les ayuda, lo que hace la instrucción es escribir un 
    mensaje en el canal de salida (\textit{out}) del sistema, que no necesariamente va a ser la salida
    estándar siempre.}

    \begin{minted}{python}
      # Python
      def jojo_reference(nombre):
          print("Mi nombre es " + nombre + " y tengo tuto")
    \end{minted}

    \begin{minted}{java}
      // Java
      void jojoReference(String nombre) {
        System.out.println("Mi nombre es " + nombre + " y tengo tuto");
      }
    \end{minted}

    Revisemos los ejemplos en detalle.
    La primera línea en ambos ejemplos es un comentario, así como en \textit{Python} los comentarios
    comienzan con \mintinline{python}{#}, en \textit{Java} comienzan con \mintinline{java}{//}.

    La siguiente línea es la \textit{firma}\footnote{Esto lo veremos en detalle en la sección 
    \ref{sec:lookup}.} de la función.
    En \textit{Python} las funciones se definen como:
    \begin{minted}{python}
      def func_name(parameters):
    \end{minted}
    sin necesidad de especificar los tipos por lo que explicamos en la sección anterior.
    En \textit{Java} la sintaxis es bastante más explícita, teniendo que declarar el tipo de retorno de
    la función y el tipo de sus parámetros de la  forma:
    \begin{minted}{java}
      returnType funcName(Param1Type param1, Param2Type param2, ...) {...}
    \end{minted}
    donde: (1) \texttt{returnType} es el tipo del valor que retorna la función, (2) \texttt{funcName} es
    el nombre de la función y, (3) \texttt{paramXType} y \texttt{paramX} son el tipo y nombre de cada 
    parámetro.
    Otro detalle que podemos ver es que en \textit{Python} la función se llama \texttt{func\_name} y en 
    \textit{Java} \texttt{funcName}, esto es por que las convenciones de ambos lenguajes son 
    distintas; pueden revisar las convenciones de \textit{Java} en la sección \ref{sec:conventions}.
    
    Por último está el cuerpo de la función. 
    En \textit{Python} el comienzo del cuerpo de una función se marca con \texttt{:} y el final de 
    ésta está dada por la sangría (o \textit{indentación}).
    En \textit{Java} los espacios en blanco no tienen importancia para que el programa 
    funcione,\footnote{¡Pero sí para hacer su código legible!} en cambio, el inicio y fin de una 
    función lo marcan la apertura y cierre de las llaves \enquote{\texttt{\{\}}}.
    En general podremos pensar que en todos los casos en los que \textit{Python} ocupa \texttt{:} el
    equivalente en \textit{Java} serán las llaves (e.g. \texttt{if}, \texttt{while}).

  \subsubsection{Recursión y control de flujo}
    Veamos algo un poquito más complicado, supongamos que queremos computar el \(n\mathrm{-ésimo}\)
    número de la \textit{sucesión de Fibonacci}, esta secuencia está definida por:
    \[
      \begin{align}
        f_0 &= 0  \\
        f_1 &= 1  \\
        f_n = f_{n - 1} + f_{n - 2}
      \end{align}
    \]

    Esta definición es naturalmente recursiva, y podemos implementarla fácilmente de la siguiente 
    forma:

    \begin{minted}{python}
      # Python
      def fibonacci(n):
        if n < 2:
          return n
        return fibonacci(n - 1) + fibonacci(n - 2)
    \end{minted}

    \begin{minted}{python}
      # Java
      int fibonacci(int n) {
        if (n < 2) {
          return n;
        }
        return fibonacci(n - 1) + fibonacci(n - 2);
      }
    \end{minted}

    Con lo que hemos visto hasta ahora debiera ser fácil 
  \begin{center}
    \texttt{Hola, soy un placeholder para que sepas que esta sección todavía no está completa :3}
  \end{center}