\subsection{Instalando el \textit{JDK}}
  El \textit{Java Development Kit} (\textit{JDK}) es un conjunto de herramientas que incluyen 
  todo lo necesario para desarrollar aplicaciones en \textit{Java} (la \textit{JVM}, el 
  compilador, la librería estándar, etc.), esto es lo que generalmente instalaremos si queremos
  programar en \textit{Java}.\footnote{Existen otras herramientas que incluyen los contenidos de 
  el \textit{JDK} de forma total o parcial, por ejemplo: \textit{Java SE}, \textit{JRE}, o 
  incluso otros lenguajes de programación que usan la \textit{JVM}.}

  \subsubsection{Windows}
    \paragraph{Chocolatey}
      Lo primero que necesitaremos para instalar las herramientas que usaremos será un gestor de 
      paquetes, utilizaremos \textit{Chocolatey}.\autocite{choco}

      Para partir abran una ventana de \textit{Powershell} como administrador.
      Una vez abierta, deben ejecutar las instrucciones:\footnote{
        \url{https://github.com/CC3002-Metodologias/apunte-y-ejercicios/blob/master/install/windows/chocolatey.ps1}
      }
      \begin{minted}{powershell}
        [Net.ServicePointManager]::SecurityProtocol = `
          [Net.SecurityProtocolType]::Tls12
        Set-ExecutionPolicy -ExecutionPolicy RemoteSigned -Force
        Invoke-WebRequest "https://chocolatey.org/install.ps1" `
          -UseBasicParsing | Invoke-Expression
      \end{minted}

      Esto otorgará los permisos necesarios y descargará e instalará el gestor de paquetes.

      Para comprobar que el programa se haya instalado correctamente corran el comando:
      \begin{minted}{powershell}
        choco -?
      \end{minted}
      
    \paragraph{(Opcional) Cmder}
      Es sabido que las terminales por defecto de \textit{Windows} dejan bastante que desear, 
      por es una buena idea instalar una terminal externa (o más bien un emulador de una).
      Existen varias opciones, pero \textit{Cmder} es una de las más completas.