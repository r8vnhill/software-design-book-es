Hasta ahora es probable que tengan experiencia utilizando algún editor de texto como 
\textit{Notepad++}, \textit{Sublime} o \textit{Visual Studio Code}, eso es bueno, pero no 
suficiente para enfrentar problemas complejos (esto no es por desmerecer a esos editores de 
texto, también son herramientas poderosas apropiadas para otros contextos).
Es aquí cuando surge la necesidad de tener un \textit{entorno de desarrollo 
integrado}\index{Entorno de desarrollo integrado} (IDE), un IDE es como un editor de texto 
cualquiera, pero poderoso.
\textit{IntelliJ}\index{IntelliJ IDEA} es un IDE desarrollado por \textit{JetBrains} especializado 
para desarrollar programas en \textit{Java}, \textit{Kotlin} y \textit{Scala}, y es la herramienta 
que usaremos de aquí en adelante.
\begin{center}
  \textit{
    Pero VSCode nunca me ha fallado ¿Por qué no puedo usarlo también para esto si le puedo 
    instalar extensiones para programar en Kotlin?
  }
\end{center}

La razón es simple, pero muy importante, como dije un poco más arriba, \textit{IntelliJ} es una
herramienta diseñada específicamente para desarrollar en \textit{Kotlin}, esto hace que sea 
muchísimo más completo que otros editores de texto y que tenga facilidades para correr, configurar y 
testear\footnote{Esto será sumamente importante en capítulos futuros} nuestros proyectos.
De nuevo, no me malinterpreten, \textit{VSCode} es una herramienta sumamente completa y 
perfectamente capaz de utilizarse en el mundo profesional, pero es bueno que conozcan otras 
alternativas también.
Es posible utilizar cualquier otra herramienta que permita hacer lo mismo que \textit{IntelliJ}
para seguir este libro, pero será responsabilidad del lector/a/e adaptar lo visto aquí para 
funcionar con dichas herramientas.\footnotemark

\footnotetext{
  Dicho esto, este libro es abierto a la colaboración y cualquier aporte que contribuya al 
  aprendizaje de les lectores es bienvenido.
  En caso de querer colaborar pueden hacerlo mediante un \textit{Pull Request} al repositorio de
  \textit{GitHub} de este libro (\url{https://github.com/islaterm/software-design-book-es}), 
  respetando las normas establecidas en el 
  \href{
    https://github.com/islaterm/software-design-book-es/blob/master/CODE_OF_CONDUCT.md
  }{
    \textit{Código de conducta}
  }.
}