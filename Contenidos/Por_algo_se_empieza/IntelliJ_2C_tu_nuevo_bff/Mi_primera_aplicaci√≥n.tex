\section{Mi primera aplicación}
  Ahora veremos cómo crear una aplicación con \textit{IntelliJ}, será algo muy simple, pero nos
  servirá como base para comenzar el segundo arco de este libro.

  Lo primero que necesitamos es entender la estructura del proyecto.
  En la carpeta del proyecto debieran haber 3 elementos: el archivo \texttt{IntelliJBasics.iml}, y 
  las carpetas \texttt{.idea} y \texttt{src}.
  Su utilidad es:
  \begin{itemize}
    \item \texttt{.idea}: contiene la configuración de IntelliJ.
    \item \texttt{src}: contiene el código fuente de nuestra aplicación.
    \item \texttt{IntelliJBasics.iml}: es el archivo de configuración del proyecto.
  \end{itemize}

  En la mayoría de los casos, no nos van a interesar mucho los archivos de configuración, así que
  interactuaremos solamente con el directorio \texttt{src}.

  \subsection{Paquetes}
    De la misma forma en que un computador se organiza en carpetas, un proyecto se organiza en 
    paquetes.

    El principal objetivo de los paquetes es darle organización a nuestro código, porque, al igual 
    que tener una carpeta repleta de archivos (te estoy mirando a ti que tienes el escritorio 
    tapizado en íconos), tener todo el código de nuestra aplicación en una carpeta (o en el mismo 
    archivo \texttt{D:}) es barbarie.

    La gracia de usar paquetes en vez de simplemente carpetas es que podemos incorporar la lógica de
    los paquetes en nuestro código.
    Además, si se utilizan correctamente podemos evitar gran parte de los problemas que podrían
    surgir al momento de interactuar con librerías externas.
    
    Tanto \textit{Java} como \textit{Kotlin} permiten organizar nuestro código en paquetes, mientras
    que otros lenguajes como \textit{Python} y \textit{C++} no.\footnote{En lugar de paquetes 
    \textit{Python} provee módulos y \textit{C++} tiene \textit{namespaces}, ambas cumplen los 
    mismos objetivos que los paquetes, pero se utilizan de forma distinta. 
    Es importante informarse de esas diferencias cuando están aprendiendo un nuevo lenguaje.}
    El estándar para nombrar paquetes es el mismo para \textit{Java} y \textit{Kotlin}, en general
    es recomendado utilizar el nombre de algún sitio web para asegurarse de que el nombre del 
    paquete es único (para evitar problemas de nombres duplicados al utilizar librerías externas).
    Veamos algunos ejemplos:
    
    \begin{java}
      package cl.ravenhill.intellijbasics; // Bien
      package cl.ravenhill.intelliJBasics; // Mal, el nombre no debe incluir mayúsculas
      package cl.ravenhill.intellij\_basics; // Mal, el nombre no debe incluir guiones bajos
      package cl.ravenhill.intellij-basics; // Permitido, pero no recomendado
    \end{java}
    
    
  \subsection{Mi primer \textit{Java}}
  \subsection{Mi primer \textit{Kotlin}}
