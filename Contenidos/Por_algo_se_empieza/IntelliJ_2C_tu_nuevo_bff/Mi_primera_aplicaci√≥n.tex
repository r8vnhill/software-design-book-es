\section{Mi primera aplicación}
  Ahora veremos cómo crear una aplicación con \textit{IntelliJ}, será algo muy simple, pero nos
  servirá como base para comenzar el segundo arco de este libro.

  Lo primero que necesitamos es entender la estructura del proyecto.
  En la carpeta del proyecto debieran haber 3 elementos: el archivo \texttt{IntelliJBasics.iml}, y 
  las carpetas \texttt{.idea} y \texttt{src}.
  Su utilidad es:
  \begin{itemize}
    \item \texttt{.idea}: contiene la configuración de IntelliJ.
    \item \texttt{src}: contiene el código fuente de nuestra aplicación.
    \item \texttt{IntelliJBasics.iml}: es el archivo de configuración del proyecto.
  \end{itemize}

  En la mayoría de los casos, no nos van a interesar mucho los archivos de configuración, así que
  interactuaremos solamente con el directorio \texttt{src}.

  \subsection{Paquetes}
  \subsection{Mi primer \textit{Java}}
  \subsection{Mi primer \textit{Kotlin}}
