Este capítulo busca presentar \textit{Kotlin} como una alternativa a \textit{Java}, no es
necesario, pero se recomienda leer el capítulo \ref{ch:java} antes de continuar este ya que
muchos de los principios que se explican ahí también se cumplen para \textit{Kotlin}.

\textit{Kotlin} es un lenguaje de programación \textit{multiplataforma}, estática y fuertemente
tipado y con una inferencia de tipos más avanzada que la de \textit{Java}.

\begin{center}
  \textit{¿Pero para qué aprender Kotlin si ya voy a aprender Java?}
\end{center}

Buena pregunta.
\vfill
\newpage

\textit{Kotlin} es un lenguaje desarrollado por \textit{JetBrains} pensado para ser totalmente 
interoperable con \textit{Java}, esto quiere decir que puedo importar código escrito en 
\textit{Java} desde \textit{Kotlin} y vice versa.
En fin, existen muchísimas razones para aprender \textit{Kotlin}, pero en vez de enumerarlas creo
que es mejor que las vayan descubriendo en el transcurso de este libro.