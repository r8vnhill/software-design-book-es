\subsection{\textit{Windows}}
  Nuevamente, para instalar el compilador confiaremos en nuestro amigo 
  \textit{Chocolatey}\index{Chocolatey}, si no saben de lo que hablo, probablemente no leyeron la
  sección \ref{sec:jdk-windows}, no voy a repetir esas instrucciones, así que vayan a leer esa
  sección y luego vuelvan aquí; lxs espero.

  Una vez que tengan \textit{Chocolatey} podemos instalar el compilador desde \textit{Powershell}
  con el comando:

  \begin{powershell}
    choco install kotlinc
  \end{powershell}

  Luego, podemos comprobar la instalación con:
  \begin{powershell}
    kotlinc -version
  \end{powershell}