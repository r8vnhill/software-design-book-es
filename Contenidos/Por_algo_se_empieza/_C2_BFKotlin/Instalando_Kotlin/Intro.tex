De la misma forma que hicimos con \textit{Java}, lo primero que necesitaremos para trabajar en
\textit{Kotlin}\index{Kotlin} es el compilador.
\begin{center}
  \textit{Sorprendente}
\end{center}

La forma más fácil de instalar el compilador de \textit{Kotlin} es instalando 
\textit{IntelliJ}\index{IntelliJ IDEA} (como veremos en el capítulo \ref{ch:intellij}), pero la 
solución más fácil no siempre es la correcta.
Instalar \textit{IntelliJ} para aprender lo más básico de \textit{Kotlin} es como andar intentar
andar en motocicleta cuando todavía estamos aprendiendo a andar en bicicleta con rueditas (pero 
tranquilxs, hacia el final de este capítulo ya habremos encontrado nuestro equilibrio espiritual).
