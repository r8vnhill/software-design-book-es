\epigraphhead[471]{
  Este libro no partía así, y si alguien leyó una versión anterior se dará cuenta.
  Solía comenzar con una descripción e instrucciones para instalar las herramientas necesarias para
  seguir este libro y eso no estaba mal, pero no me agradaba comenzar así, simplemente presentando
  las herramientas sin ningún contexto de por qué ni para qué las ibamos a utilizar.

  Puede parecer ridículo cambiar todo lo que ya había escrito solamente por eso, pero cuando nos 
  enfrentamos a problemas del mundo real esto comienza a cobrar más sentido.
  Reescribo estos capítulos por una razón simple pero sumamente importante y que será una de las
  principales motivaciones para las decisiones de diseño que tomaremos a medida que avancemos, en
  el desarrollo de software \textbf{lo único constante es el 
  \textit{cambio}}.\autocite{head-first-intro}
  Una aplicación que no puede adaptarse a los cambios, sin importar que tan bien funcione, está
  destinada a morir.

  ¿Qué sucede entonces con las herramientas que vamos a utilizar?
  Las vamos a introducir, no podemos sacar una parte tan importante, pero no las vamos a presentar
  todas a la vez, en su lugar las iremos explicando a medida que las vayamos necesitando.
}