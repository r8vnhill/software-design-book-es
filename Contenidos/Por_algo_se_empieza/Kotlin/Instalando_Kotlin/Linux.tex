\subsection{\textit{Linux}}
  Para instalar \textit{Kotlin} en sistemas \textit{Linux} se
  recomienda utilizar \textit{SDKMAN!}\index{SDKMAN!}\footnote{
    El \enquote{!} es parte del nombre
  }.
  Instalar \textit{SDKMAN!} es relativamente simple, solo deben correr
  los comandos que presentamos a continuación y seguir los pasos
  que aparezcan en pantalla.

  \begin{bash}
    curl -s "https://get.sdkman.io" | bash
    # Cambie roac por su nombre de usuario
    source "/home/roac/.sdkman/bin/sdkman-init.sh"
  \end{bash}

  Como es costumbre, podemos probar que se instaló correctamente con:

  \begin{bash}
    sdk version
  \end{bash}

  Con \textit{SDKMAN!} instalado, obtener \textit{Kotlin} es tan simple como:

  \begin{bash}
    sdk install kotlin
  \end{bash}

  Y comprobamos la instalación haciendo:
  
  \begin{bash}
    kotlinc -version
  \end{bash}