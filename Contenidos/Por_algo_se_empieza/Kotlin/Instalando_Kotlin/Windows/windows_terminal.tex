\subsubsection{Windows Terminal}
  Para instalar lo que necesitaremos basta con tener \textit{Powershell}, la terminal integrada de
  \textit{Windows} (por favor, NUNCA usen \textit{cmd}), pero actualmente \textit{Windows} provee
  una nueva terminal llamada \textit{Windows Terminal}\index{Windows Terminal}\footnote{
    Es increíble pensar que a alguien le pagaron por idear ese nombre.
  } (WT).
  \textit{WT} viene instalada en las 
  versiones más nuevas de \textit{Windows 10} y en todas las versiones de \textit{Windows 11}.
  Pueden usar el \textit{Buscador de Windows} para abrir la aplicación \enquote{\textit{Terminal}}
  o, si quieren hacerle creer a su abuela que son capaces de hackear la NASA,\footnote{En caso de 
  una investigación por parte de cualquier entidad federal o similar, el autor de este libro no 
  tiene ninguna participación con este grupo ni con las personas que lo integran, no sé cómo 
  estoy aquí, probablemente agregado por un tercero, no apoyo ninguna acción por parte de los 
  lectores de este libro.} pueden abrir \textit{Powershell} y ejecutar la instrucción:
  \begin{powershell}
    wt.exe
  \end{powershell}

  Si no encuentran la aplicación o si el comando anterior arrojó un error, pueden instalar la 
  terminal usando \textit{winget}, para esto deben ejecutar el siguiente comando en 
  \textit{Powershell}:
  \begin{powershell}
    winget install Microsoft.WindowsTerminal
  \end{powershell}

  A partir de ahora, cada vez que hable de \textit{Powershell} me estaré refiriendo a una sesión
  de \textit{Powershell} corriendo en \textit{Windows Terminal}.

  \begin{tcolorbox}[enhanced, breakable, title=\textit{Oh-My-Posh}]
    Utilizar \textit{Windows Terminal} en lugar de la \textit{terminal de Windows} (jaja) por 
    defecto es una mejora, pero podemos mejorar aún más.
    Una de las ventajas de WT es la capacidad de personalización que otorga a sus usuarios, esto 
    podemos hacerlo a mano o usar alguna herramienta que lo haga por nosotrxs, aquí les voy a 
    mostrar una de esas herramientas.

    \textit{Oh-My-Posh}\index{Oh-My-Posh} es una herramienta para personalizar \textit{Windows 
    Terminal} de forma simple, como les voy a mostrar en un instante.
    Para instalar \textit{Oh-My-Posh} abran WT y ejecuten el siguiente comando:
    \begin{powershell}
      winget install JanDeDobbeleer.OhMyPosh -s winget
    \end{powershell}

    Ahora que tenemos \textit{Oh-My-Posh} instalado, hay que decirle a \textit{Powershell} que lo 
    use, aquí voy a usar \textit{Notepad} porque viene instalado con todas las versiones de 
    \textit{Windows}, cosa que espero no volver a hacer en la vida.
    En PS:

    \begin{powershell}
      # Para poder ejecutar scripts (sdwheelerSetExecutionPolicyMicrosoftPowerShell)
      Set-ExecutionPolicy -ExecutionPolicy RemoteSigned -Force
      # Creamos el archivo sólo si no existe
      if (-not (Test-Path $PROFILE)) {
        New-Item -ItemType File -Path $PROFILE -Force
      }
      notepad.exe $PROFILE
    \end{powershell}

    Luego escribamos al comienzo del archivo el siguiente código:
    \begin{powershell}
      oh-my-posh.exe init pwsh `
        --config -join($env:POSH_THEMES_PATH,
          "\powerlevel10k_rainbow.omp.json") | `
        Invoke-Expression
    \end{powershell}
    
    Lo último que necesitas es instalar alguna de las fuentes de \url{https://www.nerdfonts.com},
    ahí descomprimen el archivo, seleccionan todos los archivos descomprimidos y con clic derecho
    debiera darles la opción de instalar.
    
    Ahora la próxima vez que abran la terminal podrán ver los cambios que hicimos.
    Si quieren darle una identidad propia a su terminal pueden revisar la 
    \href{https://ohmyposh.dev/docs}{documentación} de \textit{Oh-My-Posh}.
  \end{tcolorbox}
