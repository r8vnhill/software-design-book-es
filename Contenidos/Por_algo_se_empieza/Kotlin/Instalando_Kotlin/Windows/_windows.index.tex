\subsection{\textit{Windows}}
  \subimport{.}{winget.tex}
  \subimport{.}{windows_terminal.tex}
  \subsubsection{Chocolatey}
    Lo primero que necesitaremos para instalar las herramientas que usaremos será un gestor de 
    paquetes, utilizaremos \textit{Chocolatey}.\index{Chocolatey}
  
    Para partir abran una ventana de \textit{Powershell} como administrador.
    Una vez abierta, deben ejecutar las instrucciones:\footnote{
      \url{https://github.com/islaterm/software-design-book-es/blob/master/install/windows/chocolatey.ps1}
    }
    \begin{powershell}
      [Net.ServicePointManager]::SecurityProtocol = `
        [Net.SecurityProtocolType]::Tls12
      Set-ExecutionPolicy -ExecutionPolicy RemoteSigned -Force
      Invoke-WebRequest "https://chocolatey.org/install.ps1" `
        -UseBasicParsing | Invoke-Expression
    \end{powershell}
  
    Esto otorgará los permisos necesarios y descargará e instalará el gestor de paquetes.
  
    Para comprobar que el programa se haya instalado correctamente corran el comando:
    \begin{powershell}
      choco -?
    \end{powershell}
    %%endregion  
    Ahora toca instalar el \textit{JDK}, con chocolatey esto es simple, solamente deben ejecutar:
    \begin{powershell}
      choco install openjdk -y
    \end{powershell}
    
    Luego, para ver que se haya instalado correctamente pueden hacer \texttt{java -version}, aquí es
    muy importante que la versión que aparezca \textbf{no sea} la versión 1.8 o anteriores, en caso 
    de que esa sea la versión instalada entonces lo recomendado es desinstalar todas las versiones
    de \textit{Java} instaladas y repetir la instalación.
  
    ¡Perfecto!
    Ahora sólo nos falta instalar el compilador de \textit{Kotlin}.
    Nuevamente, para instalar el compilador confiaremos en nuestro amigo 
    \textit{Chocolatey}\index{Chocolatey}.
    Podemos instalar el compilador desde \textit{Powershell} con el comando:

    \begin{powershell}
      choco install kotlinc
    \end{powershell}

    Luego, podemos comprobar la instalación con:
    \begin{powershell}
      kotlinc -version
    \end{powershell}