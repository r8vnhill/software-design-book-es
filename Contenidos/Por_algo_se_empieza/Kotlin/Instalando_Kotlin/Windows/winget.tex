\subsubsection{\textit{Windows Package Manager} (\textit{winget})}
  Comencemos por una herramienta esencial para trabajar en \textit{Windows}.
  \textit{Windosw Package Manager}\autocite{kevinlamsUseWingetTool} es el \textit{gestor de 
  paquetes de Windows}.
  \begin{center}
    \textit{Increíble}
  \end{center}
  Como el nombre anterior es demasiado largo, lo llamaremos \textit{winget} y esperaremos que no 
  le moleste.

  El gestor \textit{winget} viene instalado por defecto en \textit{Windows 11} y en las versiones
  más recientes de \textit{Windows 10}, para ver que esté instalado necesitaremos abrir 
  \textit{Powershell} (PS) y ejecutar el siguiente comando:
  \begin{powershell}
    winget -?
  \end{powershell}
  Si el comando anterior no da error significa que \textit{winget} está instalado, si es así
  pueden pasar a la siguiente sección.

  ¿Ya se fueron?
  
  Ok, ahora veremos cómo instalar \textit{winget} en caso de que no lo tengan instalado en su
  sistema.

  Pongan atención para no perderse.

  Prepárense.

  Para instalar \textit{winget} deben acceder a la 
  \href{https://apps.microsoft.com/store/detail/app-installer/9NBLGGH4NNS1}{
    \textit{Tienda de Windows}
  } y descargar el \textit{Instalador de aplicación}.

  ¡Perfecto! Ya completamos el primer paso para aparentar ser el hacker que siempre quisiste.