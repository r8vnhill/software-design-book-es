\section{Ejercicios}
  \begin{Exercise}[title={Identificando tipos}]
    \Question Para cada una de las siguientes expresiones, indique el tipo de la variable 
    \texttt{x}:
      \subQuestion \mintinline{kotlin}{var x = true}
      \subQuestion \mintinline{kotlin}{var x = 1}
      \subQuestion \mintinline{kotlin}{var x = "1.0"}
      \subQuestion \mintinline{kotlin}{var x = 1.0}
      \subQuestion \mintinline{kotlin}{var x = Unit}
      \subQuestion \mintinline{kotlin}{var x = IntArray(10)[0]}
      \subQuestion \mintinline{kotlin}{var x = "${IntArray(10)}}
    \Question Investigue qué son los siguientes tipos y de un ejemplo de uso de cada uno:
      \subQuestion \textit{Float}
      \subQuestion \textit{Long}
      \subQuestion \textit{Char}
      \subQuestion \textit{Byte}
      \subQuestion \textit{Short}
      \subQuestion \textit{Long}
    \ExeText Utilice la terminal interactiva de \textit{Kotlin} (\texttt{kotlinc}) para verificar 
      sus respuestas.
    \Question Utilizando la notación \(FT(A_1, \dots, A_n) \rightarrow R\) indique el tipo de las
      siguientes funciones, reemplazando \texttt{\_} por el tipo correspondiente:
      \subQuestion
        \begin{minted}{kotlin}
          fun f1(x: Int, y: Int): _ {
            return x + y
          }
        \end{minted}
      \subQuestion
        \begin{minted}{kotlin}
          fun f2(x: Double): _ {
            println(x)
          }
        \end{minted}
      \subQuestion 
        \begin{minted}{kotlin}
          fun f3(x: Double, y: Double): _ {
            return println(x + y)
          }
        \end{minted}
      \subQuestion
        \begin{minted}{kotlin}
          fun f4(x: Int, y: Int): _ {
            return x + y
            println(x + y)
          }
        \end{minted}
    \Question Indique dos tipos inmutables y uno mutable.
  \end{Exercise}

  \begin{Exercise}[title={¿Qué imprime?}]
    Indique si las siguientes instrucciones arrojan un error o no, y si no lo hacen, indique qué
    imprime cada una al ser ejecutadas en la terminal interactiva.

    \Question 
      \begin{minted}{kotlin}
        val x = 1
        val y = 2
        val z = 3
        println(x + y + z)
      \end{minted}
    \Question 
      \begin{minted}{kotlin}
        val x = 0
        while (x < 10) {
          println(x)
          x = x + 1
        }
      \end{minted}
    \Question
      \begin{minted}{kotlin}
        val x = 0
        for (i in x..10) {
          println(i)
        }
      \end{minted}
    \Question
      \begin{minted}{kotlin}
        fun jojoReference(name: String): Unit {
          return Unit
          println("My name is " + name + ", and I have a dream.")
        }
      \end{minted}
    \Question \begin{minted}{kotlin}
        fun f(x: Int): Double {
          return x
        }
      \end{minted}
    \Question \begin{minted}{kotlin}
        val a = for (i in 0..10) {
          i
        }
      \end{minted}
  \end{Exercise}

  \begin{Exercise}[title={Fibonacci}]
    \Question Simplifique la implementación de la función recursiva de Fibonacci usando if-else como 
      expresión. 
      Hint: utilice una expresión de la forma \mintinline{kotlin}|return if (condicion) {expresion1} else {expresion2}|.

    \Question ¿Qué sucede si se llama a la función de Fibonacci con un número negativo?
      ¿Cómo se puede solucionar este problema?
      Discuta, no debe implementar la solución.
  \end{Exercise}