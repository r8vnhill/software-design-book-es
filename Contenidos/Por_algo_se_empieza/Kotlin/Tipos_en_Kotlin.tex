\section{Tipos en \textit{Kotlin}}
  Al igual que \textit{Java}, \textit{Kotlin} es un lenguaje de tipado estático y fuerte, pero con 
  una inferencia de tipos mucho más poderosa que la de \textit{Java}.
  El sistema de tipos de \textit{Kotlin} es un tema sumamente complejo que se escapa del alcance de 
  este libro, así que nos limitaremos a dar una pequeña introducción a éste y lo iremos
  desarrollando un poco más a lo largo del libro a medida que vaya siendo necesario.

  Lo primero que debemos saber es que \textit{Kotlin} tiene dos tipos de variables: 
  \texttt{val}ores y \texttt{var}iables.

  Un \textbf{valor}\footnote{Término propio} (o \textit{read-only variable}) es una variable que no 
  puede ser \textbf{reasignada} una vez que se le ha asignado un valor.
  Esto es más fácil entenderlo con un ejemplo, lo primero es abrir la terminal interactiva de
  \textit{Kotlin} con el comando \texttt{kotlinc} y escribir lo siguiente:

  \begin{kotlin}
    val i: Int = 8000
    i = 9000
  \end{kotlin}

  Como podemos ver, al intentar reasignar el valor de \texttt{i} obtenemos un error, esto es porque
  \texttt{i} es un valor y no una variable, por lo que no puede ser reasignada.
  El error que obtenemos es el siguiente:
  \begin{minted}{text}
    error: val cannot be reassigned
    i = 9000
    ^
  \end{minted}

  \begin{important}
    No debemos confundir un valor de solo lectura con una constante, ya que incluso si una variable
    es declarada como \texttt{val} puede ser modificada si su tipo es mutable.
    Volveremos a esto más adelante.
  \end{important}

  Por otro lado, una \textbf{variable} puede ser reasignada, para ver esto basta cambiar 
  \texttt{val} por \texttt{var} en la declaración de \texttt{i}:

  \begin{kotlin}
    var i: Int = 8000
    i = 9000
  \end{kotlin}

  En este caso, el código se ejecuta sin problemas c:

  \subsection{Tipos explícitos}
    \index{Sistema de tipos}
    \textbf{Todas} las variables en \textit{Kotlin} deben tener un tipo 
    definido\index{Sistema de tipos}, y una vez definido no puede ser cambiado.\footnote{
      Este tipo de atrocidades sólo pueden suceder en lenguajes de tipado dinámico como 
      \textit{Python} o \textit{Ruby}.
    }
    
    El tipo de una variable se define después del nombre de la variable y antes del valor que se le
    asigna, y se separa de éste con dos puntos (\texttt{:}).
    Podemos ver esto en el ejemplo anterior, donde definimos el tipo de \texttt{i} como 
    \texttt{Int}.

    Existen muchos tipos en \textit{Kotlin}, pero por ahora nos limitaremos a los más básicos:
    
    \begin{itemize}
      \item \texttt{Int}: El conjunto de todos los números enteros de 32 bits. 
        Formalmente: \(i: Int \iff i \in \mathbb{Z} \land -2^{31} \leq i \leq 2^{31} - 1\).
      \item \texttt{Double}: El conjunto de todos los números reales de \textit{doble 
        precisión}.\footnote{\url{https://en.wikipedia.org/wiki/Double-precision_floating-point_format}}
        En este caso, la definición formal es un poco más complicada, pero es cercana a:
        \(d: Double \iff d \in \mathbb{R} \land 4.9 \times 10^{-324} \leq |d| \leq 1.8 \times 10^{308}\).
      \item \texttt{String}: El conjunto de todas las cadenas de caracteres.
        Formalmente: \(s: String \iff s \in \mathbb{L}^*\), donde \(\mathbb{L}\) es el conjunto de
        todos los caracteres \textit{Unicode} que pueden ser representados en \textit{UCS-2}.
      \item \texttt{Boolean}: El conjunto de dos valores: \texttt{true} y \texttt{false}.
      \item \texttt{Unit}: El conjunto de un único valor: \texttt{Unit}.
        Esta noción es un poco más complicada, pero se puede entender como el tipo de variables que
        no almacenan información.
        Otro nombre común para este tipo es \textit{void}, pero no utilicemos ese nombre porque
        \textit{Kotlin} no tiene un tipo \textit{void}.
    \end{itemize}

    \begin{important}
      Noten que todos los tipos en \textit{Kotlin} empiezan con mayúscula, lo que debemos respetar a 
      la hora de crear nuestros propios tipos.
    \end{important}

  \subsection{Tipos inferidos}
    \index{Sistema de tipos}
    Como mencionamos en la sección anterior: \enquote{\textbf{Todas} las variables en 
    \textit{Kotlin} deben tener un tipo definido}.
    Sin embargo, \textit{Kotlin} tiene una inferencia de tipos bastante poderosa, por lo que en la 
    mayoría de los casos no es necesario definir explícitamente el tipo de una variable.
    Esto es muy útil, ya que va a reducir muchísimo la cantidad de texto que tenemos que escribir
    para hacer nuestros programas; esto también va a hacer que nuestros programas sean más fáciles
    de leer y entender al reducir la cantidad de código \textit{boilerplate} que tenemos que 
    procesar.
    Créanme cuando les digo que la mayor parte del tiempo que trabajen lo van a pasar leyendo 
    código.

    Pero ES MUY IMPORTANTE no abusar de la inferencia de tipos, ya que esto puede terminar teniendo
    el efecto contrario al que queremos: hacer que nuestros programas sean más difíciles de leer y
    entender, i.e. gastar más tiempo leyendo D:

    Para ver cómo funciona la inferencia de tipos, podemos reescrbir el ejemplo anterior de la
    siguiente manera:

    \begin{kotlin}
      var i = 8000
      i = 9000
    \end{kotlin}
    
    No vamos a ahondar más sobre este tema, ya que sería extender innecesariamente este capítulo (y
    porque no soy experto en el tema), pero a lo largo del libro iremos viendo ejemplos de cómo
    usar la inferencia de tipos correctamente, en que casos podemos usarla, y en cuales no.

    \begin{exercise}
      Indique qué tipo de dato es cada una de las siguientes variables:

      \begin{enumerate}
        \item \texttt{var i = true}
        \item \texttt{var i = 8000}
        \item \texttt{var i = "8000"}
        \item \texttt{var i = Unit}
        \item \texttt{var i = 8000.0}
      \end{enumerate}

    \end{exercise}

  \subsection{Funciones}
    Un último tipo de dato que vamos a ver en este capítulo son las funciones.
    Las funciones son un tipo de dato que toma una cantidad de parámetros y devuelve un valor.
    Formalmente, sea \(FT\) el tipo de las funciones, tenemos que:
    \[
      FT(A_1, \dots, A_n) \rightarrow R
    \]
    con \(A_i\) los tipos de los parámetros de la función, y \(R\) el tipo de retorno de la función.

    En \textit{Kotlin} tenemos dos formas de definir funciones: utilizando la palabra reservada
    \texttt{fun}\footnote{Trad. Diversión} o utilizando una expresión lambda.
    En esta sección veremos sólo la primera forma, y dejaremos la segunda para más adelante.
    
    \subsubsection{Lo más básico}
      Una función se define de la siguiente forma:

      \begin{minted}{text}
        fun <nombre>(<parámetros>): <tipo de retorno> {
          <cuerpo de la función>
        }
      \end{minted}

      Ahora, podemos definir una función simple que imprima un mensaje en pantalla:

      \begin{kotlin}
        fun jojoReference(name: String): Unit {
          println("My name is " + name + ", and I have a dream.")
          return Unit
        }
      \end{kotlin}
      
      Veamos qué está pasando en este ejemplo:

      \begin{itemize}
        \item La palabra reservada \texttt{fun} nos indica que estamos definiendo una función.
        \item El nombre de la función es \texttt{jojoReference}.
          Noten que el nombre de la función está escritoAsí, esto se llama \textit{camelCase} y es
          una convención que se usa en \textit{Kotlin} para nombrar variables y funciones.
        \item La función toma un parámetro, que se llama \texttt{name} y es de tipo \texttt{String}.
        \item No vamos a utilizar el resultado de la función, así que podemos definir el tipo de 
          retorno como \texttt{Unit}.
        \item El cuerpo de la función es el bloque de código que se ejecuta cuando llamamos a la
          función.
          En este caso, la función imprime un mensaje en pantalla (utilizando la función 
          \texttt{println(String): Unit}\footnote{El nombre viene de \textit{\textbf{print} 
          \textbf{l}i\textbf{n}e}, ya que lo que hace es imprimir una línea de texto en la pantalla, 
          i.e. texto seguido de un salto de línea.}).
        \item La función termina con la palabra reservada \texttt{return}, que indica que la función
          termina su ejecución y devuelve el valor que se encuentra después de la palabra reservada.
          En este caso, la función devuelve el valor \texttt{Unit}.   
      \end{itemize}

      Ahora, podemos llamar a la función de la siguiente manera:

      \begin{kotlin}
        jojoReference("Giorno Giovanna")
      \end{kotlin}

    \subsubsection{Lo otro más básico}
      Pasemos a la parte que a nadie le gusta, documentar el código.
      Para documentar algo en \textit{Kotlin} escribiremos un comentario que comience con \texttt{/**}
      y termine con \texttt{*/}.
      Dentro de este comentario escribiremos qué hace la función, qué parámetros toma, y qué valor
      devuelve.
      En el ejemplo anterior podríamos escribir lo siguiente:

      \begin{kotlin}
        /**
         * Prints a Jojo's Bizarre Adventure reference to the console with the given name.
         */
        fun jojoReference(name: String): Unit {
          println("My name is " + name + ", and I have a dream.")
          return Unit
        }
      \end{kotlin}
      
      Es de vital importancia documentar el código, ya que el código que escribimos hoy puede ser
      utilizado por nosotros o por otras personas dentro de un año, y si no documentamos el código
      no vamos a saber qué hace cada función, y esto puede hacer que tengamos que botar código a la
      basura.

      En el resto del libro omitiremos la documentación en la mayoría de los ejemplos para no
      extender innecesariamente el texto, pero esto lo haremos sólo porque el iremos explicando el
      código en cada ejemplo, pero en la vida real \textbf{siempre} debemos documentar el código.

    \subsubsection{Lo no tan básico}
      Lo último que nos queda es simplificar el código que escribimos en el ejemplo anterior porque
      siempre podemos hacerlo mejor.
      
      Primero, como no utilizaremos el valor de retorno de la función, podemos omitir la última 
      línea.
      
      Luego, podemos utilizar la característica llamada \textit{string interpolation} para
      simplificar la concatenación de cadenas de texto.
      La \textit{interpolación de cadenas} es una característica que nos permite insertar variables
      dentro de una cadena de texto, y \textit{Kotlin} nos permite hacer esto con el símbolo \$.
      Por ejemplo, si tenemos una variable \texttt{name} de tipo \texttt{String} y queremos
      concatenarla con una cadena de texto, podemos hacerlo de las siguientes maneras:

      \begin{kotlin}
        val name = "Giorno Giovanna"
        println("My name is " + name + ", and I have a dream.")
        println("My name is $name, and I have a dream.")
      \end{kotlin}

      Como podemos ver, la segunda forma es mucho más simple y legible que la primera.

      Por último, si el tipo de retorno de la función es \texttt{Unit}, podemos omitirlo.

      Con esto, podemos escribir la función de la siguiente manera:

      \begin{kotlin}
        fun jojoReference(name: String) {
          println("My name is $name, and I have a dream.")
        }
      \end{kotlin}

      ¿Podemos simplificarlo más?
      Sí, pero dejaremos eso para más adelante ;)