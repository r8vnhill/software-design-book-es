
\subsection{Implementación básica (recursiva)}
  La implementación más sencilla es la recursiva, esta es la que se muestra en la ecuación
  \ref{eq:Funcion_de_Fibonacci}.

  Empecemos por entender un poco lo que es una instrucción de control de flujo.

  \begin{defaultbox}[Instrucción de control de flujo]
    Una instrucción de control de flujo\index{Control de flujo} es una instrucción que controla el 
    orden en el que se ejecutan las instrucciones de un programa.
  \end{defaultbox}

  En \textit{Kotlin} existen cuatro tipos de instrucciones de control de flujo:

  \begin{itemize}
    \item \textit{if-else}: es una \textbf{expresión} que evalúa una condición y ejecuta una 
      instrucción si la condición es verdadera y otra si es falsa.
    \item \textit{when}: es una \textbf{expresión} que evalúa una condición y ejecuta una 
      instrucción dependiendo del valor de la condición.
      Pueden pensar en esto como una versión más potente del \textit{if-else}.
    \item \textit{for}: es una \textbf{instrucción} que ejecuta una directiva un número determinado 
      de veces.
    \item \textit{while}: es una \textbf{instrucción} que ejecuta una directiva mientras una 
      condición sea verdadera.
  \end{itemize}

  Para la implementación recursiva de la función de Fibonacci vamos a usar la expresión 
  \textit{if-else}.
  La implementación de la función de Fibonacci en \textit{Kotlin} es la siguiente:

  \begin{kotlin}
    fun recursiveFibonacci(n: Int): Int {
      if (n <= 1) {
        return n
      } else {
        return recursiveFibonacci(n - 1) + recursiveFibonacci(n - 2)
      }
    }
  \end{kotlin}

  Veamos lo que está pasando:

  \begin{itemize}
    \item La función \textit{recursiveFibonacci} recibe un número entero y devuelve un
      número entero.
    \item La función evalúa si el número recibido es menor o igual a 1.
    \item Si el número es menor o igual a 1, la función devuelve el número recibido.
    \item Si el número es mayor a 1, la función devuelve la suma de la función llamada con el
      número recibido menos 1 y la función llamada con el número recibido menos 2.
  \end{itemize}

  Algo importante sobre las expresiones \textit{if-else} y \textit{when} es que son expresiones

  \begin{center}
    \textit{omg!}
  \end{center}

  esto significa que pueden ser evaluadas y devolver un valor, a esto se le llama reducción.
  Con esto en mente, veremos que podemos utilizar la expresión \textit{if-else} para asignar
  valores a variables.
  Veamos un ejemplo:

  \begin{kotlin}
    fun printIsPositive(n: Int) {
      val isPositive = if (n > 0) {
        true
      } else {
        false
      }
      println(isPositive)
    }
  \end{kotlin}

  Recordemos que si el tipo de retorno es \texttt{Unit} no es necesario especificarlo.

  \begin{exercise}
    Simplifique la implementación de la función de Fibonacci usando \textit{if-else} como 
    expresión.
    \textit{Hint: utilice una expresión de la forma \texttt{return if (condicion) \{expresion1\} 
    else \{expresion2\}}.}
  \end{exercise}

  \begin{exercise}
    ¿Qué sucede si se llama a la función de Fibonacci con un número negativo?
    ¿Cómo se puede solucionar este problema?
    Discuta, no debe implementar la solución.
  \end{exercise}