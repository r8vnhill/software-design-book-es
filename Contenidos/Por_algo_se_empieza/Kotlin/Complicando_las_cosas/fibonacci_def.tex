\subsection{La función de Fibonacci}
  \label{subsec:La_funcion_de_Fibonacci}

  La función de Fibonacci es uno de los ejemplos más conocidos de una función recursiva (función 
  que se llama a sí misma). 
  Se define de la siguiente manera:

  \begin{equation}
    \label{eq:Funcion_de_Fibonacci}
    f(n) = \begin{cases}
      0 & \text{si } n = 0 \\
      1 & \text{si } n = 1 \\
      f(n - 1) + f(n - 2) & \text{si } n > 1
    \end{cases}
  \end{equation}

  En las secciones siguientes vamos a ver cómo implementar esta función en Kotlin usando distintas
  técnicas.