\subsection{Programación dinámica}
  La programación dinámica es una técnica de programación que consiste en guardar los resultados
  intermedios de una función para evitar calcularlos nuevamente.
  En el caso de la función de Fibonacci, sabemos que para
  calcular el valor de \texttt{f(6)} necesitamos calcular el valor de \texttt{f(5)} y \texttt{f(4)}.
  Si guardamos el valor de \texttt{f(5)} podemos calcular el valor de \texttt{f(6)} sin tener que
  calcular nuevamente el valor de \texttt{f(5)}.
  De esta manera, podemos calcular el valor de \texttt{f(6)} sin tener que calcular el valor de
  \texttt{f(5)} ni el valor de \texttt{f(4)}.
  
  Para implementar la programación dinámica vamos a utilizar un arreglo de enteros, este es un nuevo
  tipo de dato de \textit{Kotlin}.
  Un arreglo es una colección de elementos del mismo tipo, podemos crear un arreglo de enteros de la
  siguiente manera:
  
  \begin{kotlin}
    val array = IntArray(10)
  \end{kotlin}

  En este caso creamos un arreglo de 10 elementos de tipo entero.

  Ahora volvemos al tema de \textbf{mutabilidad}.

  \begin{defaultbox}[mutabilidad]
    Un objeto es mutable si su estado puede cambiar después de ser creado, en constraste, un objeto 
    es inmutable si su estado no puede cambiar después de ser creado.
  \end{defaultbox}

  Un arreglo es un tipo de dato mutable, es decir, podemos cambiar la información que contiene.

  Para acceder a un elemento del arreglo utilizamos el operador \texttt{[]}, por ejemplo:

  \begin{kotlin}
    val array = IntArray(3)
    array[0] = 1  // El primer elemento del arreglo es el elemento 0
    array[1] = 2
    array[2] = 3  // El último elemento del arreglo es el elemento 2
  \end{kotlin}

  Si intentamos acceder a un elemento que no existe, por ejemplo \texttt{array[3]}, obtendremos un
  error.

  Con esto podemos implementar la función de Fibonacci usando programación dinámica:

  \begin{kotlin}
    fun dynamicFibonacci(n: Int): Int {
      val memo = IntArray(n + 1)
      memo[0] = 0
      memo[1] = 1
      for (i in 2..n) {
        memo[i] = memo[i - 1] + memo[i - 2]
      }
      return memo[n]
    }
  \end{kotlin}

  Con esto ya estamos listxs para enfrentar al mundo.
