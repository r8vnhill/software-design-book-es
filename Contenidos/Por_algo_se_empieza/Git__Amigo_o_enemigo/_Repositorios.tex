\section{¿Repositorios?}
  Uno de los conceptos claves de \textit{Git} son los repositorios\index{Repositorio}.
  En la práctica, un repositorio no es más que una carpeta en su computador, lo que lo hace especial
  es cómo usamos esa carpeta.

  En el caso de \textit{Git}, un repositorio va a ser una carpeta que contiene un directorio con una
  carpeta llamada \texttt{.git} con ciertos archivos de configuración especiales.
  Podemos crear un repositorio de forma simple usando el comando \idx{\texttt{git init}}.
  El siguiente ejemplo crea una carpeta y luego la configura como un repositorio de \textit{Git}.

  \begin{powershell}
    mkdir git_example
    cd git_example
    git init
  \end{powershell}

  Ahora, podremos ver que se creó una carpeta \texttt{.git} dentro de \texttt{git\_example}, sin 
  embargo, esta carpeta en general está oculta.
  Para ver la carpeta desde la terminal podemos hacer:
  
  \begin{tcolorbox}[enhanced, breakable, title=Powershell]
    \begin{powershell}
      # Desde `git_example`
      Get-ChildItem . -Hidden
    \end{powershell}
  \end{tcolorbox}

  \begin{tcolorbox}[enhanced, breakable, title=Bash]
    \begin{bash}
      # Desde `git_example`
      ls -a .
    \end{bash}
  \end{tcolorbox}

  Ya tenemos nuestro repositorio, ahora hay que sacarle provecho.
  Lo primero que haremos será conocer a su nuevo comando favorito: 
  \idx{\texttt{git status}}.
  Este comando nos servirá para ver el estado del repositorio en cualquier momento, si ahora lo 
  ejecutamos en nuestro repositorio debiera decir:
  \begin{minted}{text}
    On branch master

    No commits yet

    nothing to commit (create/copy files and use "git add" to track)
  \end{minted}

  Aca vemos muchos conceptos nuevos: \textit{branch}, \textit{master}, \textit{commit}, etc.
  Vamos a comenzar con lo más básico y haremos nuestro primer \textit{commit}.

  \subsection{\textit{Commits}}
    Tomemos un pequeño desvío de la computación y pensemos en un libro de historia.
    Los libros de historia registran sucesos históricos para que luego puedan ser estudiados por 
    otrxs que estén interesados en conocer y analizar estos sucesos.
    ¿Pero qué es un suceso historico?
    Podríamos argumentar que todo lo que sucede en nuestro día a día es un suceso histórico, pero no
    todo lo que sucede en el mundo será relatado en los libros de historia, alguien (historiadores 
    principalmente) tiene que decidir qué sucesos serán registrados.

    Podemos pensar nuestro repositorio como un libro de historia, cada cambio que hagamos dentro del
    repositorio puede ser un suceso histórico, pero no todos esos cambios debieran ser registrados.
    A la acción de registrar un \enquote{suceso} en nuestro repositorio le llamaremos \idxit{Commit}.

    Veamos cómo poner esto en práctica.
    Primero, crearemos un archivo dentro del repositorio que se llame \texttt{README.md} con el 
    siguiente contenido:

    \begin{md}
      # Mi pequeño gran repositorio

      **SIEMPRE** deben incluir un _readme_ en sus repositorios.
    \end{md}

    Este archivo es bastante particular, pero también estándar.
    Una buena práctica es siempre agregar un archivo \texttt{README.md} en nuestros repositorios,
    esto le servirá a todxs lxs que tengan que trabajar después con nuestros repositorios.
    El formato \texttt{.md} indica que el archivo utiliza el lenguaje \idxit{Markdown}.
    No entraremos en detalles en la sintáxis de \textit{Markdown}, para esto pueden referirse a las
    bibliografías.\autocite{gh-markdown}

    Ahora estamos listos para hacer un \textit{Commit}, pero primero veamos el estado del
    repositorio con \texttt{git status}, esto debiera resultar en:

    \begin{minted}{text}
      On branch master

      No commits yet

      Untracked files:
        (use "git add <file>..." to include in what will be committed)
              README.md

      nothing added to commit but untracked files present (use "git add" to track)
    \end{minted}

    Como es esperable, el mensaje sigue diciendo que no hay \textit{commits}, pero ahora aparece
    una nueva sección \textit{Untracked files} con el archivo que creamos.
    Los \textit{untracked files} son los archivos que no hemos registrado en nuestro libro de
    historia y, como veremos un poquito más adelante veremos, no siempre vamos a querer registrar
    todos nuestros cambios, pero es sumamente importante nunca dejar \textit{untracked files} 
    \enquote{sin resolver}.

    Ahora, pongámosle énfasis a la última línea del mensaje: \enquote{\textit{nothing added to 
    commit}}; esto significa que no le hemos dicho a \textit{Git} qué cambios son los que debe 
    registrar.
    Podemos agregar estos cambios con el comando \idx{\texttt{git add}} de la siguiente forma:

    \begin{bash}
      git add README.md
    \end{bash}

    De nuevo, podemos hacer \texttt{git status} para ver los cambios.

    Por último, nos queda hacer \textit{Commit} (omg!).
    Para esto usamos el comando \idx{\texttt{git commit}} pasándole un mensaje descriptivo para 
    identificar el \textit{commit}:

    \begin{bash}
      git commit -m "doc(repo): Created README"
    \end{bash}
