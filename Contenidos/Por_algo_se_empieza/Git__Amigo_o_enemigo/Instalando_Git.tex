\section{Instalando \textit{Git}}
  %region : WINDOWS  
 	\subsection{\textit{Windows}}
    \subsubsection{\textit{Chocolatey} (Recomendado)}
      Nuevamente, el método recomendado será utilizar \textit{Chocolatey}\index{Chocolatey} para 
      instalar \textit{Git} en \textit{Windows}\index{Windows}.
      
      El proceso de instalación es simple, solamente deben ejecutar 
      \textit{Powershell}\index{Powershell} como administrador y escribir:
      \begin{powershell}
        choco install git 
        # O de forma equivalente:
        # cinst git
      \end{powershell}
    
    \subsubsection{\textit{Git for Windows}}
      \textit{Git for Windows}\index{Git for Windows} es un conjunto de herramientas que incluye \textit{Git 
      BASH} (una interfaz de consola que emula la terminal de un sistema \textit{UNIX} 
      que viene con \textit{Git} instalado), \textit{Git GUI} (una interfaz gráfica para 
      manejar \textit{Git}) e integración con \textit{Windows Explorer} (esto significa 
      que pueden hacer \textit{clic} derecho en una carpeta y abrirla desde \textit{Git 
      BASH} o \textit{Git GUI}).
      
      Para instalarla deben descargar el cliente desde el 
      \href{https://gitforwindows.org}{sitio oficial} de \textit{Git for Windows} y 
      seguir las instrucciones del instalador.
  %endregion

  %region : LINUX
  \subsection{Linux}
    La forma más fácil de instalar \textit{Git} es utilizando el gestor de paquetes del
    sistema operativo que estén usando, el problema de esto es que dependiendo de su distribución
    de \textit{Linux}\index{Linux} las instrucciones de instalación serán distintas, por esto se 
    mostrará como instalar en arquitecturas basadas en \textit{Debian} (como \textit{Ubuntu}), para 
    instalar en otro SO deberán revisar las instrucciones de la 
    \href{https://git-scm.com/download/linux}{documentación oficial}.

    Para instalar deben abrir una consola y ejecutar:
    \begin{bash}
      sudo apt install git-all 
    \end{bash}
  %endregion

  %region : MAC
  \subsection{macOS}
    \subsubsection{\textit{Git} mediante \textit{Xcode CLT} (Recomendado)}
      La forma más simple de instalar \textit{Git} en sistemas \textit{macOS}\index{macOS} es a 
      través de \textit{Xcode CLT}\index{Xcode CLT} (¡No confundir con \textit{Xcode}!), en este 
      caso, lo primero que deberán hacer es instalar el \textit{CLT} desde una terminal:
      
      \begin{bash}
        xcode-select --install
      \end{bash}
      
      Una vez instalada esta herramienta\footnote{Pueden verificar que se haya instalaado de forma 
      correcta ejecutando \texttt{xcode-select -p}.} para instalar \textit{Git} bastará ejecutar el
      siguiente comando en la terminal:

      \begin{bash}
        git --version
      \end{bash}
    \subsubsection{\textit{Git} mediante \textit{Homebrew}}
      Otra forma de instalar \textit{Git} es utilizando \textit{Homebrew}, esto también es simple de
      hacer.
      En una terminal ejecuten:

      \begin{bash}
        brew install git
      \end{bash}

      Luego, para comprobar la instalación hacemos:

      \begin{bash}
        git --version
      \end{bash}
  %endregion