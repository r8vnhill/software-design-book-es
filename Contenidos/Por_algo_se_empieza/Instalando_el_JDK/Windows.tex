\subsection{Windows}
  \subsubsection{Chocolatey}
    Lo primero que necesitaremos para instalar las herramientas que usaremos será un gestor de 
    paquetes, utilizaremos \textit{Chocolatey}.\autocite{choco}

    Para partir abran una ventana de \textit{Powershell} como administrador.
    Una vez abierta, deben ejecutar las instrucciones:\footnote{
      \url{https://github.com/CC3002-Metodologias/apunte-y-ejercicios/blob/master/install/windows/chocolatey.ps1}
    }
    \begin{minted}{powershell}
      [Net.ServicePointManager]::SecurityProtocol = `
        [Net.SecurityProtocolType]::Tls12
      Set-ExecutionPolicy -ExecutionPolicy RemoteSigned -Force
      Invoke-WebRequest "https://chocolatey.org/install.ps1" `
        -UseBasicParsing | Invoke-Expression
    \end{minted}

    Esto otorgará los permisos necesarios y descargará e instalará el gestor de paquetes.

    Para comprobar que el programa se haya instalado correctamente corran el comando:
    \begin{minted}{powershell}
      choco -?
    \end{minted}

  \subsubsection{(Opcional) Cmder}
    Es sabido que las terminales por defecto de \textit{Windows} dejan bastante que desear, 
    por es una buena idea instalar una terminal externa (o más bien un emulador de una).
    Existen varias opciones, pero \textit{Cmder} es una de las más completas.

    Para instalar la terminal utilizaremos \textit{Chocolatey}.
    En una terminal de \textit{Powershell} con permisos de administrador ejecuten:\footnote{
      \url{https://github.com/CC3002-Metodologias/apunte-y-ejercicios/blob/master/install/windows/cmder.ps1}
    }

    \begin{minted}{powershell}
      cinst cmder -y
    \end{minted}

    Con esto basta para tener \textit{Cmder} instalado, pero una de las principales ventajas
    de utilizar este emulador de consola es la capacidad de personalizarlo.
    A partir de aquí continuaremos desde \textit{Cmder}.
    
    Lo siguiente será instalar herramientas que ayudarán a entregar de mejor forma la información 
    al momento de usar la consola.
    Para esto, deberán ejecutar los siguientes comandos:

    \begin{minted}{powershell}
      Install-PackageProvider NuGet -MinimumVersion '2.8.5.201' -Force
      Set-PSRepository -Name PSGallery -InstallationPolicy Trusted
      Install-Module -Name 'oh-my-posh'
      Install-Module -Name 'Get-ChildItemColor' -AllowClobber
    \end{minted}

    Lo último es configurar el perfil de \textit{Powershell}, esto se hace en un archivo que
    es el equivalente a \texttt{.bashrc} de los sistemas \textit{Linux}.
    Para abrir este archivo ejecuten:

    \begin{minted}{powershell}
      ise $PROFILE
    \end{minted}

  Esto abrirá el entorno de \textit{scripting} de \textit{Powershell} (si nunca han 
  configurado la consola, entonces debiera estar vacío).
  Como último paso, escriban lo siguiente en el archivo de configuración y guarden los 
  cambios.\footnote{
    \url{https://github.com/CC3002-Metodologias/apunte-y-ejercicios/blob/master/install/Microsoft.PowerShell_profile.ps1}
  }

  \begin{minted}{powershell}
    # Helper function to set location to the User Profile directory
    function cuserprofile { Set-Location ~ }
    Set-Alias ~ cuserprofile -Option AllScope

    Import-Module 'oh-my-posh' -DisableNameChecking

    # CHOCOLATEY PROFILE
    $ChocolateyProfile = `
      "$env:ChocolateyInstall\helpers\chocolateyProfile.psm1"
    if (Test-Path($ChocolateyProfile)) {
      Import-Module "$ChocolateyProfile"
    }

    Set-PSReadlineOption -BellStyle None
    Set-Theme Honukai
  \end{minted}

  La última línea solamente define el \textit{tema} de la consola, pueden ver una lista de
  \textit{temas} disponibles en el 
  \href{https://github.com/JanDeDobbeleer/oh-my-posh#themes}{repositorio de \textit{Oh-My-Posh}}

\subsubsection{OpenJDK con Chocolatey (Recomendado)}
  La primera opción es instalar la versión de código abierto del \textit{JDK}.
  Con chocolatey esto es simple, solamente deben ejecutar:
  \begin{minted}{powershell}
    cinst openjdk -y
  \end{minted}
  
  Luego, para ver que se haya instalado correctamente pueden hacer \texttt{java -version}, aquí es
  muy importante que la versión que aparezca \textbf{no sea} la versión 1.8 o anteriores, en caso 
  de que esa sea la versión instalada entonces lo recomendado es desinstalar todas las versiones
  de \textit{Java} instaladas y repetir la instalación.

\subsubsection{OpenJDK sin Chocolatey}
  Si no funciona, o no quieren usar el gestor de paquetes, también se puede instalar el 
  \textit{JDK} manualmente.
  
  Primero deben descargar los binarios del \textit{JDK} desde 
  \href{
    https://download.java.net/java/GA/jdk15.0.2/0d1cfde4252546c6931946de8db48ee2/7/GPL/openjdk-15.0.2_windows-x64_bin.zip
    }{aquí}.
  
  Con los binarios descargados, extraigan el \texttt{.zip} en algún directorio y luego 
  abran \textit{Powershell} como administrador y ubíquense en la carpeta donde extrajeron el 
  archivo.
  Una vez ahí, ejecuten:

  \begin{minted}{powershell}
    Move-Item -Path .\jdk-15 -Destination $Env:Programfiles
    [Environment]::SetEnvironmentVariable("JAVA_HOME", 
                                          "$Env:Programfiles\jdk-15")
    [Environment]::SetEnvironmentVariable(
      "Path", 
      [Environment]::GetEnvironmentVariable('Path', 
      [EnvironmentVariableTarget]::Machine) + ";$($Env:JAVA_HOME)\bin", 
      [EnvironmentVariableTarget]::Machine)
  \end{minted}

  Luego, pueden probar la instalación de la misma manera que en la opción anterior.

  \subsubsection{Oracle Java SE}
    Lo primero es descargar el \textit{JDK} desde el 
    \href{
      https://www.oracle.com/java/technologies/javase-jdk15-downloads.html
    }{sitio de \textit{Oracle}}. 
    Una vez descargado ejecuten el instalador y sigan las instrucciones.

    Por último, deben agregar el \textit{path} de \textit{Java} a las variables de 
    entorno, para esto abran \textit{Powershell} como administrador y ejecuten:

    \begin{minted}{powershell}
      [Environment]::SetEnvironmentVariable("JAVA_HOME", 
                                            "$Env:Programfiles\Java\jdk-15")
      [Environment]::SetEnvironmentVariable(
        "Path", 
        [Environment]::GetEnvironmentVariable('Path', 
        [EnvironmentVariableTarget]::Machine) + ";$($Env:JAVA_HOME)\bin", 
        [EnvironmentVariableTarget]::Machine)
    \end{minted}