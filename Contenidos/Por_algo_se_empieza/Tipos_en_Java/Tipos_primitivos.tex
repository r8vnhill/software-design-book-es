\subsection{Tipos primitivos}
  Okay, ahora sabemos que \textit{Java} tiene tipos, falta saber 
  cuáles son esos tipos.

  En \textit{Java} los tipos se pueden agrupar en dos categorías: los
  \textbf{tipos primitivos} y los \textbf{objetos}. 
  La mayor parte de este libro se enfocará en cómo utilizar y 
  aprovechar las propiedades de los objetos pero primero debemos 
  entender la diferencia que tienen estos de un tipo primitivo.

  Los \textit{tipos primitivos} son datos que ocupan una cantidad fija de espacio en la
  memoria y que pueden representarse directamente como valores numéricos.
  Esto hace que sean más eficientes de utilizar, ya que una vez que se les asigna un 
  lugar en la memoria difícilmente tendrán que moverse a otra dirección (algo que 
  sucederá mucho con los objetos).
  
  La tabla \ref{tab:primitive} muestra los tipos primitivos de \textit{Java} (en 
  \textit{Python} también existen los tipos primitivos pero la forma en que funcionan es más 
  compleja).

  \begin{table}[ht!]
    \centering
    \begin{tabular}{| c | m{5em} | m{7em} | m{12em} | c |}
      \hline
      Tipo                      & Uso de memoria  & Rango 
        & Uso                                               
        & Valor por defecto           \\
      \hline\hline
      \mintinline{java}{byte}     & 8-bits          & [-128, 127]  
        & Representar arreglos masivos de números pequeños  
        & \mintinline{java}{0}        \\\hline
      \mintinline{java}{short}    & 16-bits         & [-32.768, 32.767]
        & Representar arreglos grandes de números pequeños  
        & \mintinline{java}{0}        \\\hline
      \mintinline{java}{char}     & 16-bits         & [0, 65.535]
        & Caracteres del estándar \textit{ASCII}            
        & \mintinline{java}{'\u0000'} \\\hline
      \mintinline{java}{int}      & 32-bits         & \([-2^{31}, 2^{31} - 1]\)
        & Estándar para representar enteros               
        & \mintinline{java}{0}        \\\hline
      \mintinline{java}{long}     & 64-bit          & \([-2^{63}, 2^{63} - 1]\)
        & Representar enteros que no quepan en 32-bits
        & \mintinline{java}{0L}       \\\hline
      \mintinline{java}{float}    & 32-bit          & Estándar \textit{IEEE 754x32}
        & Representar números reales cuando la precisión no es importante
        & \mintinline{java}{0.0f}     \\\hline
      \mintinline{java}{double}   & 64-bit          & Estándar \textit{IEEE 754x64}
        & Representar números reales con mediana precisión
        & \mintinline{java}{0.0}      \\\hline
      \mintinline{java}{boolean}  & No definido     & \mintinline{java}{true| o |false}
        & Valores binarios
        & \mintinline{java}{false}    \\
      \hline
    \end{tabular}
    \caption{
      Tipos primitivos en \textit{Java} (los valores por defecto son los que toma la
      variable si no se le entrega un valor explícitamente \textbf{y es un campo} de una
      clase)
    }
    \label{tab:primitive}
  \end{table}

  
  \begin{important}
    Noten que la sintaxis en \textit{Java} para los valores \textit{boolean} es 
    \mintinline{java}{true| y |false}, mientras que en \textit{Python} es 
    \mintinline{python}{True| y |False}.
  \end{important}

  \subsubsection{\textit{Type promotion}}
    Los tipos primitivos nos permitirán hacer las operaciones más básicas que vamos a necesitar: suma,
    resta, multiplicación, comparación, etc.
    Pero como \textit{Java} es de tipado estático, todas las operaciones entre primitivos también
    deben tener un tipo definido.
    Un ejemplo de esto es que si sumamos dos números enteros el resultado también deberá ser un 
    entero.

    El \enquote{problema} surge cuando tenemos una expresión en la que no es totalmente evidente el 
    tipo de retorno.
    Para solucionar esto \textit{Java} implementa lo que se conoce como \textbf{\textit{type 
    promotion}}, que se puede resumir en los siguientes pasos:
    \begin{enumerate}
      \item Si hay valores de tipo \mintinline{java}{byte|, |short|, o |char} en una 
        operación el compilador los convierte en \mintinline{java}{int}
      \item Si todos los valores de la operación son del mismo tipo, entonces retornan un valor de ese 
        tipo; esto quiere decir que \texttt{double + double = double}, pero también que 
        \texttt{int / int = int}.
      \item Si quedan valores de distinto tipo, entonces se transforman al tipo \enquote{más grande}, 
        partiendo por \mintinline{java}{long}, luego \mintinline{java}{float} y, si no puede ser 
        ninguno de los anteriores, \mintinline{java}{double}.
    \end{enumerate}
  %

  \subsubsection{\textit{Wrappers}}
    Todos los tipos primitivos en \textit{Java} tienen un objeto \enquote{equivalente} 
    para brindar operaciones que no se pueden realizar directamente con los tipos 
    primitivos, más adelante veremos algunas de estas funcionalidades.
    A estos objetos equivalentes se les llama \textbf{\textit{wrappers}}.
    Algunos ejemplos de esto pueden ser \mintinline{java}{int} e \textit{Integer}, o 
    \mintinline{java}{char} y \mintinline{java}{Character}.
    Estos objetos, junto con los \textit{strings} son tipos especiales que pueden definirse y 
    utilizarse de una forma más simple que el resto de los objetos, algunos ejemplos:
    
    \begin{minted}{java}
      // Podemos declarar un ``Integer`` de la misma forma que haríamos
      // con un ``int``
      int a = 1;
      Integer b = 1;
      // Podemos operar `a` y `b` como si fueran del mismo tipo
      System.out.println(a + b);
      // Un caracter se declara entre comillas simples
      char c1 = 'c';
      // Mientras que un String con comillas dobles
      String c2 = "c";  // Noten que c1 es un primitivo y c2 un objeto
      // Podemos concatenar strings
      System.out.println("Ola" + "comotai");
      // Los strings pueden concatenarse con objetos de otros tipos; en
      // este caso, el resultado de la concatenación siempre será un 
      // String
      String s = 11 + ": entonces";
    \end{minted}

    \begin{exercise}
      Dentro de \textit{jshell} pueden hacer \texttt{/vars} para obtener los datos de las variables 
      que hayan creado, por ejemplo:
      \begin{minted}{java}
        jshell> int a = 1;
        jshell> double b = 2;
        jshell> a + b;
        jshell> /vars
            int a = 1
            double b = 2.0
            double $3 = 3.0
      \end{minted}

      Para las siguientes preguntas utilicen \textit{jshell} para verificar sus respuestas.
      \begin{enumerate}
        \item ¿Cuál es el tipo resultante de sumar un \texttt{int} con un \texttt{Integer}?
        \item ¿Cuál es el resultado de hacer \mintinline{java}{'f' + 'c'}?
          ¿De qué tipo es el resultado?
          ¿Cambia el resultado si hacemos \mintinline{java}{char fc = 'f' + 'c'}?
        \item ¿Qué resultado se obtiene de hacer \mintinline{java}{"f" + 'c'}?
          ¿Cuál es el tipo resultante?
        \item \mintinline{java}{Integer.MAX_VALUE} es una constante igual a \(2^{31} - 1\), que es el 
          número más grande que se puede almacenar en un \mintinline{java}{int}.
          ¿Cuáles son el resultado y tipo de hacer \mintinline{java}{Integer.MAX_VALUE + 1}?
      \end{enumerate}
    \end{exercise}
  %
%