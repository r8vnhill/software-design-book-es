\subsection{Inferencia de tipos}
  Las últimas versiones de Java han añadido bastante \textit{azúcar sintáctica}\footnote{Sintáxis
  que facilita la lectura y/o escritura de un programa.} al lenguaje, en gran parte para hacerse
  cargo de una de las críticas más comunes de \textit{Java} que es lo verboso que es el lenguaje.
  En particular, cuando leemos un fragmento de código hay veces en las que el tipo de las variables
  nos puede ser evidente su tipo, tomemos el siguiente código en \textit{Python} como ejemplo.

  \begin{minted}{python}
    once = 11
    apruebo = "entonces"
  \end{minted}

  Del código anterior podemos ver claramente que \texttt{once} es un \emph{entero} y 
  \texttt{apruebo} es un \emph{string}.
  Pero si usamos un lenguaje de tipado estático entonces necesitamos declarar el tipo de las 
  variables, por lo que el mismo ejemplo en \textit{Java} sería algo como:

  \begin{minted}{java}
    int once = 11;
    String apruebo = "entonces";
  \end{minted}

  La inferencia de tipos le permite resolver el tipo de las variables al compilador cuando estos son
  \enquote{evidentes}, evitando así que el programador tenga que especificarlos manualmente.
  En \textit{Java} esto se puede hacer utilizando la keyword \mintinline{java}{var}, con lo que el
  ejemplo anterior podría reemplazarse por:

  \begin{minted}{java}
    var once = 11;
    var apruebo = "entonces";
  \end{minted}

  Dejar que el compilador infiera los tipos puede hacer mucho más fácil escribir y leer el código,
  pero si se abusa de esta capacidad puede hacer el código muchísimo más difícil de mantener.
  La inferencia de tipos debe utilizarse solamente cuando el tipo de la variable se puede distinguir
  sin problemas cuando se le declara y que no de cabida a ambigüedades.
  En el ejemplo anterior la variable \texttt{once} no debiera declararse como \texttt{var} ya que,
  como mencionamos en la sección anterior, en \textit{Java} existen dos representaciones para los
  números enteros, \mintinline{java}{int} e \texttt{Integer}, por lo que no es evidente cuál es su 
  tipo.
  
  Algunos ejemplos de usos correctos e incorrectos:

  \begin{minted}{java}
    /* INCORRECTO */
    // No queda claro si es un int o un Integer
    var uno = 1;
    // No sabemos a priori qué retorna el método
    var ola = comotai();  

    /* CORRECTO */
    // Esto sólo puede ser un string
    var yes = "good";
    // Se define claramente que es un arreglo de ``int''
    var fib = new int[] {0, 1, 1, 2, 3, 5, 8};
  \end{minted}

  Sin embargo, la inferencia de tipos en \textit{Java} es muy limitada en comparación con otros 
  lenguajes que comparten esta característica.
  Lo único que podremos declarar con tipos implícitos serán variables locales dentro de métodos.
  Esto quedará más claro a medida que vayamos utilizando esta característica a lo largo del libro.