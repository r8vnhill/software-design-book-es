\subsubsection{Recursión y control de flujo}
    Veamos algo un poquito más complicado, supongamos que queremos computar el \(n\mathrm{-ésimo}\)
    número de la \textit{sucesión de Fibonacci}, esta secuencia está definida por:
    \[
      \begin{aligned}
        f_0 &= 0  \\
        f_1 &= 1  \\
        f_n &= f_{n - 1} + f_{n - 2}
      \end{aligned}
    \]

    Esta definición es naturalmente recursiva, y podemos implementarla fácilmente de la siguiente 
    forma:

    \begin{minted}{python}
      # Python
      def fibonacci(n):
        if n < 2:
          return n
        return fibonacci(n - 1) + fibonacci(n - 2)
    \end{minted}

    \begin{minted}{java}
      // Java
      int fibonacci(int n) {
        if (n < 2) {
          return n;
        }
        return fibonacci(n - 1) + fibonacci(n - 2);
      }
    \end{minted}

    Con lo que hemos visto hasta ahora debiera ser fácil darse cuenta que ambos códigos son 
    equivalentes.
    
    Ahora, si bien la recursión puede ser una solución \enquote{elegante},\footnote{Al menos según 
    Olmedo} requiere de más recursos que una solución iterativa ya que, como veremos más adelante,
    los llamados sucesivos a funciones comienzan a llenar la pila de ejecución y si no tenemos 
    cuidado la podemos llenar (nuestro querido \textit{stack overflow}).

    Podemos implementar la función de Fibonacci de manera relativamente simple con el siguiente
    algoritmo:

    \begin{minted}{python}
      # Python
      def fibonacci(n):
        a = 0
        b = 1
        for k in range(0, n + 1):
          c = b + a
          a = b
          b = c
        return a
    \end{minted}

    \begin{minted}{java}
      // Java
      int fibonacci(int n) {
        int a = 0;
        int b = 1;
        for (int k = 0; k <= n; k++) {
          int c = b + a;
          a = b;
          b = c;
        }
        return a;
      }
    \end{minted}

    Aquí se comienza a notar un poco más la diferencia entre ambos lenguajes, en particular la 
    sintaxis del ciclo \mintinline{java}{for}.
    Revisemos esto en detalle, el ciclo se define en base a 3 componentes: 
    \begin{inparaenum}[(1)]
      \item un estado inicial (\texttt{initialize}),
      \item una condición de repetición (\texttt{test}) y
      \item una actualización del estado (\texttt{update}),
    \end{inparaenum}
    con esto podríamos escribir un ciclo genérico como:
    
    \begin{minted}{java}
      for (initialize; test; update) { ... }
    \end{minted}

    En el ejemplo específico que vimos recién definimos el estado inicial de \texttt{k} como 
    \mintinline{java}{int k = 0}. 
    Luego, en cada iteración comprobamos la condición de repetición \mintinline{java}{k <= n}, esto 
    quiere decir que si al final de una iteración se tiene que \(\mathtt{k} \leq \mathtt{n}\), 
    entonces se realizará otra pasada por el ciclo.
    El último componente, \texttt{update}, es una instrucción que se ejecuta \emph{al final} de cada
    ciclo y que generalmente es una asignación (e.g. \texttt{i++}, \texttt{i += 2}, 
    \texttt{i /= 10}).
    Todas las componentes del \java{for} son opcionales, lo que permite hacer un ciclo infinito con
    \java{for (;;)} (una muestra de maldad pura).

    Veamos ahora un algoritmo más eficiente (y complicado) para generar la secuencia:\footnote{Tanto 
    la versión recursiva como la iterativa con \java{for} son algoritmos con complejidad \(O(n)\), 
    mientras que utilizando un algoritmo del tipo \textit{divide-and-conquer} se obtiene una 
    complejidad de \(O(\log(n))\).}

    \begin{minted}{python}
      # Python
      def fibonacci(n):
        if n <= 0:
          return 0
        i = n - 1
        aux_one = 0
        aux_two = 1
        a, b = aux_two, aux_one
        c, d = aux_one, aux_two
        while i > 0:
          if i % 2 == 1:  # si `i` es impar
            aux_one = d * b + c * a
            aux_two = d * (b + a) + c * b
            a, b = aux_one, aux_two
          aux_one = c ** 2 + d ** 2
          aux_two = d * (2 * c + d)
          c, d = aux_one, aux_two
          i /= 2
        return a + b
    \end{minted}

    \begin{minted}{java}
      // Java
      int fibonacci(int n) {
        if (n <= 0) {
          return 0;
        }
        i = n - 1;
        auxOne = 0;
        auxTwo = 1;
        int a, b, c, d;
        a = c = auxOne;
        b = d = auxTwo;
        while (i > 0) {
          if (i % 2 == 1) {
            auxOne = d * b + c * a;
            auxTwo = d * (b + a) + c * b;
            a = auxOne;
            b = auxTwo;
          }
          auxOne = c * c + d * d;
          auxTwo = d * (2 * c + d);
          c = auxOne;
          d = auxTwo;
          i /= 2;
        }
        return a + b;
      }
    \end{minted}
    
    No entraremos en detalles sobre los segmentos de código recién mostrados ya que no es el enfoque
    del libro, pero es una buena idea leer ambos con detenimiento para ver las diferencias entre
    \textit{Python} y \textit{Java}.

    Con esto ya nos familiarizamos con la sintáxis básica de \textit{Java}, en el capítulo siguiente
    tomaremos un desvío y nos despediremos de \textit{jshell}
    \begin{center}
      uwu
    \end{center}
    pero no estén tristes, porque a partir de aquí es donde se pone bueno.

    \begin{exercise}
      Considere la función:
      \begin{minted}{java}
        void countdown(int n) {
          System.out.println(n--);
        }
      \end{minted}
      ¿Qué imprime?

      ¿Qué sucede si cambio la instrucción \java{n--} por \java{--n}?
      ¿Por qué?
    \end{exercise}