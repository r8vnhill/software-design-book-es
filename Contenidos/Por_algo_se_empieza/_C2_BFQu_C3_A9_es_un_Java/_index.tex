\chapter{¿Qué es un Java?}
  \label{ch:java}
  \textit{Java} es uno de los lenguajes de programación más utilizados en el mundo (de ahí la 
  necesidad de enseñarles éste y no otro lenguaje), se caracteriza por ser un lenguaje basado en 
  clases, orientado a objetos, estática y fuertemente tipado, y (casi totalmente) independiente del sistema 
  operativo.
  
  \begin{center}
    \textit{¿Qué?}
  \end{center}

  Tranquilos, vamos a ir de a poco.
  Comencemos por uno de los puntos que hizo que \textit{Java} fuera adoptado tan ampliamente en la
  industria, la independencia del sistema operativo.
  Cuando \textit{Sun Microsystems}\footnote{Actualmente \textit{Java} es propiedad de 
  \textit{Oracle Corporation}} publicó la primera versión de \textit{Java} (en 1996), los 
  lenguajes de programación predominantes eran \textit{C} y \textit{C++} (y en menor medida 
  \textit{Visual Basic} y \textit{Perl}).
  Estos lenguajes tenían en común que interactuaban directamente con la API del sistema operativo,
  lo que implicaba que un programa escrito para un sistema \textit{Windows} no funcionaría de la
  misma manera en un sistema \textit{UNIX}.
  \textit{Java} por su parte planteó una alternativa distinta, delegando la tarea de compilar y 
  ejecutar los programas a una máquina virtual (más adelante veremos en más detalle parte del 
  funcionamiento de la \textit{JVM} para entender sus beneficios y desventajas).
  Esto último hizo que, en vez de cambiar el código del programa para crear una aplicación para 
  uno u otro sistema operativo, lo que cambiaba era la versión de la \textit{JVM} permitiendo así
  que un mismo código funcionara de la misma forma en cualquier plataforma capaz de correr la 
  máquina virtual.\footnote{Actualmente casi todos los sistemas operativos son capaces de usar la 
  \textit{JVM}, en particular el sistema \textit{Android} está implementado casi en su totalidad 
  para usar esta máquina virtual.}
  Pasarían varios años antes de que surgieran otros lenguajes que compartieran esa característica
  (destacando entre ellos \textit{Python 2}, publicado el año 2000).

  No tiene mucho sentido seguir hablando de \textit{Java} si no podemos poner en práctica lo que 
  vayamos aprendiendo, así que es un buen momento para instalarlo.
  
  \subimport{Instalando_el_JDK/}{_index.tex}
  \subimport{Tipos_en_Java/}{_index.tex}
  \subimport{Primeros_pasos/}{_index.tex}
  \subimport{.}{Ejercicios.tex}

\printbibliography[keyword=java]