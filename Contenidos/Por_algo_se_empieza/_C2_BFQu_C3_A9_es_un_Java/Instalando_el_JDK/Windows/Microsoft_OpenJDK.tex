\subsubsection{Microsoft OpenJDK}
  \textit{Microsoft} cuenta con una distribución propia de \textit{OpenJDK}, esta distribución no
  tiene ninguna diferencia respecto a la mostrada anteriormente, sólo que el número de versión
  puede variar un poco.

  La primera opción de instalación (la más fácil) es utilizar el gestor de paquetes nativo de 
  \textit{Windows} (\textit{Windows Package Manager}\index{Windows Package Manager}).
  Este gestor de paquetes viene incluido en las versiones más modernas de \textit{Windows 10} y en
  todas las versiones de \textit{Windows 11}.
  Pueden ver si su computador tiene esta herramienta ejecutando:

  \begin{powershell}
    winget --version
  \end{powershell}

  En caso de que eso funcione correctamente, para instalar el \textit{JDK} basta que ejecuten:

  \begin{powershell}
    winget install Microsoft.OpenJDK.17
  \end{powershell}

  Si lo anterior no funciona, entonces desde la terminal:

  \begin{powershell}
    $url = 'https://aka.ms/download-jdk/' `
      + 'microsoft-jdk-17.0.1.12.1-windows-x64.msi'
    Invoke-WebRequest -Uri $url -OutFile 'microsoft-jdk-17.msi'
    cmd /c 'msiexec /i microsoft-jdk-17.msi ' `
      + 'ADDLOCAL=FeatureMain,FeatureEnvironment,' `
      + 'FeatureJarFileRunWith,FeatureJavaHome ' `
      + 'INSTALLDIR="c:\Program Files\Java\"'
    Remove-Item 'microsoft-jdk-17.msi'
  \end{powershell}