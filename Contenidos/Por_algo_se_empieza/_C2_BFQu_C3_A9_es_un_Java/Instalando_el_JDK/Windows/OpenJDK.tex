\subsubsection{OpenJDK}
  La primera opción es instalar la versión de código abierto del \textit{JDK}\index{OpenJDK}.

  Para esto tenemos dos opciones:
  %%region : OPENJDK + CHOCOLATEY
  \begin{defaultbox}[OpenJDK con Chocolatey (Recomendado)]
    %%region : CHOCOLATEY
    Lo primero que necesitaremos para instalar las herramientas que usaremos será un gestor de 
    paquetes, utilizaremos \textit{Chocolatey}.\autocite{choco}\index{Chocolatey}
  
    Para partir abran una ventana de \textit{Powershell} como administrador.
    Una vez abierta, deben ejecutar las instrucciones:\footnote{
      \url{https://github.com/islaterm/software-design-book-es/blob/master/install/windows/chocolatey.ps1}
    }
    \begin{powershell}
      [Net.ServicePointManager]::SecurityProtocol = `
        [Net.SecurityProtocolType]::Tls12
      Set-ExecutionPolicy -ExecutionPolicy RemoteSigned -Force
      Invoke-WebRequest "https://chocolatey.org/install.ps1" `
        -UseBasicParsing | Invoke-Expression
    \end{powershell}
  
    % Esto otorgará los permisos necesarios y descargará e instalará el gestor de paquetes.
  
    % Para comprobar que el programa se haya instalado correctamente corran el comando:
    % \begin{powershell}
    %   choco -?
    % \end{powershell}
    % %%endregion 
    
    % Con chocolatey esto es simple, solamente deben ejecutar:
    % \begin{powershell}
    %   cinst openjdk -y
    % \end{powershell}
    
    % Luego, para ver que se haya instalado correctamente pueden hacer \texttt{java -version}, aquí es
    % muy importante que la versión que aparezca \textbf{no sea} la versión 1.8 o anteriores, en caso 
    % de que esa sea la versión instalada entonces lo recomendado es desinstalar todas las versiones
    % de \textit{Java} instaladas y repetir la instalación.
  \end{defaultbox}
  %%region : CMDER
  \begin{tcolorbox}[enhanced,breakable, colframe=SpringGreen!75!black, title=(Opcional) Cmder]
    Es sabido que las terminales por defecto de \textit{Windows} dejan bastante que desear, 
    por eso es una buena idea instalar una terminal externa (o más bien un emulador de una).
    Existen varias opciones, pero \textit{Cmder}\index{Cmder} es una de las más completas.
  
    Para instalar la terminal utilizaremos \textit{Chocolatey}.
    En una terminal de \textit{Powershell} con permisos de administrador ejecuten:
  
    \begin{powershell}
      cinst cmder -y # Equivalente a 'choco install cmder -y'
    \end{powershell}
  
    Con esto basta para tener \textit{Cmder} instalado, pero una de las principales ventajas
    de utilizar este emulador de consola es la capacidad de personalizarlo.
    A partir de aquí continuaremos desde \textit{Cmder}.
    
    Lo siguiente será instalar herramientas que ayudarán a entregar de mejor forma la información 
    al momento de usar la consola.
    Para esto, deberán ejecutar los siguientes comandos:
  
    \begin{powershell}
      Install-PackageProvider NuGet -MinimumVersion '2.8.5.201' `
        -Force
      Set-PSRepository -Name PSGallery -InstallationPolicy Trusted
      Install-Module -Name 'oh-my-posh'
      Install-Module -Name 'Get-ChildItemColor' -AllowClobber
    \end{powershell}
  
    Lo último es configurar el perfil de \textit{Powershell}, esto se hace en un archivo que
    es el equivalente a \texttt{.bashrc} de los sistemas \textit{Linux}.
    Para abrir este archivo ejecuten:
  
    \begin{powershell}
      # Primero verificamos si existe un perfil de PS, si no existe 
      # lo creamos
      if (-not $(Test-Path $PROFILE -Type Leaf)) {
        New-Item $PROFILE
      }
      # Aquí pueden usar el editor de texto que quieran
      notepad $PROFILE
    \end{powershell}
  
    Esto abrirá notepad \texttt{D:}\footnote{Ojalá sea la última vez en su vida en la que tengan que 
    hacerlo.} (si nunca han configurado la consola, entonces debiera estar vacío).
    Como último paso, escriban lo siguiente en el archivo de configuración y guarden los 
    cambios.
    
    \begin{powershell}
      # Helper function to set location to the User Profile directory
      function cuserprofile { Set-Location ~ }
      Set-Alias ~ cuserprofile -Option AllScope
    
      Import-Module 'oh-my-posh' -DisableNameChecking
    
      # CHOCOLATEY PROFILE
      $ChocolateyProfile = `
        "$env:ChocolateyInstall\helpers\chocolateyProfile.psm1"
      if (Test-Path($ChocolateyProfile)) {
        Import-Module "$ChocolateyProfile"
      }
    
      Set-PSReadlineOption -BellStyle None
      Set-Theme Honukai
    \end{powershell}
    
    La última línea solamente define el \textit{tema} de la consola, pueden ver una lista de
    \textit{temas} disponibles en el 
    \href{https://github.com/JanDeDobbeleer/oh-my-posh#themes}{repositorio de \textit{Oh-My-Posh}}
  \end{tcolorbox}
  %%endregion CMDER  

  \begin{tcolorbox}[enhanced, breakable, title=OpenJDK sin Chocolatey]
    Si no funciona, o no quieren usar el gestor de paquetes, también se puede instalar el 
    \textit{JDK}\index{OpenJDK} manualmente.
    Para esto pueden abrir \textit{Powershell} como administrador y ejecutar lo siguiente (sujétense):

    \begin{powershell}
      $originalLocation = Get-Location # Para poder regresar
      $url = 'https://download.java.net/java/GA/jdk17.0.1/' `
        + '2a2082e5a09d4267845be086888add4f/12/GPL/' `
        + 'openjdk-17.0.1_windows-x64_bin.zip'
      # Creamos una carpeta para instalar Java
      New-Item -Path '$Env:ProgramFiles' -Name 'Java' -ItemType Directory
      Set-Location '$Env:ProgramFiles\Java'
      # Descargamos y descomprimimos el JDK
      Invoke-WebRequest -Uri $url -OutFile 'openjdk-17.0.1.zip'
      Expand-Archive .\openjdk-17.0.1.zip -DestinationPath .
      Remove-Item 'openjdk-17.0.1.zip'
    \end{powershell}

    Lo siguiente será configurar las variables de entorno necesarias, esto puede hacerse desde la
    \enquote{Configuración avanzada del sistema}, o si continuan en la terminal:

    \begin{powershell}
      Set-Location 'jdk-17.0.1'
      [Environment]::SetEnvironmentVariable("JAVA_HOME", 
                                            "$(Get-Location)")
      [Environment]::SetEnvironmentVariable(
        "Path", 
        [Environment]::GetEnvironmentVariable('Path', 
        [EnvironmentVariableTarget]::Machine) + ";$(Get-Location)\bin", 
        [EnvironmentVariableTarget]::Machine)
      Set-Location $originalLocation
      Update-SessionEnvironment  
    \end{powershell}

    Luego, pueden probar la instalación de la misma manera que en la opción anterior.
  \end{tcolorbox}