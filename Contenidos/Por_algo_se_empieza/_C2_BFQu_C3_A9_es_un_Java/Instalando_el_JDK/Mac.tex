\subsection{MacOS}
  \subsubsection{\textit{Open JDK} (Recomendado)}
    Primero, deben descargar la versión apropiada para su sistema operativo desde la 
    \href{https://adoptopenjdk.net/releases.html?variant=openjdk15&jvmVariant=hotspot}{página del 
    proyecto \textit{AdoptOpenJDK}}, en particular el archivo \texttt{.pkg}.

    Luego, desde una terminal ubicada en la carpeta donde descargaron los binarios:\footnote{No uso 
    Mac, por lo que no tengo como probar que las instrucciones funciones, si tuvieran problemas por 
    favor díganme para solucionarlos y actualizar las instrucciones.}

    \begin{minted}{bash}
      installer -pkg OpenJDK15U-jdk_x64_mac_hotspot_15.0.2_7.pkg \
        -target /
    \end{minted}

    En caso de haberse instalado correctamente debiera haberse creado un directorio en \\
    \texttt{/Library/Java/JavaVirtualMachines/}.
    Por último, ejecuten el comando \texttt{java -version} para comprobar que la instalación 
    funcione.