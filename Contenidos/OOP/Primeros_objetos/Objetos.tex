\section{Objetos}
  Hasta ahora, gran parte de lo que hemos visto ha sido programación imperativa, es decir,
  programación basada en el concepto de \textit{procedimientos} o \textit{funciones}.
  Hemos implementado \textit{algoritmos} que nos permiten resolver problemas como series de
  instrucciones que se ejecutan de acuerdo a una lógica.
  Sin embargo, en la vida real, los problemas que debemos resolver no son tan simples como
  una serie de instrucciones, sino que son más complejos y requieren de una mayor abstracción
  para poder ser resueltos.

  En este capítulo veremos la programación orientada a objetos, un paradigma de programación
  que nos permite abstraer\footnote{
    Abstraer es el proceso de eliminar detalles innecesarios de un objeto o sistema para 
    enfocarnos en sus características esenciales (\cite{ObjectorientedProgramming2023}).
    Este será un concepto clave en todo lo que resta del libro.
  } problemas complejos en objetos que interactúan entre sí.

  \begin{defaultbox}[Programación orientada a objetos]
    La programación orientada a objetos es un paradigma de computación que se organiza en base a 
    objetos en vez de acciones y datos en vez de lógica.
    Aquí lo que realmente nos importa son los objetos que queremos manipular más que la lógica para
    manipularlos.
  \end{defaultbox}
  
  Un objeto es una \textit{abstracción} que contiene información y maneras de manejar esta información
  La información contenida dentro de un objeto no es visible desde afuera (a esto se le conoce como transparencia)
  Un objeto se compone de su estado (campos) y su comportamiento (métodos).

  En \textit{Kotlin} podemos crear objetos con la palabra reservada \textit{object} seguida del 
  nombre del objeto.
  Para crear un objeto desde \textit{IntelliJ} primero crearemos un nuevo proyecto de la misma forma
  que lo hicimos en el capítulo anterior, nombrémoslo \enquote{oop-introduction}.
  Con el proyecto creado crearemos un nuevo paquete llamado \url{cl.ravenhill.oop.essential}, 
  hacemos click derecho sobre el paquete y seleccionamos la opción \textit{New} \(\rightarrow\) 
  \textit{Kotlin Class/File} \(\rightarrow\) \textit{Object} y lo llamaremos \textit{Dice}.
  Esto nos creará un archivo llamado \textit{Dice.kt} dentro del paquete que acabamos de crear con
  el siguiente contenido:

  \begin{kotlin}
    package cl.ravenhill.oop.essential

    object Dice {
    }
  \end{kotlin}

  Como podemos ver, el objeto \textit{Dice} no tiene ningún campo ni método, por lo que no es muy
  útil.
  Para agregar campos y métodos a un objeto debemos agregarlos dentro de las llaves del objeto.

  Agreguemos un campo llamado \textit{sides} que será un número entero que representará la cantidad
  de caras del dado.
  Para agregar un campo a un objeto debemos agregar la palabra reservada \textit{var} seguida del
  nombre del campo, podemos omitir el tipo de dato ya que \textit{Kotlin} lo infiere 
  automáticamente y es obvio por contexto.
  Digamos que nuestro dado tiene 6 caras, por lo que nuestro campo \textit{sides} tendrá el valor
  por defecto de 6.

  \begin{kotlin}
    object Dice {
      var sides = 6
    }
  \end{kotlin}

  Ahora agreguemos un método llamado \textit{roll} que nos permita tirar el dado.

  \begin{kotlin}
    /**
     * Object that represents an n-sided dice.
     * 
     * @property sides Number of sides of the dice.
     */
    object Dice {
      var sides = 6

      /**
       * Rolls the dice.
       * 
       * @return A random number between 1 and the number of sides of the dice.
       */
      fun roll(): Int {
        return (1..sides).random()
      }
    }
  \end{kotlin}

  Como podemos ver, el método \textit{roll} no recibe ningún parámetro y retorna un número entero.
  Para generar un número aleatorio entre 1 y 6 podemos usar la función \textit{random} de la clase
  \textit{IntRange} que nos permite generar un número aleatorio entre un rango de números.

  Noten las anotaciones \texttt{@property} y \texttt{@return} que agregamos al objeto y al método.
  Estas anotaciones nos permiten agregar documentación más detallada a nuestro código.
  Estas anotaciones son opcionales, pero es una buena práctica agregarlas para hablar un lenguaje
  común con otros desarrolladores.
  
  Luego, podemos utilizar el objeto que creamos de la siguiente forma:

  \begin{kotlin}
    fun main() {
      println("Dice roll: ${Dice.roll()}")
      Dice.sides = 20
      println("Dice roll: ${Dice.roll()}")
    }
  \end{kotlin}

  \begin{important}
    Los objetos son estáticos, es decir, no podemos crear instancias de ellos, por lo que no podemos
    tener múltiples objetos del mismo tipo.
    Los objetos son útiles cuando queremos crear una única instancia de un objeto que se comparte
    por toda la aplicación.
  \end{important}

  Por último, hacemos:

  \begin{powershell}
    git add .
    git commit -m "FEAT Adds Dice object"
  \end{powershell}