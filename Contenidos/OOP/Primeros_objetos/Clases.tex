\section{Clases}
  Una clase es un modelo que define la estructura de un objeto.
  Podemos pensarla como una plantilla que podemos rellenar con datos para crear objetos.
  Una clase es una abstracción de un objeto, es decir, es una representación de un \textbf{tipo} de
  objeto.
  Por ejemplo, podemos definir una clase llamada \textit{Player} que represente a un jugador de 
  nuestro juego.

  Para definir una clase debemos usar la palabra reservada \textit{class} seguida del nombre de la
  clase.
  En \textit{IntelliJ} debemos hacer click derecho sobre el paquete \url{cl.ravenhill.kygo}
  y seleccionar la opción \textit{New} \texttt{->} \textit{Kotlin Class/File} y seleccionar la 
  opción \textit{Class}.
  Esto creará el siguiente archivo:

  \begin{kotlin}
    package cl.ravenhill.kygo

    class Player {
    }
  \end{kotlin}

  Ahora podemos agregar campos a la clase:

  \begin{kotlin}
    class Player {
      var name: String = ""
      var health: Int = 0
    }
  \end{kotlin}

  Como podemos ver, los campos de una clase se definen de la misma forma que los campos de un 
  objeto.

  Ahora, podemos crear objetos a partir de la clase \textit{Player}:

  \begin{kotlin}
    fun main() {
      val player1 = Player()
      player1.name = "Mugi"
      player1.health = 8000
      println(player1.name)
      println(player1.health)
      
      val player2 = Player()
      player2.name = "Jaiva"
      player2.health = 7000
      println(player2.name)
      println(player2.health)
    }
  \end{kotlin}

  \subsection{Constructores primarios}
    Un constructor es un bloque de código (\textbf{no} una función) que se ejecuta al momento de 
    crear un objeto a partir de una clase.
    En \textit{Kotlin} podemos definir dos tipos de constructores: \textbf{primarios} y
    \textbf{secundarios}.

    Por ahora veamos cómo definir un constructor primario, ya que podemos introducir los 
    constructores secundarios más adelante cuando los necesitemos.
    Un constructor primario se define dentro de un bloque \textit{init} (de nuevo, \textbf{no} es 
    una función).
    Un constructor primario puede recibir parámetros, los cuales se definen en la declaración de la
    clase, con una sintáxis similar a la de los métodos.
    Esto se ve mejor con un ejemplo:
    
    \begin{kotlin}
      class Player(name: String, health: Int) {
        val name: String
        val health: Int

        init {
          this.name = name
          this.health = health
        }
      }
    \end{kotlin}
    
    Vamos viendo lo que está pasando:

    \begin{itemize}
      \item Primero definimos los campos de la clase, los cuales son \textit{name} y 
        \textit{health}.
      \item Luego definimos los campos de la clase.
      \item Dentro del bloque \textit{init} asignamos los valores de los parámetros a los campos de
        la clase.
    \end{itemize}

    Pero hay dos cosas que probablemente les hagan ruido:

    \begin{itemize}
      \item ¿Qué son \texttt{this.name} y \texttt{this.health}?
      \item ¿Estamos asignando nuevos valores a variables de sólo lectura? 
    \end{itemize}

    La segunda pregunta es fácil de responder: no, no estamos asignando nuevos valores a variables 
    de sólo lectura.
    Como el constructor primario se ejecuta al momento de crear un objeto, los campos de la clase
    aún no se han inicializado, por lo que podemos asignarles un valor.

    La primera pregunta es un poco más compleja de responder, porque necesitamos entender el 
    concepto de \textit{method-lookup} que veremos más adelante.
    Por ahora, basta con decir que \texttt{this.name} y \texttt{this.age} hacen referencia a los
    campos de la clase, y no a los parámetros del constructor.

    \begin{note}
      Existen otras (mejores) formas de definir un constructor primario, pero por ahora nos 
      limitaremos a ésta para no introducir tantos conceptos nuevos.
    \end{note}

    Ahora podemos repetir el ejemplo anterior, pero usando el constructor primario que acabamos de
    definir:

    \begin{kotlin}
      fun main() {
        val player1 = Player("Mugi", 8000)
        println(player1.name)
        println(player1.health)

        val player2 = Player("Jaiva", 7000)
        println(player2.name)
        println(player2.health)
      }
    \end{kotlin}

    Por último:

    \begin{powershell}
      git add .
      git commit -m "FEAT Creates Player class"
    \end{powershell}