\subsection{Instalando Git}
  \subsubsection{Windows}
    Para instalar \textit{Git} en Windows, podemos hacerlo utilizando \textit{winget}, para esto
    abran PS y ejecuten el siguiente comando:

    \begin{powershell}
      winget install Git.Git
    \end{powershell}

    Por último comprobemos que la instalación fue exitosa ejecutando el siguiente comando:

    \begin{powershell}
      git version
    \end{powershell}

  \subsubsection{Linux (Debian)}
    Para instalar \textit{Git} en Linux, podemos hacerlo utilizando el gestor de paquetes de su
    distribución, para esto abran una terminal y ejecuten lo siguiente:

    \begin{bash}
      sudo apt-get update
      sudo apt-get install git-all
    \end{bash}

    Por último comprobemos que la instalación fue exitosa ejecutando el siguiente comando:

    \begin{bash}
      git version
    \end{bash}

  \subsubsection{MacOS}
    Para instalar \textit{Git} en MacOS, podemos hacerlo utilizando \textit{Homebrew}, para esto
    abran una terminal y ejecuten el siguiente comando:

    \begin{bash}
      brew install git
    \end{bash}

    Por último comprobemos que la instalación fue exitosa ejecutando el siguiente comando:

    \begin{bash}
      git version
    \end{bash}
