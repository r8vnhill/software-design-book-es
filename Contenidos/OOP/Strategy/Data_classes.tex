\section{\textit{Data classes}}
  \label{sec:oop:strategy:data_classes}
  
  Actualizar la interfaz
  \begin{kotlin}
    interface Card {
      val name: String
      val text: String
      val effect: Effect
      val serializer: CardSerializer
    }
  \end{kotlin}

  Agregamos efectos a las cartas
  \begin{kotlin}
    abstract class AbstractCard(
      override val name: String,
      override val text: String,
      override val effect: Effect,
      override val serializer: CardSerializer
    ) : Card {

      fun toFile(filename: String) = serializer.toFile(this, filename)

      fun serialize() = serializer.serialize(this)

      fun useOn(player: Player) = effect.useOn(player)
    }
  \end{kotlin}

  Noten que ahora que tenemos los efectos definidos como clases separadas, la implementación de las
  cartas mágicas que definimos dejan de tener sentido.
  Por lo tanto, las vamos a eliminar.
  Pero si borramos esas clases también pierde sentido que las cartas mágicas sean clases abstractas,
  así que podríamos cambiarla por una clase concreta.
  Con esto, la clase \texttt{MagicCard} quedaría así:

  \begin{kotlin}
    class MagicCard(
      name: String,
      text: String,
      effect: Effect,
      serializer: CardSerializer
    ) : AbstractCard(name, text, effect, serializer)
  \end{kotlin}

  También debemos actualizar las cartas de monstruos y trampas.

  \begin{kotlin}
    class MonsterCard(
      name: String,
      text: String,
      val attack: Int,
      effect: Effect,
      serializer: AbstractCardSerializer
    ) : AbstractCard(name, text, effect, serializer) {
      fun attack(player: Player) {
        player.takeDamage(this.attack)
      }
    }
  \end{kotlin}

  \begin{kotlin}
    class TrapCard(
      name: String,
      text: String,
      effect: Effect,
      serializer: CardSerializer
    ) : AbstractCard(name, text, effect, serializer)
  \end{kotlin}

  Ahora actualicemos hagamos un nuevo test para probar que las cartas funcionan correctamente.
  Para eso creemos una clase \texttt{MagicCardTest} y modelemos cada una de las cartas mágicas
  que habíamos definido como un test.

  \begin{kotlin}
    class MagicCardTest : FunSpec({
      lateinit var player: Player

      beforeTest {
        player = Player(
          "Jaiva", 8000, mutableListOf(
            MagicCard(
              "Saucepan of Avidity",
              "Draw 2 cards",
              DrawCardsEffect(2),
              XmlCardSerializer()
            ),
            MagicCard(
              "Crimson Beverage",
              "Heal 500 HP",
              ModifyHealthEffect(500),
              XmlCardSerializer()
            )
          )
        )
      }

      test("Crimson Beverage should heal 500 HP") {
        val crimsonBeverage = MagicCard(
          "Crimson Beverage",
          "Heal 500 HP",
          ModifyHealthEffect(500),
          XmlCardSerializer()
        )
        player.health shouldBe 8000
        crimsonBeverage.useOn(player)
        player.health shouldBe 8500
      }

      test("Indigo Potion should heal 400 HP") {
        val indigoPotion = MagicCard(
          "Indigo Potion",
          "Heal 400 HP",
          ModifyHealthEffect(400),
          XmlCardSerializer()
        )
        player.health shouldBe 8000
        indigoPotion.useOn(player)
        player.health shouldBe 8400
      }

      test("Saucepan of Avidity should draw 2 cards") {
        val saucepanOfAvidity = MagicCard(
          "Saucepan of Avidity",
          "Draw 2 cards",
          DrawCardsEffect(2),
          XmlCardSerializer()
        )
        player.hand shouldBe emptyList()
        saucepanOfAvidity.useOn(player)
        player.hand.size shouldBe 2
        player.deck.size shouldBe 0
      }
    })
  \end{kotlin}

  Si corremos los tests, veremos que todos pasan.
  Veamos que podemos mejorar del diseño.

  Notemos que ahora nuestras cartas se encargan principalmente de guardar información y de
  delegar sus funcionalidades a otras clases.
  Para estos casos, \textit{Kotlin} nos provee de una mejor forma de modelar estas clases.

  \begin{defaultbox}[\textit{Data Class}]
    Una \textit{data class} es una clase que se encarga principalmente de guardar información.
    Estas clases tienen varios beneficios:
    \begin{itemize}
      \item \textit{Kotlin} provee de una implementación por defecto de los métodos 
        \texttt{equals()}, \texttt{hashCode()} y \texttt{toString()}.
      \item \textit{Kotlin} provee de una implementación por defecto de los métodos
        \texttt{copy()} y \texttt{componentN()}.
    \end{itemize}
  \end{defaultbox}

  \begin{important}
    Una \textit{data class} no puede ser hereada, pero una \textit{data class} si puede heredar de
    otra clase.
  \end{important}

  Para convertir una clase en una \textit{data class} debemos agregar la palabra clave
  \texttt{data} antes de la palabra clave \texttt{class}.
  Por ejemplo, la clase \texttt{MagicCard} podría quedar así:

  \begin{kotlin}
    data class MagicCard(
      override val name: String,
      override val text: String,
      override val effect: Effect,
      override val serializer: CardSerializer
    ) : Card
  \end{kotlin}

  Notarán que agregamos \mintinline{kotlin}|override val| antes de los atributos de la clase.
  Esto es porque ahora que la clase es una \textit{data class}, todos los parámetros que recibe
  el constructor deben ser propiedades de la clase.

  Si corremos los tests, veremos que todos siguen pasando.

  Con esto, podemos hacer \textit{commit} de los cambios y pasar a lo último que nos falta.

  \begin{powershell}
    git add .
    git commit -m "REFACTOR Magic card is now a data class"
  \end{powershell}

  \begin{exercise}
    Transforme las clases \texttt{MonsterCard} y \texttt{TrapCard} en \textit{data classes}.
  \end{exercise}