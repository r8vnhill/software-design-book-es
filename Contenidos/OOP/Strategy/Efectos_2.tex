\section{Efectos: segundo intento}
  Veamos como podemos utilizar el patrón \textit{Strategy} para implementar los efectos de las 
  cartas.

  Primero, creemos un paquete \texttt{cl.ravenhill.kygo.effects} dentro de la carpeta \texttt{src}
  y dentro de este paquete creemos una interfaz \texttt{Effect} con el método 
  \mintinline{kotlin}|Effect::useOn(Player)|.

  \begin{kotlin}
    interface Effect {
      fun useOn(player: Player)
    }
  \end{kotlin}

  Ahora, podemos crear un test para probar que un efecto que modifique la vida de un jugador
  funcione correctamente.

  \begin{kotlin}
    class ModifyHealthEffectTest : FunSpec({
      lateinit var modifyHealthEffect: ModifyHealthEffect
      
      beforeTest {
        modifyHealthEffect = ModifyHealthEffect(500)
      }
      
      test("ModifyHealthEffect should modify health by the given amount") {
        val player = Player("Jaiva", 8000, mutableListOf())
        player.health shouldBe 8000
        modifyHealthEffect.useOn(player)
        player.health shouldBe 8500
      }
    })
  \end{kotlin}

  Comprobemos que el test falla.

  Ahora podemos implementar la clase \texttt{ModifyHealthEffect} de la siguiente manera:

  \begin{kotlin}
    class ModifyHealthEffect(private val amount: Int) : Effect {
      override fun useOn(player: Player) {
        player.health += amount
      }
    }
  \end{kotlin}

  Si corremos el test ahora, veremos que pasa.
  Nos queda implementar el efecto de robar cartas.

  Partimos de un test que nos asegure que el efecto de robar cartas funciona correctamente.

  \begin{kotlin}
    class DrawCardsEffectTest : FunSpec({
      lateinit var drawCardsEffect: DrawCardsEffect

      beforeTest {
        drawCardsEffect = DrawCardsEffect(2)
      }

      test("DrawCardsEffect should draw the given amount of cards") {
        val player = Player(
          "Jaiva", 8000, mutableListOf(
            MagicCard(
              "Saucepan of Avidity",
              "Draw 2 cards",
              DrawCardsEffect(2),
              XmlCardSerializer()
            ),
            MagicCard(
              "Crimson Beverage",
              "Heal 500 HP",
              ModifyHealthEffect(500),
              XmlCardSerializer()
            )
          )
        )
        player.hand.size shouldBe 0
        drawCardsEffect.useOn(player)
        player.hand.size shouldBe 2
      }
    })
  \end{kotlin}
    
  Ahora corremos el test para ver que falla y pasamos a implementar el efecto.

  \begin{kotlin}
    class DrawCardsEffect(private val amount: Int) : Effect {
      override fun useOn(player: Player) = player.draw(amount)
    }
  \end{kotlin}

  Si corremos los tests ahora veremos que todo funciona como esperábamos.
  Aprovechemos de hacer \textit{commit} de nuestro trabajo.

  \begin{powershell}
    git add .
    git commit -m "FEAT Adds effects"
  \end{powershell}
