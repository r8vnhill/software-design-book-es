\section{Efectos: Primer intento}
  \label{sec:oop:strategy:effect:1}

  Para estos ejemplos consideremos que queremos implementar tres cartas mágicas:

  \begin{itemize}
    \item \textit{Crimson Beverage}: Aumenta en 500 la vida de un jugador.
    \item \textit{Indigo Potion}: Aumenta en 400 la vida de un jugador.
    \item \textit{Saucepan of Avidity}: Roba 2 cartas del mazo del jugador.
  \end{itemize}

  Comencemos por crear tests para estas cartas.
  En el paquete \url{cl.ravenhill.kygo.cards} comencemos creando las clases 
  \texttt{CrimsonBeverageTest} y \texttt{IndigoPotionTest}:

  \begin{kotlin}
    class CrimsonBeverageTest : FunSpec({
      val initHealth = 8000
      lateinit var player: Player
      lateinit var crimsonBeverage: CrimsonBeverage

      beforeTest {
        player = Player("Jaiva", initHealth, mutableListOf())
        crimsonBeverage = CrimsonBeverage()
      }

      test("Crimson Beverage should increase 500 HP") {
        player.health shouldBe initHealth
        crimsonBeverage.useOn(player)
        player.health shouldBe initHealth + 500
      }
    })
  \end{kotlin}

  \begin{kotlin}
    class IndigoPotionTest : FunSpec({
      val initHealth = 8000
      lateinit var player: Player
      lateinit var indigoPotion: IndigoPotion

      beforeTest {
        player = Player("Jaiva", initHealth, mutableListOf())
        indigoPotion = IndigoPotion()
      }

      test("Indigo Potion should increase 400 HP") {
        player.health shouldBe initHealth
        indigoPotion.useOn(player)
        player.health shouldBe initHealth + 400
      }
    })
  \end{kotlin}

  Con los tests podemos empezar a pensar en la implementación concreta.
  Podríamos decir que cada tipo de carta mágica es una especialización de una carta mágica así que
  tiene sentido querer implementar las cartas que queremos como subclases de \texttt{MagicCard}.
  Primero, transformemos la clase \texttt{MagicCard} en una clase abstracta y renombrémosla
  \texttt{AbstractMagicCard} y agreguemos un método abstracto \texttt{useOn(Player)}.

  \begin{kotlin}
    abstract class AbstractMagicCard(
      name: String, text: String, serializer: CardSerializer) :
        AbstractCard(name, text, serializer) {
      abstract fun useOn(player: Player)
    }
  \end{kotlin}

  Ahora, podemos implementar las cartas mágicas como subclases de \texttt{AbstractMagicCard}.

  \begin{kotlin}
    class CrimsonBeverage :
        AbstractMagicCard("Crimson Beverage", "Increase 500 health", XmlCardSerializer()) {
      override fun useOn(player: Player) {
        player.health += 500
      }
    }
  \end{kotlin}

  \begin{kotlin}
    class IndigoPotion :
        AbstractMagicCard("Indigo Potion", "Increase 400 health", XmlCardSerializer()) {
      override fun useOn(player: Player) {
        player.health += 400
      }
    }
  \end{kotlin}

  Ahora, adaptamos la clase \texttt{Player} para poder modificar su salud utilizando un 
  \textit{setter} y mejoremos un poco su diseño:

  \begin{kotlin}
    class Player(
      val name: String,
      health: Int,
      val deck: MutableList<Card>
    ) {
      var health: Int =
        if (health > 0) health else throw InvalidStatException("Health", health)
        set(value) {
          field = max(value, 0)
        }

      fun takeDamage(damage: Int) {
        this.health -= damage
      }
    }
  \end{kotlin}

  Donde \texttt{InvalidStatException} está definida de la misma forma que en el capítulo anterior.

  Ahora podemos ejecutar los tests y ver que todo funciona como esperábamos.

  Antes de continuar con la implementación de la carta \texttt{Saucepan of Avidity} vamos a
  refactorizar un poco el código.

  Primero notarán que las dos cartas mágicas que hemos implementado tienen un comportamiento
  muy similar: aumentan la vida del jugador en un valor fijo.
  Podríamos decir que la única diferencia entre ambas es el valor de la vida que se aumenta.
  Tiene sentido entonces agrupar el comportamiento común en una clase abstracta y dejar que
  las subclases implementen el valor de la vida que se aumenta.

  \begin{kotlin}
    abstract class AbstractHealthMagicCard(
      name: String, text: String, serializer: CardSerializer, private val health: Int
    ) : AbstractMagicCard(name, text, serializer) {
      override fun useOn(player: Player) {
        player.health += health
      }
    }
  \end{kotlin}

  Y reimplementamos las cartas mágicas como subclases de \texttt{AbstractHealthMagicCard}.

  \begin{kotlin}
    class CrimsonBeverage : AbstractHealthMagicCard(
      "Crimson Beverage",
      "Increase 500 health",
      XmlCardSerializer(),
      500
    )
  \end{kotlin}

  \begin{kotlin}
    class IndigoPotion : AbstractHealthMagicCard(
      "Indigo Potion",
      "Increase 400 health",
      XmlCardSerializer(),
      400
    )
  \end{kotlin}

  Ahora podemos ejecutar los tests y ver que todo sigue funcionando como esperábamos.

  Nos queda implementar la carta \texttt{Saucepan of Avidity}, como costumbre, primero
  escribimos el test.

  \begin{kotlin}
    class SaucepanOfAvidityTest : FunSpec({
      val initHealth = 8000
      lateinit var player: Player
      lateinit var saucepanOfAvidity: SaucepanOfAvidity

      beforeTest {
        player = Player(
          "Jaiva",
          initHealth,
          mutableListOf(CrimsonBeverage(), IndigoPotion())
        )
        saucepanOfAvidity = SaucepanOfAvidity()
      }

      test("Saucepan of Avidity should draw 2 cards") {
        player.hand shouldBe emptyList()
        saucepanOfAvidity.useOn(player)
        player.hand.size shouldBe 2
        player.deck.size shouldBe 0
      }
    })
  \end{kotlin}

  Corramos el test para ver que todo falla como esperamos y ahora implementemos la carta.

  \begin{kotlin}
    class SaucepanOfAvidity : AbstractMagicCard(
      "Saucepan of Avidity",
      "Draw 2 cards",
      XmlCardSerializer()
    ) {
      override fun useOn(player: Player) {
        player.draw(2)
      }
    }
  \end{kotlin}
  
  Y agregamos lo necesario a la clase \texttt{Player} para que el test pase.

  \begin{kotlin}
    class Player(...) {
      ...
      private val _hand: MutableList<Card> = mutableListOf()
      val hand: List<Card>
        get() = _hand.toList()
    
      fun draw(n: Int = 1) {
        for (i in 1..n) {
          if (deck.isEmpty()) {
            break
          }
          _hand.add(deck.removeAt(0))
        }
      }
    }
  \end{kotlin}

  Con esto debiera bastar para que los tests pasen.
  Hagamos \textit{commit}, y pasemos a revisar nuestro diseño.

  \begin{powershell}
    git add .
    git commit -m "FEAT Adds Magic Cards"
  \end{powershell}
  