\chapter{En efecto}
\label{sec:strategy}
  
  En este capítulo seguiremos implementando \textit{Kygo}, un proyecto que tenemos triste y
  abandonado desde hace varios capítulos.
  Hasta ahora tenemos tres tipos de cartas: \textit{Monstruos}, \textit{Mágicas} y \textit{Trampas}.
  Pero tanto las cartas mágicas como las trampas no hacen nada.
  En este capítulo vamos a implementar efectos de modo que las cartas mágicas y trampas puedan
  hacer algo.

  Para empezar, necesitamos agregar las dependencias de \textit{Kotest} a nuestro proyecto.
  Para hacer esto necesitaremos transformar nuestro proyecto en un proyecto \textit{Gradle}.
  Para hacer esto, iremos a \textit{File} $\rightarrow$ \textit{New} $\rightarrow$ 
  \textit{Project...}.
  Aquí, haremos un proyecto de tipo \textit{Gradle} de la misma forma que antes, lo llamaremos
  \texttt{kygo} y lo ubicaremos en la misma carpeta que el proyecto del \cref{chap:oop2} (es 
  importante que el proyecto se llame de la misma forma que el de ese capítulo ya que estamos 
  intentando \enquote{sobrescribir} el proyecto).

  Una vez que hayamos creado el proyecto, podemos configurarlo de la misma forma que hicimos en el
  \cref{subsec:gradle}.

  Con esto, de la misma forma que en ese capítulo, podemos hacer \textit{commit} de lo que hicimos:

  \begin{powershell}
    git add .
    git commit -m "PROJECT Migration to Gradle"
  \end{powershell}

  \subimport{.}{Efectos_1.tex}
  \subimport{.}{Strategy.tex}
  \subimport{.}{Efectos_2.tex}
  \subimport{.}{Data_classes.tex}
  \subimport{.}{Coverage.tex}
  
  \textbf{TODO}
  \begin{itemize}
    \item DDT
  \end{itemize}

  \printbibliography[keyword=strategy]