\section{Getters}
  Comencemos a implementar a los jugadores del juego.
  Llamaremos a estos jugadores \textit{Bakemon trainers} (entrenadores de Bakemon).

  Un entrenador tiene un nombre, un equipo de Bakemon, y un inventario de ítems.

  Comencemos por crear una clase \texttt{TrainerTest} en el paquete \url{cl.ravenhill.bakemon}:

  \begin{kotlin}
    class TrainerTest : FunSpec({
      val team = mutableListOf<Bakemon>(
        FireBakemon("Karmander", 130, 5, 20),
        WaterBakemon("Kokodile", 130, 5, 20)
      )
      val items = mutableListOf(Potion(), RaiseAttack())
      lateinit var trainer: Trainer
      
      beforeTest {
        trainer = Trainer("Ack", team, items)
      }
      
      test("A Trainer can be created with a name, a team and items") {
        trainer.name shouldBe "Ack"
        trainer.team shouldBe team
        trainer.items shouldBe items
      }
    })
  \end{kotlin}

  Como siempre, podemos correr los tests para ver que fallan.

  Ahora veamos como implementar a los entrenadores.
  Para esto crearemos una clase \texttt{Trainer} como sigue:

  \begin{kotlin}
    class Trainer(
      val name: String,
      val team: MutableList<Bakemon>,
      val items: MutableList<Item>
    )
  \end{kotlin}

  Ahora corramos los tests para ver que pasan.

  Pero

  \begin{center}
    ¿Tenemos un problema?
  \end{center}

  Sip.

  Tanto \texttt{team} como \texttt{items} son \mintinline{kotlin}|val|ores, pero como ya vimos, eso
  no significa que no podamos modificarlos.
  Ya hemos dicho muchas veces que nos interesa \textit{encapsular el cambio}, y algo escencial para
  lograrlo es que un objeto maneje su propio estado.
  Pero si podemos modificar \texttt{team} y \texttt{items} desde afuera, entonces el objeto no
  tiene control de lo que pasa con su estado.
  Pero nos gustaría poder agregar y quitar Bakemon e ítems de los entrenadores, por lo que 
  necesitamos que las propiedades sean mutables\dots

  ¿Qué podemos hacer?

  \begin{defaultbox}[Getter]
    Un \textit{getter} es un método que se ejecuta cada vez que se intenta acceder a una propiedad.
  \end{defaultbox}

  En \textit{Kotlin}, podemos definir un \textit{getter} utilizando la palabra reservada 
  \mintinline{kotlin}|get|.
  La sintaxis es muy similar a la de un \textit{setter} con la diferencia de que un \textit{getter}
  retorna un valor y no tiene parámetros.

  Para nuestro ejemplo, usaremos una técnica llamada \textit{backing field} (campo de respaldo), que
  consiste en crear una propiedad privada que almacena el valor de la propiedad pública.
  Luego, el \textit{getter} de la propiedad pública retorna una copia del valor de la propiedad
  privada.
  De esta forma, el objeto tiene control sobre el estado de la propiedad, y no podemos modificarlo
  desde fuera sin que el objeto se entere.
  Veamos cómo hacemos esto:

  \begin{kotlin}
    class Trainer(
      val name: String,
      team: MutableList<Bakemon>,
      items: MutableList<Item>
    ) {
      private val _team = team
      private val _items = items
      val team: List<Bakemon>
        get() {
          return _team.toList()
        }
      val items: List<Item>
        get() {
          return _items.toList()
        }
    }
  \end{kotlin}

  Veamos lo que estamos haciendo.
  Primero, creamos dos propiedades privadas llamadas \texttt{\_team} y \texttt{\_items} que almacenan
  los valores de las propiedades públicas \texttt{team} y \texttt{items}.
  Luego, creamos dos propiedades públicas llamadas \texttt{team} y \texttt{items} que tienen un
  \textit{getter} que retorna una copia inmutable (con \mintinline{kotlin}|toList()|) de los
  valores de las propiedades privadas.

  Corramos los tests para ver que no hayamos roto nada y luego hagamos un commit.

  \begin{powershell}
    git add .
    git commit -m "FEAT Adds Trainer class"
  \end{powershell}