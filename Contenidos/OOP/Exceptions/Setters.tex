\section{Setters}
  \label{sec:setters}

  Al igual que cuando restringimos la vida mínima de un \textit{Bakémon} a 0, también deberíamos
  restringir una vida máxima.
  Además, tendría sentido que el ataque no pueda ser negativo.
  Pero si se dan cuenta, estaremos haciendo todos esos chequeos cada vez que haya un método que
  quiera modificar la vida o el ataque de un \textit{Bakémon}.

  \begin{center}
    ¡No encapsulamos el cambio!1
  \end{center}

  En efecto, lo bueno es que \textit{Kotlin} nos ofrece una forma sencilla de solucionar esto.

  \begin{defaultbox}[Setter]
    Un \textit{setter} es un método que se ejecuta cada vez que se asigna un valor a una propiedad.
  \end{defaultbox}

  En esta parte voy a escribir los tests a posteriori, ya que primero tenemos que entender qué es un
  \textit{setter} para entender cómo usarlo y probarlo.

  En \textit{Kotlin} podemos definir un \textit{setter} para una propiedad utilizando la palabra
  reservada \mintinline{kotlin}|set|.
  Puede que esto les suene familiar, ya que lo usamos antes para definir una variable que fuera
  accesible desde fuera de la clase pero que solo fuera modificable desde dentro de la clase; cuando
  usamos \mintinline{kotlin}|private set|.
  
  Por ejemplo, podemos definir un \textit{setter} para el ataque de un \textit{Bakémon} de la
  siguiente forma:

  \begin{kotlin}
    abstract class AbstractBakemon(
      override val name: String,
      override var health: Int,
      override val level: Int,
      attack: Int
    ) : Bakemon {
      override var attack: Int = attack
        set(value) {
          field = if (value < 0) 0 else value
        }
        ...
    }
  \end{kotlin}

  Veamos lo que hicimos aquí.
  Primero, cambiamos \mintinline{kotlin}|attack| en los parámetros del constructor a un parámetro
  \enquote{normal} (le quitamos el \mintinline{kotlin}|override var|) y creamos una propiedad
  \mintinline{kotlin}|attack| en el cuerpo de la clase.
  Luego definimos un setter con \mintinline{kotlin}|set(value)|, donde \mintinline{kotlin}|value|
  es el valor que se le asigna a la propiedad.\footnote{Noten que podemos pasarle un valor inicial.}
  Finalmente, dentro del setter, asignamos el valor a la propiedad \mintinline{kotlin}|field|, que
  es una variable que representa la propiedad (algo así como un \mintinline{kotlin}|this| para las
  propiedades).
  En este caso, si el valor es negativo, asignamos 0, de lo contrario asignamos el valor que se
  le asignó a la propiedad.

  De la misma forma, podemos definir un setter para la vida de un \textit{Bakémon}, pero esta vez
  necesitamos que exista un límite superior para la vida.
  Para esto, primero definamos una propiedad que represente el límite superior de la vida en la 
  interfaz \mintinline{kotlin}|Bakemon|:

  \begin{kotlin}
    interface Bakemon {
      val maxHealth: Int
      ...
    }
  \end{kotlin}

  Y luego la sobrescribimos en la clase \mintinline{kotlin}|AbstractBakemon|:

  \begin{kotlin}
    abstract class AbstractBakemon(
      override val name: String,
      final override val maxHealth: Int,
      override val level: Int,
      attack: Int
    ) : Bakemon {
      override var health: Int = maxHealth
        set(value) {
          field = when {
            value > maxHealth -> maxHealth
            value < 0 -> 0
            else -> value
          }
        }
      ...
    }
  \end{kotlin}

  Ahora veamos lo que hicimos.
  Primero agregamos un parámetro \mintinline{kotlin}|maxHealth| al constructor de la clase 
  abstracta, y lo definimos como \mintinline{kotlin}|final| para que no pueda ser sobrescrito.
  Luego, definimos la propiedad \mintinline{kotlin}|health| y la inicializamos con el valor de
  \mintinline{kotlin}|maxHealth|.
  Por último, definimos el setter de modo que si el valor es mayor que el límite superior de la
  vida, se asigna el límite superior de la vida, si es menor que 0, se asigna 0, y de lo contrario
  se asigna el valor que se le asignó a la propiedad.
  
  \begin{center}
    Espera, ¿por qué queremos que la vida máxima no pueda ser sobrescrita?
  \end{center}

  Porque luego iniciamos la propiedad \mintinline{kotlin}|health| con el valor de
  \mintinline{kotlin}|maxHealth| y si luego sobrescribimos la propiedad en alguna de las subclases
  de \mintinline{kotlin}|AbstractBakemon| el constructor podría tener comportamientos ineseparados.
  Para entender por qué esto es un problema veamos un ejemplo:

  \begin{kotlin}
    open class A {
      open val x: Int = 0
      init {
        println("x = $x")
      }
    }
  \end{kotlin}

  \begin{kotlin}
    class B : A() {
      private val l = listOf(1, 2, 3)
      override val x: Int = l.size
    }
  \end{kotlin}

  \begin{kotlin}
    fun main() {
      B()
    }
  \end{kotlin}

  ¿Qué pasa si ejecutamos este código?
  
  \begin{enumerate}
    \item \texttt{B} llama al constructor de su superclase
    \item \texttt{A} inicializa la propiedad \mintinline{kotlin}|x| con el valor 0
    \item \texttt{A} imprime \mintinline{kotlin}|x = 0|
    \item \texttt{B} inicializa la propiedad \mintinline{kotlin}|x| con el tamaño de la lista
  \end{enumerate}

  Esto es un problema ya que si bien definimos que \mintinline{kotlin}|x| es un 
  \mintinline{kotlin}|val|or, vemos que tiene dos valores distintos.

  En nuestro caso, el problema surge de que \textbf{todos} los miembros definidos en una interfaz
  son abiertos (ya que no tiene sentido que no lo sean) y al sobrescribir una componente abierta
  esta se mantiene abierta.

  Ahora pasemos a los tests.
  Partamos por crear un \textit{test factory} para el ataque de un \textit{Bakémon}:

  \begin{kotlin}
    fun `Bakemon attack can be set within limits`(
      factory: BakemonFactory,
      name: String, health: Int, level: Int, attack: Int
    ) = funSpec {
      lateinit var bakemon: Bakemon

      beforeTest {
        factory.name = name
        factory.health = health
        factory.level = level
        factory.attack = attack
        bakemon = factory.createBakemon()
      }

      test("Bakemon attack can be set within limits") {
        bakemon.attack shouldBe 4
        bakemon.attack = 5
        bakemon.attack shouldBe 5
      }

      test("Bakemon attack can't be set to a negative value") {
        bakemon.attack shouldBe 4
        bakemon.attack = -1
        bakemon.attack shouldBe 0
      }
    }
  \end{kotlin}

  El test es bastante similar a lo que hemos visto hasta ahora.
  Y ahí está la gracia, a pesar de que definimos un método para asignar el ataque, nosotros
  ocupamos la propiedad \mintinline{kotlin}|attack| como si fuera una variable cualquiera.
  
  Lo que está sucediendo por detrás es que el compilador de \textit{Kotlin} (al momento de traducir
  el programa a un lenguaje de nivel más bajo) genera un método \texttt{setAttack} y cambia todas
  las asignaciones a la propiedad \mintinline{kotlin}|attack| por llamadas a este método, i.e.
  \mintinline{kotlin}|bakemon.attack = 5| se traduce a \mintinline{kotlin}|bakemon.setAttack(5)|.

  Aprovechemos de agregar un \textit{test factory} para la vida de un \textit{Bakémon}:

  \begin{kotlin}
    fun `Bakemon hp is set within limits`(
      factory: BakemonFactory,
      name: String, health: Int, level: Int, attack: Int
    ) = funSpec {
      lateinit var bakemon: Bakemon

      beforeTest {
        factory.name = name
        factory.health = health
        factory.level = level
        factory.attack = attack
        bakemon = factory.createBakemon()
      }

      test("Bakemon health can be set within limits") {
        bakemon.health shouldBe 25
        bakemon.health = 20
        bakemon.health shouldBe 20
      }

      test("Bakemon health can't be set outside limits") {
        bakemon.health shouldBe 25
        bakemon.health = 30
        bakemon.health shouldBe 25
        bakemon.health = -1
        bakemon.health shouldBe 0
      }
    }
  \end{kotlin}

  Por último, incluimos los \textit{factories} en los tests:

  \begin{kotlin}
    class FireBakemonTest : FunSpec({
      ...
      include(
        `Bakemon hp is set within limits`(FireBakemonFactory(), "Karmander", 25, 5, 4)
      )
      include(
        `Bakemon attack can be set within limits`(
          FireBakemonFactory(), "Karmander", 25, 5, 4
        )
      )
    })  
  \end{kotlin}

  \begin{kotlin}
    class WaterBakemonTest : FunSpec({
      ...
      include(
        `Bakemon hp is set within limits`(WaterBakemonFactory(), "Kokodile", 25, 5, 4)
      )
      include(
        `Bakemon attack can be set within limits`(
          WaterBakemonFactory(), "Kokodile", 25, 5, 4
        )
      )
    })
  \end{kotlin}

  Con esto, podemos correr los tests para ver que pasan todos menos el de las pociones, esto se debe
  a que agregamos una cantidad de vida máxima que no existía antes.
  Para solucionarlo basta cambiar el test de \texttt{PotionTest} por:

  \begin{kotlin}
    test("Potions heal 30 HP to Bakemon") {
      bakemon.health = 100
      potion.useOn(bakemon)
      bakemon.health shouldBe 130
    }
  \end{kotlin}

  Recordemos correr los tests para ver que todo está bien y luego hacer un commit:

  \begin{powershell}
    git add .
    git commit -m "FEAT Adds Bakemon attack and health bounds"
  \end{powershell}

  No sé si necesite decirles a estas alturas, pero tenemos un problema.
  Si bien los \textit{Bakémon} pueden tener un ataque y vida máximos, no hay nada que nos impida
  usar el constructor de \textit{Bakemon} para crear un \textit{Bakémon} con vida y ataque
  negativos.
  ¿Qué podemos hacer para evitar esto?
  Existen varias formas de lidiar con esto, y la respuesta depdende de cómo queremos que se ocupe
  nuestro código, pero una solución posible sería utilizar excepciones.
  Sin embargo, hay un par de cosas que debieramos entender antes de continuar.
