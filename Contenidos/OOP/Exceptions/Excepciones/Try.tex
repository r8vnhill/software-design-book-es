\subsection{Atrapando excepciones}
  \label{sec:atrapando-excepciones}

  En \textit{Kotlin} podemos atrapar excepciones con la expresión 
  \mintinline{kotlin}|try ... catch ... finally|, esto se divide en tres partes:

  \begin{itemize}
    \item \textbf{try}: Contiene el código que puede lanzar una excepción.
    \item \textbf{catch}: Contiene el código que se ejecuta si se lanza una excepción.
    \item \textbf{finally}: Contiene el código que se ejecuta \textbf{siempre}\footnote{SIEMPRE}, ya 
      sea que se lance una excepción o no.
  \end{itemize}

  Por ejemplo, si queremos abrir un archivo y mostrar su contenido en pantalla, podemos hacerlo de 
  la siguiente forma:

  \begin{kotlin}
    fun main() {
      try {
        val file = openFile("file.txt")
        println(file)
      } catch (e: FileNotFoundException) {
        println(e.message)
      } finally {
        println("Finally")
      }
    }
  \end{kotlin}

  Esto entregará un resultado como el siguiente:

  \begin{minted}{text}
    File file.txt not found
    Finally
  \end{minted}

  Siempre que usemos una expresión \mintinline{kotlin}{try} debemos tener una o más expresiones
  \mintinline{kotlin}{catch} que capturan las excepciones que pueden ser lanzadas por el código
  dentro de la expresión \mintinline{kotlin}{try} y/o una expresión \mintinline{kotlin}{finally}
  que se ejecuta siempre, ya sea que se lance una excepción o no.

  \begin{note}
    Como \mintinline{kotlin}|try ... catch ... finally| es una expresión, podemos usarla en
    cualquier lugar donde se pueda usar una expresión.
    Esto significa que podemos usarla para asignar una variable de la misma forma que lo haríamos
    con una expresión \mintinline{kotlin}{if} o \mintinline{kotlin}{when}.
  \end{note}

  