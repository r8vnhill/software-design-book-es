\subsection{Arrojando excepciones}
  Para lanzar una excepción, usamos la palabra reservada \mintinline{kotlin}|throw| seguida de
  una expresión que evalúe a un objeto de tipo \texttt{Throwable}.
  Por ejemplo, si queremos lanzar una excepción de tipo \texttt{Exception} podemos hacerlo de la
  siguiente forma:

  \begin{kotlin}
    throw Exception("Error message")
  \end{kotlin}

  Sin embargo, en la práctica es una mala práctica lanzar una excepción de tipo \texttt{Exception}
  ya que no es específica al problema.
  En su lugar, es mejor crear una excepción propia que herede de \texttt{Exception} y que sea
  más específica al problema que estamos tratando de resolver.

  Para crear una excepción propia, debemos crear una clase que herede de \texttt{Exception}.
  Por ejemplo, si queremos crear una excepción que indique que un archivo no existe, podemos
  hacerlo de la siguiente forma:

  \begin{kotlin}
    class FileNotFoundException(filename: String) : Exception("File $filename not found")
  \end{kotlin}

  Luego podríamos utilizar esta excepción de la siguiente forma:

  \begin{kotlin}
    fun openFile(filename: String): String {
      if (File(filename).exists()) {
        return File(filename).readText()
      } else {
        throw FileNotFoundException(filename)
      }
    }

    fun main() {
      openFile("nonexistent.txt")
    }
  \end{kotlin}

  Esto entregará un resultado como el siguiente:

  \begin{minted}{text}
    Exception in thread "main" FileNotFoundException: File nonexistent.txt not found
      at MainKt.openFile(Main.kt:9)
      at MainKt.main(Main.kt:14)
      at MainKt.main(Main.kt)
  \end{minted}

  Como podemos ver, la excepción se imprime en pantalla junto con la pila de ejecución.
  Esto nos permite saber en qué parte del programa se lanzó la excepción y por qué.

  Adicionalmente, existen dos formas de documentar funciones que lanzan excepciones.
  La primera es agregando el tag \mintinline{kotlin}|@throws| seguido del nombre de la excepción
  que puede ser lanzada por la función y su causa en la documentación de la función.
  Por ejemplo, si queremos documentar que la función \mintinline{kotlin}|openFile| puede lanzar
  una excepción de tipo \texttt{FileNotFoundException}, podemos hacerlo de la siguiente forma:

  \begin{kotlin}
    /**
     * Opens a file and returns its contents.
     *
     * @throws FileNotFoundException if the file does not exist.
     */
    fun openFile(filename: String): String {
      if (File(filename).exists()) {
        return File(filename).readText()
      } else {
        throw FileNotFoundException(filename)
      }
    }
  \end{kotlin}

  La segunda forma es utilizando el tag \mintinline{kotlin}|@Throws| seguido de una lista de
  excepciones que puede lanzar la función.
  Por ejemplo, si queremos documentar que la función \mintinline{kotlin}|openFile| puede lanzar
  una excepción de tipo \texttt{FileNotFoundException}, podemos hacerlo de la siguiente forma:

  \begin{kotlin}
    /**
     * Opens a file and returns its contents.
     */
    @Throws(FileNotFoundException::class)
    fun openFile(filename: String): String {
      if (File(filename).exists()) {
        return File(filename).readText()
      } else {
        throw FileNotFoundException(filename)
      }
    }
  \end{kotlin}

  Esta segunda forma es útil más que nada para poder utilizar estas funciones desde \textit{Java}, 
  ya que \textit{Java} requiere que las funciones que lanzan excepciones lo declaren en la firma
  de la función.
  En \textit{Kotlin} esto no es necesario así que la primera forma es la más recomendada.
  