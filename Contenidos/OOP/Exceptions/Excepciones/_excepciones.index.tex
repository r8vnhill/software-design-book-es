\section{Excepciones}
  \label{sec:excepciones}

  En programación generalmente vamos a tener que tratar con situaciones en las que tengamos que 
  decidir qué hacer cuando algo no sale como esperábamos.
  Por ejemplo, ¿qué sucede si intentamos abrir un archivo que no existe?
  Podríamos imprimir un mensaje de error en pantalla y no hacer nada más, pero en la práctica esto
  no es muy útil.
  De hecho, ¿cómo testeamos que el programa imprime algo en pantalla?
  Un mensaje en pantalla en general es información que deja de existir una vez que se imprime
  y no podemos hacer nada con ella.
  Existen formas de capturar la salida de un programa, pero en general es algo complejo de hacer.

  Una forma más útil de tratar con situaciones en las que algo no sale como esperábamos es lanzar
  una excepción.

  \begin{defaultbox}[Excepción]
    Una excepción es un objeto que representa un comportamiento excepcional que ocurre durante la
    ejecución de un programa.
  \end{defaultbox}

  La principal ventaja de las excepciones es que estas pueden ser arrojadas y capturadas en 
  cualquier parte del programa.
  Cuando una excepción es arrojada, el programa deja de ejecutarse y se salta a la parte del
  código que la captura, para esto, la máquina virtual busca en la pila de ejecución un 
  \textit{frame} que capture la excepción y pasa el control a ese \textit{frame} (los otros 
  \textit{frames} son descartados).
  Si no se encuentra ningún \textit{frame} que capture la excepción, se imprime un mensaje de error
  y el programa termina.

  En \textit{Kotlin} las excepciones son un tipo particular de objetos que heredan de la clase
  \textit{Throwable}.
  Existen dos tipos de \texttt{Throwable}: \texttt{Exception} y \texttt{Error}.
  
  Los \texttt{Error} son errores que ocurren en la máquina virtual y que no deben ser capturados.
  Por ejemplo, si se lanza una \texttt{StackOverflowError} es porque la pila de ejecución se ha
  llenado y no hay más espacio para seguir ejecutando el programa.
  El nombre de las clases que heredan de \texttt{Error} terminan en \texttt{Error} para indicar
  que no deben ser capturadas.

  Las \texttt{Exception} por otro lado, son errores que ocurren dentro de nuestro programa y que 
  pueden ser manejadas por el mismo.
  Las \texttt{Exception} pueden clasificarse en dos tipos: \textit{excepciones propias} y
  \textit{excepciones de la librería estándar}.
  Las \textit{excepciones propias} son las que creamos nosotros para indicar que algo salió mal
  en nuestro programa.
  Las \textit{excepciones de la librería estándar} son las que lanzan las funciones de la librería
  estándar de \textit{Kotlin} cuando algo sale mal, este tipo de excepciones pueden ser capturadas,
  pero no es recomendable hacerlo ya que, al no ser parte de nuestro código, no tenemos seguridad
  de cuál es la razón por la que se lanzó la excepción.

  \subimport{.}{Throw.tex}
  \subimport{.}{Try.tex}
  \subimport{.}{Usando_excepciones.tex}