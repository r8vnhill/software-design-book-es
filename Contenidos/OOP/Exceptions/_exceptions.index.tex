\chapter{Cuando todo male sal}
  Este capítulo podría llamarse algo así como \enquote{\textit{Kotlin} 2: La venganza}.
  Aquí seguiremos expandiendo el ejemplo de los capítulos anterioress, pero deteniendonos en ciertos
  detalles que pueden mejorarse de lo que ya hemos visto.

  Comencemos implementando algunas funcionalidades nuevas para hacer más interesante nuestro juego.

  Algo común que podríamos querer hacer es que un jugador pueda usar ítems en un combate.
  Por ejemplo, un jugador podría usar un ítem para curarse, o para aumentar su ataque.

  Partamos creando una interfaz para los ítems, diremos que los ítems tienen un nombre, una 
  descripción y una función que se ejecuta cuando el ítem es usado.
  Crearemos la interfaz en un paquete llamado \url{cl.ravenhill.items}.

  \begin{kotlin}
    interface Item {
      var name: String
      var description: String
      fun useOn(bakemon: Bakemon)
    }
  \end{kotlin}

  Ahora crearemos dos ítems, uno para curar y otro para aumentar el ataque.
  Los llamaremos \texttt{Potion} y \texttt{RaiseAttack} respectivamente.
  Una \texttt{Potion} cura 30 puntos de vida, y un \texttt{RaiseAttack} aumenta el ataque en un 
  10\%.

  Comencemos creando los tests para los ítems.

  \begin{kotlin}
    class PotionTest : FunSpec({
      lateinit var potion: Potion
      lateinit var bakemon: FireBakemon

      beforeTest {
        potion = Potion()
        bakemon = FireBakemon("Karmander", 100, 5, 4)
      }
      
      test("Potions heal 30 HP to Bakemon") {
        potion.useOn(bakemon)
        bakemon.health shouldBe 55
      }
    })
  \end{kotlin}

  \begin{kotlin}
    class RaiseAttackTest : FunSpec({
      lateinit var raiseAttack: RaiseAttack
      lateinit var bakemon: FireBakemon

      beforeTest {
        raiseAttack = RaiseAttack()
        bakemon = FireBakemon("Karmander", 100, 10, 20)
      }

      test("RaiseAttack increase attack by 10%") {
        raiseAttack.useOn(bakemon)
        bakemon.attack shouldBe 22
      }
    })
  \end{kotlin}

  Como siempre, podemos correr los tests para ver que fallan.

  Ahora veamos como implementar estos ítems.

  \begin{kotlin}
    class Potion : Item {
      override var name: String = "Potion"
      override var description: String = "Heals 30 HP to Bakemon"

      override fun useOn(bakemon: Bakemon) {
        bakemon.health += 30
      }
    }
  \end{kotlin}

  \begin{kotlin}
    class RaiseAttack : Item {
      override var name: String = "Raise Attack"
      override var description: String = "Increases attack by 10%"

      override fun useOn(bakemon: Bakemon) {
        bakemon.attack = (bakemon.attack * 1.1).toInt()
      }
    }
  \end{kotlin}

  Notarán que ahora \textit{IntelliJ} nos reclama attack es un \mintinline{kotlin}|val|or, 
  cambiémoslo a \mintinline{kotlin}|var|.

  Con esto ya debieramos poder correr los tests y ver que pasan.
  Ahora hacemos \textit{commit} y nos preparamos para el primer problema de este capítulo.

  \begin{powershell}
    git add .
    git commit -m "FEAT Adds items"
  \end{powershell}

  \subimport{.}{Setters.tex}
  \subimport{.}{JVM.tex}
  \subimport{Excepciones/}{_excepciones.index.tex}
  \subimport{.}{Getters.tex}
  \subimport{.}{Ejercicios.tex}
  
  \printbibliography[keyword=exceptions]