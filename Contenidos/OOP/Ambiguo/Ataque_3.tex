\section{Ataques: Terccer intento}
  \label{sec:ataques-tercer-intento}

  Volvamos a nuestro proyecto y veamos cómo podemos implementar los ataques utilizando 
  \textit{double dispatch}.
  Para esto, primero agreguemos los métodos necesarios a la interfaz \mintinline{kotlin}|Bakemon|:
  \begin{kotlin}
    interface Bakemon {
      ...
      fun attackBakemon(bakemon: Bakemon)
      fun attackedByFireBakemon(bakemon: FireBakemon)
      fun attackedByWaterBakemon(bakemon: WaterBakemon)
      fun attackedByGrassBakemon(bakemon: GrassBakemon)
    }
  \end{kotlin}

  Ahora, implementemos estos métodos en cada una de las clases que implementan 
  \mintinline{kotlin}|Bakemon|:

  \begin{kotlin}
    class FireBakemon(name: String, health: Int, level: Int, attack: Int) :
        AbstractBakemon(name, health, level, attack) {
      override fun attackBakemon(bakemon: Bakemon) = bakemon.attackedByFireBakemon(this)

      override fun attackedByFireBakemon(bakemon: FireBakemon) =
        takeDamage(bakemon.attack)
      override fun attackedByWaterBakemon(bakemon: WaterBakemon) =
        takeDamage((bakemon.attack * 1.5).toInt())
      override fun attackedByGrassBakemon(bakemon: GrassBakemon) =
        takeDamage(bakemon.attack / 2)
      ...
    }
  \end{kotlin}

  \begin{kotlin}
    class WaterBakemon(name: String, health: Int, level: Int, attack: Int) :
        AbstractBakemon(name, health, level, attack) {
      override fun attackBakemon(bakemon: Bakemon) = bakemon.attackedByWaterBakemon(this)

      override fun attackedByFireBakemon(bakemon: FireBakemon) =
        takeDamage(bakemon.attack / 2)
    
      override fun attackedByWaterBakemon(bakemon: WaterBakemon) =
        takeDamage(bakemon.attack)
    
      override fun attackedByGrassBakemon(bakemon: GrassBakemon) =
        takeDamage((bakemon.attack * 1.5).toInt())
      ...
    }
  \end{kotlin}

  Pero

  \begin{center}
    Ya empezó \texttt{¬¬}
  \end{center}

  tenemos un problema.

  Estamos repitiendo los mismos cálculos de daño en cada una de las clases que implementan, esto
  podría no parecer tanto problema, pero si tenemos 10 tipos de Bakemon, entonces tendremos que
  implementar 10 veces el mismo código, pero más importante aún, si queremos cambiar el cálculo
  de daño, tendremos que hacerlo en 10 lugares distintos.
  Es decir que no encapsulamos el cambio.

  Podemos resolver esto de manera simple creando métodos auxiliares en la clase abstracta.
  Crearemos dos métodos adicionales, uno para calcular el daño de una debilidad y otro para
  calcular el daño de una resistencia.
  Pero preguntémonos primero.
  ¿Desde dónde quiero usar estos métodos?
  
  \begin{center}
    Desde los métodos \mintinline{kotlin}|attackedBy|? uwu
  \end{center}

  Asies, entonces, si pesa lo mismo que un pato...

  \begin{center}
    \textit{¡Es bruja!}
  \end{center}

  Estos métodos los vamos a implementar en la clase abstracta y lo vamos a utilizar solo desde sus 
  subclases, entonces, como por la propiedad de encapsulamiento queremos restringir el acceso a
  las componentes internas de un objeto lo más posible, nos gustaría que los métodos 
  \mintinline{kotlin}{takeDamage} fueran privadas dentro de jerarquía de clases.
  Por suerte, \textit{Kotlin} tiene una pequeña palabra secreta que pareciera que nos leyó la mente:
  \mintinline{kotlin}{protected}.
  
  \begin{defaultbox}[Modificador de privilegios \mintinline{kotlin}{protected}]
    \index{Modificador de privilegios \mintinline{kotlin}{protected}}
    Un miembro \mintinline{kotlin}{protected} es accesible desde la clase que lo define y desde las
    que heredan directa o indirectamente de ella.    
  \end{defaultbox}
  
  Entonces, los métodos \texttt{takeDamage} podrían ser protected.
  Pero los miembros de una interfaz deben ser públicas, entonces eliminaremos al método 
  \mintinline{kotlin}{Bakemon::takeDamage(Int)} de la interfaz y lo implementaremos sólo en la clase
  abstracta.

  \begin{kotlin}
    abstract class AbstractBakemon(...) : Bakemon {
      protected fun takeDamage(damage: Int) {
        health -= damage
        if (health < 0) {
          health = 0
        }
      }

      protected fun takeWeaknessDamage(damage: Int) = takeDamage((damage * 1.5).toInt())

      protected fun takeResistanceDamage(damage: Int) = takeDamage(damage / 2)
    }
  \end{kotlin}

  Este cambio va a romper nuestros tests, ya que ahora los métodos \mintinline{kotlin}{takeDamage}
  no son visibles desde ellos.
  Esto es esperado, lo primero que haremos será borrar la función 
  \mintinline{kotlin}{`Bakemon can take damage`} del archivo \texttt{TestFactories.kt} y todos sus
  usos en los tests.

  Para lo siguiente vamos a necesitar hacer un cambio un poco más grande.
  Vamos a modificar nuestros factories para hacerlos más flexibles.

  Otra de las ventajas de utilizar \textit{factory method}\index{Factory method pattern} es que
  podemos crear objetos de manera dinámica, es decir, podemos reutilizar una misma factory para
  crear objetos que varían poco entre sí.
  La idea es permitir asignar valores a las factories después de que estas ya hayan sido creadas.
  Para esto partamos por modificar la interfaz \mintinline{kotlin}{BakemonFactory} para que tenga
  variables de instancia que puedan ser modificadas y simplifiquemos el método 
  \mintinline{kotlin}|createBakemon| para que no reciba ningún parámetro.

  Pero utilicemos los tests para tener una idea de cómo debería quedar la interfaz.

  \begin{kotlin}
    class FireBakemonFactoryTest : FunSpec({
      ...
      beforeTest {
        factory = FireBakemonFactory()
        factory.name = name
        factory.health = health
        factory.level = level
        factory.attack = attack
      }

      test("A FireBakemon should be created") {
        val bakemon = factory.createBakemon()
        bakemon.name shouldBe name
        bakemon.health shouldBe health
        bakemon.level shouldBe level
        bakemon.attack shouldBe attack
      }
    })
  \end{kotlin}

  \begin{kotlin}
    class WaterBakemonFactoryTest : FunSpec({
      ...
      beforeTest {
          factory = WaterBakemonFactory()
          factory.name = name
          factory.health = health
          factory.level = level
          factory.attack = attack
      }

      test("A WaterBakemon should be created") {
          val bakemon = factory.createBakemon()
          bakemon.name shouldBe name
          bakemon.health shouldBe health
          bakemon.level shouldBe level
          bakemon.attack shouldBe attack
      }
    })
  \end{kotlin}

  Aquí notemos que en ambos tests tenemos el mismo código, entonces podemos pensar en una 
  \textit{test factory} que nos permita reusar el código común:

  \begin{kotlin}
    fun `a Bakemon factory can create Bakemon`(
      factory: BakemonFactory, name: String, health: Int, level: Int, attack: Int
    ) = funSpec {
      beforeTest {
        factory.name = name
        factory.health = health
        factory.level = level
        factory.attack = attack
      }

      test("A Bakemon should be created") {
        val bakemon = factory.createBakemon()
        bakemon.name shouldBe name
        bakemon.health shouldBe health
        bakemon.level shouldBe level
        bakemon.attack shouldBe attack
      }
    }
  \end{kotlin}

  Ahora podemos reescribir nuestros tests de la siguiente manera:

  \begin{kotlin}
    class FireBakemonFactoryTest : FunSpec({
      val name = "Karmander"
      val health = 25
      val level = 5
      val attack = 4
      
      include(
        `a Bakemon factory can create Bakemon`(
          FireBakemonFactory(), name, health, level, attack
        )
      )
    })
  \end{kotlin}

  \begin{kotlin}
    class WaterBakemonFactoryTest : FunSpec({
      val name = "Kokodile"
      val health = 25
      val level = 5
      val attack = 4

      include(
        `a Bakemon factory can create Bakemon`(
          WaterBakemonFactory(), name, health, level, attack
        )
      )
    })
  \end{kotlin}
    
  Ahora nos quedaría modificar las funciones \mintinline{kotlin}|`Bakemon can attack Bakemon`| y
  \mintinline{kotlin}|`Bakemon equality and hashcode test`|
  para que
  utilice la nueva interfaz de las factories.

  \begin{kotlin}
    fun `Bakemon equality and hashcode test`(...) = funSpec {
      ...
      beforeTest {
        factory.name = name
        factory.health = health
        factory.level = level
        factory.attack = attack
        bakemon = factory.createBakemon()
      }
      ...
    }

    fun `Bakemon can attack Bakemon`(...) = funSpec {
      ...
      beforeTest {
        factory.name = name
        factory.health = health
        factory.level = level
        factory.attack = attack
        bakemon = factory.createBakemon()
        enemyFactory.name = enemyName
        enemyFactory.health = enemyHealth
        enemyFactory.level = enemyLevel
        enemyFactory.attack = enemyAttack
        enemy = enemyFactory.createBakemon()
      }
      ...
      test(testNames.second) {
        factory.attack = enemyHealth * 2
        bakemon = factory.createBakemon()
        bakemon.attackBakemon(enemy)
        enemy.health shouldBe 0
      }
    }
  \end{kotlin}

  Noten que en el test ahora podemos modificar el ataque del \textit{Bakémon} enemigo antes de
  crearlo.

  Naturalmente, si corremos los tests ahora, estos van a fallar.
  
  Procedamos a modificar la interfaz \mintinline{kotlin}{BakemonFactory} para que tenga variables
  de instancia que puedan ser modificadas.

  \begin{kotlin}
    interface BakemonFactory {
      var name: String
      var health: Int
      var level: Int
      var attack: Int
      fun createBakemon(): Bakemon
    }
  \end{kotlin}

  Y la implementación de las factories:

  \begin{kotlin}
    class FireBakemonFactory : BakemonFactory {
      override lateinit var name: String
      override var health: Int = 0
      override var level: Int = 0
      override var attack: Int = 0

      override fun createBakemon() = FireBakemon(name, health, level, attack)
    }
  \end{kotlin}

  \begin{kotlin}
    class WaterBakemonFactory : BakemonFactory {
      override lateinit var name: String
      override var health: Int = 0
      override var level: Int = 0
      override var attack: Int = 0

      override fun createBakemon() = WaterBakemon(name, health, level, attack)
    }
  \end{kotlin}

  \begin{note}
    La palabra reservada \mintinline{kotlin}{lateinit} no puede ser usada en variables de tipos
    numéricos, por lo que en este caso tenemos que inicializarlas con un valor por defecto.
  \end{note}

  Nos quedaría modificar los métodos \mintinline{kotlin}|attackedBy| de las clases
  \mintinline{kotlin}{FireBakemon} y \mintinline{kotlin}{WaterBakemon} para que utilicen nuestras
  nuevas implementaciones de \mintinline{kotlin}{takeDamage} de la siguiente manera:

  \begin{kotlin}
    class FireBakemon(name: String, health: Int, level: Int, attack: Int) :
        AbstractBakemon(name, health, level, attack) {
      override fun attackBakemon(bakemon: Bakemon) = bakemon.attackedByFireBakemon(this)

      override fun attackedByFireBakemon(bakemon: FireBakemon) =
        takeDamage(bakemon.attack)

      override fun attackedByWaterBakemon(bakemon: WaterBakemon) =
        takeWeaknessDamage(bakemon.attack)

      override fun attackedByGrassBakemon(bakemon: GrassBakemon) =
        takeResistanceDamage(bakemon.attack)
      ...
    }
  \end{kotlin}

  \begin{kotlin}
    class WaterBakemon(
      name: String,
      health: Int,
      level: Int,
      attack: Int
    ) : AbstractBakemon(name, health, level, attack) {
      override fun attackBakemon(bakemon: Bakemon) = bakemon.attackedByWaterBakemon(this)

      override fun attackedByFireBakemon(bakemon: FireBakemon) =
        takeResistanceDamage(bakemon.attack)

      override fun attackedByWaterBakemon(bakemon: WaterBakemon) =
        takeDamage(bakemon.attack)

      override fun attackedByGrassBakemon(bakemon: GrassBakemon) =
        takeWeaknessDamage(bakemon.attack)
      ...
    }
  \end{kotlin}

  Con esto deberíamos poder correr los tests y ver que todos pasan.
  Nos queda hacer \textit{commit} de los cambios.

  \begin{powershell}
    git add .
    git commit -m "REFACTOR Adapts code to use Double Dispatch on Bakemon attacks"
  \end{powershell}
