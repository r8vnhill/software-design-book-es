\section{Abstract Factory Pattern}
\label{sec:abstract_factory_pattern}
  Ya hicimos tests e implementamos los tipos de \textit{Bakémon} que queríamos, nos falta el último
  paso del TDD: refactorizar.

  Comencemos con la siguiente pregunta: ¿Existen los Bakémon sin tipo?
  Podría ser, pero en la manera que estamos modelando el juego no tiene mucho sentido.
  ¿De qué nos sirve entonces la clase \textit{Bakémon}?
  

  No hay mucho que hacer para nuestros tipos de \textit{Bakémon}, pero notarán que los tests que
  escribimos para las clases de \textit{Bakémon} son muy similares.
  ¿Hay alguna forma de reusar ese código?
  La respuesta es que sí, pero primero debemos solucionar un problema importante.

  Veamos qué es lo que cambia entre los tests de las clases de \textit{Bakémon}.
  En todos los tests, creamos un objeto de la clase que estamos probando, y luego comparamos
  el objeto creado con otro objeto creado con los mismos parámetros.
  Noten que lo único que cambia entre los tests es el tipo de \textit{Bakémon} que estamos
  probando.
  En otras palabras, estamos concretizando la clase \texttt{Bakemon} en cada test.
  
  \textit{Kotest} provee una funcionalidad llamada \textit{test factories} que nos permite
  reusar tests de una manera muy sencilla.
  Para usarlas, crearemos una función que va a utilizar un método de \textit{Kotest} llamado
  \texttt{funSpec}.
  Se entenderá mejor poniéndolo en práctica.

  Primero, creemos un archivo en el paquete \url{cl.ravevnhill.bakemon} del directorio de tests 
  haciendo click derecho en el paquete y seleccionando \textit{New} \(\rightarrow\) 
  \textit{Kotlin File/Class}, y en la ventana que se abre, seleccionaremos \textit{File} y
  lo llamaremos \textit{TestFactories}.

  \begin{kotlin}
    
  \end{kotlin}