\section{Ejercicios}
  \begin{important}
    Recuerde:
    \begin{itemize}
      \item Hacer \textit{commit} de sus cambios luego de cada ejercicio.
      \item Definir el método \texttt{toString} en las clases que lo requieran.
    \end{itemize}
  \end{important}

  \begin{Exercise}
    \textit{Smalltalk} es un lenguaje de programación puramente orientado a objetos.
    Es un lenguaje de programación que se utilizó mucho para el desarrollo de la programación 
    orientada a objetos.
    \textit{Smalltalk} es considerado como el primer lenguaje de programación orientado a objetos y
    el primero con tipado dinámico.
    En \textit{Smalltalk} todo es un objeto, incluso los números y los booleanos, además las 
    instrucciones de control de flujo son métodos de los objetos.

    Al ser completamente dinámico ningún objeto tiene un tipo definido en tiempo de compilación por
    lo que cuando se envía un mensaje, el receptor del mensaje será el que decida qué hacer con él.

    \Question Cree una interfaz \texttt{SmallTalkNumber} en el paquete \url{cl.ravenhill.lilvoices}
      con lo siguiente:
      \begin{minted}{kotlin}
        interface SmalltalkNumber {
          val value: Number

          fun plus(other: SmalltalkNumber): SmalltalkNumber
          fun minus(other: SmalltalkNumber): SmalltalkNumber
          fun times(other: SmalltalkNumber): SmalltalkNumber
          fun dividedBy(other: SmalltalkNumber): SmalltalkNumber
      }
      \end{minted}
    \Question Cree una clase \texttt{SmalltalkIntegerTest} que pruebe que un número entero puede ser 
      creado con un valor entero de \textit{Kotlin} como parámetro.
    \Question Programe la clase \texttt{SmalltalkInteger} que implemente la interfaz 
      \texttt{SmalltalkNumber} y que pueda ser creado con un valor entero de \textit{Kotlin} como 
      parámetro.
    \Question Cree una clase \texttt{SmalltalkFloatTest} que pruebe que un número flotante puede ser
      creado con un valor flotante de \textit{Kotlin} como parámetro.
      Utilice el tipo \mintinline{kotlin}|Float| de \textit{Kotlin} para representar los números
      flotantes.
    \Question Programe la clase \texttt{SmalltalkFloat} que implemente la interfaz 
      \texttt{SmalltalkNumber} y que pueda ser creado con un valor flotante de \textit{Kotlin} como 
      parámetro.
    \Question Escriba tests para probar que los números enteros pueden ser sumados, restados, 
      multiplicados y divididos.
    \Question Implemente las operaciones de suma, resta, multiplicación, división, módulo y potencia
      entre números enteros.
    \Question Escriba tests para probar que los números flotantes pueden ser sumados, restados,
      multiplicados y divididos.
    \Question Implemente las operaciones de suma, resta, multiplicación y división entre números 
      flotantes.
    \ExeText Para los siguientes ejercicios considere las siguientes reglas:
      \begin{itemize}
        \item Las operaciones entre números del mismo tipo retornan un número del mismo tipo.
        \item Las operaciones entre números de distinto tipo retornan un número del tipo más 
          preciso (en este caso, el tipo \mintinline{kotlin}|SmalltalkFloat|).
      \end{itemize}
    \Question Escriba tests para probar que los números enteros y flotantes pueden ser sumados,
      restados, multiplicados y divididos.
    \Question Implemente las operaciones de suma, resta, multiplicación y división entre números
      enteros y flotantes utilizando \textit{double dispatch} para desambiguar el tipo de los
      operandos.
  \end{Exercise}