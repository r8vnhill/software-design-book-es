 \subsection{Atacando}

  Ahora que tenemos implementado el método \\\mintinline{kotlin}|Bakemon::takeDamage(Int)|, podemos
  implementar el método \mintinline{kotlin}|Bakemon::attackBakemon(Bakemon)|.
  Como siempre, comencemos por los tests.

  \begin{kotlin}
    fun `Bakemon can attack Bakemon`(
      factory: BakemonFactory,
      name: String,
      health: Int,
      level: Int,
      attack: Int,
      enemyFactory: BakemonFactory,
      enemyName: String,
      enemyHealth: Int,
      enemyLevel: Int,
      enemyAttack: Int,
      expectedDamage: Int
    ) = funSpec {
      lateinit var bakemon: Bakemon
      lateinit var enemy: Bakemon

      beforeTest {
        bakemon = factory.createBakemon(name, health, level, attack)
        enemy = enemyFactory.createBakemon(enemyName, enemyHealth, enemyLevel, enemyAttack)
      }

      test("An attack should reduce the enemy's health") {
        bakemon.attackBakemon(enemy)
        enemy.health shouldBe enemyHealth - expectedDamage
      }

      test("An attack should not set the enemy's health below 0") {
        enemy.takeDamage(enemyHealth)
        bakemon.attackBakemon(enemy)
        enemy.health shouldBe 0
      }
    }
  \end{kotlin}

  Este método debería recibir un \textit{Bakémon} y realizar el cálculo de daño correspondiente.
  Con esto, ya tenemos los tests para los ataques entre \textit{Bakémon} de distintos tipos.
  Noten que reusamos la función \mintinline{kotlin}|`Bakemon can fight`| para definir peleas entre
  \textit{Bakémon} de distintos tipos.
  
  Ahora incluyamos el \textit{test factory} en los tests de cada tipo de \textit{Bakémon}:

  \begin{kotlin}
    class FireBakemonTest : FunSpec({
      ...
      include(
        "Fire Bakemon -> Fire Bakemon:",
        `Bakemon can attack Bakemon`(
          FireBakemonFactory(), "Karmander", 25, 5, 4,
          FireBakemonFactory(), "Karmander", 25, 5, 4, 4
        )
      )
      include(
        "Fire Bakemon -> Water Bakemon:",
        `Bakemon can attack Bakemon`(
          FireBakemonFactory(), "Karmander", 25, 5, 4,
          WaterBakemonFactory(), "Kokodile", 25, 5, 4, 2
        )
      )
    })
  \end{kotlin}

  \begin{kotlin}
    class WaterBakemonTest : FunSpec({
      ...
      include(
        "Water Bakemon -> Fire Bakemon:",
        `Bakemon can attack Bakemon`(
          WaterBakemonFactory(), "Kokodile", 25, 5, 4,
          FireBakemonFactory(), "Karmander", 25, 5, 4, 8
        )
      )
      include(
        "Water Bakemon -> Water Bakemon:",
        `Bakemon can attack Bakemon`(
          WaterBakemonFactory(), "Kokodile", 25, 5, 4,
          WaterBakemonFactory(), "Kokodile", 25, 5, 4, 4
        )
      )
    })
  \end{kotlin}

  Corramos los tests para ver cómo fallan.

  Ahora, pasemos a implementar el método \mintinline{kotlin}|Bakemon::attackBakemon(Bakemon)|.

  Como los ataques no consideran fortalezas o debilidades, debieramos notar que el método
  va a ser idéntico para todos los tipos de \textit{Bakémon}, así que podemos definirlo en la
  clase abstracta.

  \begin{kotlin}
    abstract class AbstractBakemon(...) : Bakemon {
      ...
      override fun attackBakemon(bakemon: Bakemon) {
        bakemon.takeDamage(attack)
      }
    }
  \end{kotlin}

  Corramos los tests y veamos que ahora pasan.
  Por útlimo, podemos hacer \textit{commit} de los cambios.

  \begin{kotlin}
    git add .
    git commit -m "FEAT Adds attackBakemon method"
  \end{kotlin}