 
  Con esto, ya tenemos los tests para los ataques entre \textit{Bakémon} de distintos tipos.
  Noten que reusamos la función \mintinline{kotlin}|`Bakemon can fight`| para definir peleas entre
  \textit{Bakémon} de distintos tipos.
  Aquí es donde entra en juego el parámetro \mintinline{kotlin}|message| que le pasamos a la 
  función, ya que así evitamos tener varios tests con el mismo nombre.\footnote{Si bien no es
  necesario, es una buena práctica que los nombres no se repitan dentro de un mismo test.}

  Corramos los tests para ver cómo fallan.

  Ahora, pasemos a implementar el método \mintinline{kotlin}|FireBakemon::attackBakemon(Bakemon)|.

  \begin{kotlin}
    override fun attackBakemon(bakemon: Bakemon) {
      bakemon.health -= attack
    }
  \end{kotlin}

  Si intentamos correr los tests veremos que siguen fallando.
  Este error debiera sernos conocido, ya que lo vimos al momento de hablar sobre valores y 
  variables.
  Esto podemos solucionarlo simplemente cambiando la propiedad \mintinline{kotlin}|health| de
  \mintinline{kotlin}{Bakemon}, \mintinline{kotlin}{WaterBakemon} y \mintinline{kotlin}{FireBakemon}
  a \mintinline{kotlin}{var}.