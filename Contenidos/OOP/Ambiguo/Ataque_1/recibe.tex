\subsection{Recibiendo daño}
  Comencemos por lo más simple y digamos que los \textit{Bakémon} pueden atacarse entre sí ignorando
  las fortalezas y debilidades de los tipos.
  Para esto, vamos a crear primero un método \texttt{Bakemon::takeDamage(Int)} que nos permitirá
  reducir la vida de un \textit{Bakémon} en una cantidad determinada asegurándonos de que la vida no
  sea negativa.

  \begin{kotlin}
    interface Bakemon {
      ...
      fun takeDamage(damage: Int)
    }
  \end{kotlin}

  Luego, podemos implementar los tests para este método, para esto podemos crear un \textit{test 
  factory} que nos permita crear los tests para cada tipo de \textit{Bakémon} de forma sencilla:

  \begin{kotlin}
    fun `Bakemon can take damage`(
      factory: BakemonFactory,
      name: String,
      health: Int,
      level: Int,
      attack: Int
    ) = funSpec {
      lateinit var bakemon: Bakemon

      beforeTest {
        bakemon = factory.createBakemon(name, health, level, attack)
      }
      test("Taking damage should reduce the Bakemon's health") {
        bakemon.takeDamage(5)
        bakemon.health shouldBe 20
      }
      test(
        "Taking damage should reduce the Bakemon's health to 0 if the damage is " +
            "greater than its health"
      ) {
        bakemon.takeDamage(30)
        bakemon.health shouldBe 0
      }
    }
  \end{kotlin}

  Ahora podemos crear los tests para cada tipo de \textit{Bakémon} de forma sencilla:

  \begin{kotlin}
    class FireBakemonTest : FunSpec({
      ...
      include(`Bakemon can take damage`(FireBakemonFactory(), "Karmander", 25, 5, 4))
    })
  \end{kotlin}

  \begin{kotlin}
    class FireBakemonTest : FunSpec({
      ...
      include(`Bakemon can take damage`(WaterBakemonFactory(), "Kokodile", 25, 5, 4))
    })
  \end{kotlin}

  Ahora podemos implementar el método \texttt{Bakemon::takeDamage(Int)} en cada tipo concreto, pero
  antes, 

  \begin{kotlin}
    class FireBakemon(...) : Bakemon {
      ...
      override fun takeDamage(damage: Int) {
        health -= damage
        if (health < 0) {
          health = 0
        }
      }
    }
  \end{kotlin}

  \begin{kotlin}
    class WaterBakemon(...) : Bakemon {
      ...
      override fun takeDamage(damage: Int) {
        health -= damage
        if (health < 0) {
          health = 0
        }
      }
    }
  \end{kotlin}
  
  Ahora podemos correr los tests y ver que todos pasan \texttt{:)}

  Nos quedaría mejorar nuestro diseño, y aquí debieran notar que el método \texttt{takeDamage}
  es idéntico para todos los tipos de \textit{Bakémon}, por lo que podríamos usar una clase
  abstracta para implementarlo.
  Para esto primero creemos un nuevo paquete \url{cl.ravenhill.bakemon.types} tanto en la aplicación
  como en los tests, y movamos todas las clases de tipos de \textit{Bakémon} a este nuevo paquete.
  Ahora, creemos una clase abstracta \texttt{AbstractBakemon} que implemente la interfaz
  \texttt{Bakemon} y que contenga el método \texttt{takeDamage}:

  \begin{kotlin}
    abstract class AbstractBakemon(
      override val name: String,
      override var health: Int,
      override val level: Int,
      override val attack: Int
    ) : Bakemon {
      override fun takeDamage(damage: Int) {
        health -= damage
      }
    }
  \end{kotlin}

  Ahora, podemos hacer que las clases de tipos de \textit{Bakémon} hereden de esta clase abstracta
  y eliminar el método \texttt{takeDamage} de cada una de ellas:

  \begin{kotlin}
    class FireBakemon(name: String, health: Int, level: Int, attack: Int) :
        AbstractBakemon(name, health, level, attack) {
      override fun equals(other: Any?) = when {...}
      override fun hashCode() = ...
      override fun toString() = ...
    }
  \end{kotlin}

  \begin{kotlin}
    class WaterBakemon(name: String, health: Int, level: Int, attack: Int) :
        AbstractBakemon(name, health, level, attack) {
      override fun equals(other: Any?) = when {...}
      override fun hashCode() = ...
      override fun toString() = ...
    }
  \end{kotlin}