\section{Ataques: la cuarta es la vencida}
  \label{sec:ataques-4}

  Lo último que haremos en este capítulo es hacer un sistema de combates un poco más interesante.
  Hasta ahora tenemos que cada \textit{Bakémon} tiene un tipo y de acuerdo a ese tipo, tiene 
  ventajas y debilidades.
  Además cada \textit{Bakémon} tiene un sólo ataque del mismo tipo que el \textit{Bakémon}.

  Nos gustaría que cada \textit{Bakémon} tuviera varios ataques, y que cada ataque tenga un tipo.
  Además, nos gustaría que cada \textit{Bakémon} pueda tener hasta cuatro ataques diferentes.

  \subsection{Ataques}
    Partamos por crear un paquete \texttt{attacks} y dentro de él, una interfaz \texttt{Attack} con
    lo siguiente:

    \begin{kotlin}
      interface Attack {
        val name: String
        val damage: Int
      }
    \end{kotlin}

    Con esto, podemos comenzar con los tests.

    \begin{kotlin}
      class FireAttackTest : FunSpec({
        val name = "Ember"
        val damage = 10

        lateinit var attack: Attack

        beforeTest {
          attack = FireAttack(name, damage)
        }

        test("Two attacks with the same name and damage should be equal") {
          val a = FireAttack(name, damage)
          attack shouldBe a
        }
        
        test("Two attacks with the same name and damage should have the same hashCode") {
          val a = FireAttack(name, damage)
          attack shouldHaveSameHashCodeAs a
        }
      })
    \end{kotlin}

    \begin{kotlin}
      class WaterAttackTest : FunSpec({
        val name = "Water Gun"
        val damage = 10

        lateinit var attack: Attack

        beforeTest {
          attack = WaterAttack(name, damage)
        }

        test("Two attacks with the same name and damage should be equal") {
          val a = WaterAttack(name, damage)
          attack shouldBe a
        }

        test("Two attacks with the same name and damage should have the same hashCode") {
          val a = WaterAttack(name, damage)
          attack shouldHaveSameHashCodeAs a
        }
      })
    \end{kotlin}

    Veamos que los tests fallan y pasemos a implementar las clases \texttt{FireAttack} y 
    \texttt{WaterAttack}.

    \begin{kotlin}
      class FireAttack(
        override val name: String,
        override val damage: Int
      ) : Attack {
        override fun equals(other: Any?) = when {
          this === other -> true
          other !is FireAttack -> false
          else -> name == other.name &&
              damage == other.damage
        }

        override fun hashCode() = Objects.hash(FireAttack::class, name, damage)

        override fun toString() = "FireAttack(name='$name', damage=$damage)"
      }
    \end{kotlin}

    \begin{kotlin}
      class WaterAttack(override val name: String, override val damage: Int) : Attack {
        override fun equals(other: Any?) = when {
          this === other -> true
          other !is WaterAttack -> false
          else -> name == other.name &&
                  damage == other.damage
        }

        override fun hashCode() = Objects.hash(WaterAttack::class, name, damage)

        override fun toString() = "WaterAttack(name='$name', damage=$damage)"
      }
    \end{kotlin}

    Si corremos los tests, veremos que ahora pasan.

    Veremos que nuevamente los tests que escribimos son casi idénticos, creemos un factory de ataques
    para evitar repetir código.

    Comencemos por crear un \textit{test factory} para probar nuestras fábricas.

    \begin{kotlin}
      fun `an attack factory can create attacks`(
        factory: AttackFactory,
        name: String,
        damage: Int
      ) = funSpec {
        lateinit var attack: Attack

        beforeTest {
          factory.name = name
          factory.damage = damage
          attack = factory.createAttack()
        }

        test("An attack should be created") {
          attack.name shouldBe name
          attack.damage shouldBe damage
        }
      }
    \end{kotlin}

    Ahora podemos usarlo en nuestros tests.
    Creemos las clases \texttt{FireAttackFactoryTest} y \texttt{WaterAttackFactoryTest} en el paquete
    \texttt{factories}.

    \begin{kotlin}
      class FireAttackFactoryTest : FunSpec({
        include(`an attack factory can create attacks`(FireAttackFactory(), "Ember", 10))
      })
    \end{kotlin}

    \begin{kotlin}
      class WaterAttackFactoryTest : FunSpec({
        include(
          `an attack factory can create attacks`(WaterAttackFactory(), "Water Gun", 10)
        )
      })
    \end{kotlin}

    Con esto podemos comenzar a implementar las fábricas.

    Crearemos una interfaz \texttt{AttackFactory} y tres clases que la implementen:
    \texttt{FireAttackFactory}, \texttt{WaterAttackFactory} y \texttt{GrassAttackFactory}.

    \begin{kotlin}
      interface AttackFactory {
        var name: String
        var damage: Int
        fun createAttack(): Attack
      }
    \end{kotlin}

    \begin{kotlin}
      class FireAttackFactory : AttackFactory {
        override lateinit var name: String
        override var damage: Int = 0

        override fun createAttack() = FireAttack(name, damage)
      }
    \end{kotlin}

    \begin{kotlin}
      class WaterAttackFactory : AttackFactory {
        override lateinit var name: String
        override var damage: Int = 0

        override fun createAttack() = WaterAttack(name, damage)
      }
    \end{kotlin}

    Corramos los tests y veremos que pasan.

    Ahora podemos crear un \textit{test factory} para reusar código en los tests de los ataques:

    \begin{kotlin}
      fun `attack equality and hashcode test`(
        factory: AttackFactory, name: String, damage: Int
      ) = funSpec {
        lateinit var attack: Attack

        beforeTest {
          factory.name = name
          factory.damage = damage
          attack = factory.createAttack()
        }

        test("Two attacks with the same name and damage should be equal") {
          val a = factory.createAttack()
          attack shouldBe a
        }

        test("Two attacks with the same name and damage should have the same hashCode") {
          val a = factory.createAttack()
          attack shouldHaveSameHashCodeAs a
        }
      }
    \end{kotlin}

    Y los usamos en nuestros tests:

    \begin{kotlin}
      class FireAttackTest : FunSpec({
        val name = "Ember"
        val damage = 10
        include(`attack equality and hashcode test`(FireAttackFactory(), name, damage))
      })
    \end{kotlin}

    \begin{kotlin}
      class WaterAttackTest : FunSpec({
        val name = "Water Gun"
        val damage = 10
        include(`attack equality and hashcode test`(WaterAttackFactory(), name, damage))
      })
    \end{kotlin}

    Si corremos los tests, debieran pasar.
    Con esto tenemos suficiente avance como para hacer un \textit{commit}.

    \begin{powershell}
      git add .
      git commit -m "FEAT Adds attacks and attack factories"
    \end{powershell}

  \subsection{Fortalezas y debilidades}
    Lo último que nos queda por hacer es probar las fortalezas y debilidades de los ataques.
    