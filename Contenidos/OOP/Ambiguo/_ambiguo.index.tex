\chapter{El capítulo ambiguo}
\label{chap:ambiguo}
  
Además de esto, cada tipo tiene una ventaja sobre otro tipo.
Cuando un \textit{Bakémon} ataca a otro \textit{Bakémon} de un tipo que tiene ventaja
sobre el tipo del \textit{Bakémon} atacante, el ataque hace 
\(\frac{3}{2} \cdot \mathtt{attack}\) de daño.
Cuando un \textit{Bakémon} ataca a otro \textit{Bakémon} de un tipo resistente, el ataque
hace \(\frac{1}{2} \cdot \mathtt{attack}\) de daño.
Si no es ninguno de los casos anteriores, el ataque hace \(\mathtt{attack}\) de daño.
Las fortalezas y debilidades de los tipos se muestran en la \cref{tab:tipos}.

\begin{table}[ht!]
  \centering
  \begin{tabular}{ |p{2cm}|p{2cm}|p{2cm}|p{2cm}|  }
    \hline
              & \multicolumn{3}{|c|}{Atacado} \\
    \hline
    Atacante  & \textit{Fuego}              & \textit{Agua}           & \textit{Planta} \\
    \hline
    \textit{Fuego}  & \(\times 1\)           & \(\times 1/2\)         & \(\times 3/2\) \\
    \hline
    \textit{Agua}   & \(\times 3/2\)         & \(\times 1\)           & \(\times 1/2\) \\
    \hline
    \textit{Planta} & \(\times 1/2\)         & \(\times 3/2\)         & \(\times 1\) \\
    \hline
    \end{tabular}
  \caption{Tipos de Bakémon y sus ventajas y resistencias}
  \label{tab:tipos}
\end{table}
  \printbibliography[keyword=ambiguo]