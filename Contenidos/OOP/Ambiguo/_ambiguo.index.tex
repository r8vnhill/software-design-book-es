\chapter{El capítulo ambiguo}
  \label{chap:ambiguo}
    
  En este capítulo seguiremos implementando nuestro juego de Bakémon.
  En el capítulo anterior, implementamos la clase \texttt{Bakemon} y sus subclases.
  En este capítulo implementaremos ataques y un sistema de combates simple.

  Cada tipo de Bakémon tiene ventajas y debilidades contra otros tipos de Bakémon.
  Cuando un \textit{Bakémon} ataca a otro \textit{Bakémon} de un tipo que tiene ventaja
  sobre el tipo del \textit{Bakémon} atacante, el ataque hace 
  \(\frac{3}{2} \cdot \mathtt{attack}\) de daño.
  Cuando un \textit{Bakémon} ataca a otro \textit{Bakémon} de un tipo resistente, el ataque
  hace \(\frac{1}{2} \cdot \mathtt{attack}\) de daño.
  Si no es ninguno de los casos anteriores, el ataque hace \(\mathtt{attack}\) de daño.
  Las fortalezas y debilidades de los tipos se muestran en la \cref{tab:tipos}.

  \begin{table}[ht!]
    \centering
    \begin{tabular}{ |p{2cm}|p{2cm}|p{2cm}|p{2cm}|  }
      \hline
                & \multicolumn{3}{|c|}{Atacado} \\
      \hline
      Atacante  & \textit{Fuego}              & \textit{Agua}           & \textit{Planta} \\
      \hline
      \textit{Fuego}  & \(\times 1\)           & \(\times 1/2\)         & \(\times 3/2\) \\
      \hline
      \textit{Agua}   & \(\times 3/2\)         & \(\times 1\)           & \(\times 1/2\) \\
      \hline
      \textit{Planta} & \(\times 1/2\)         & \(\times 3/2\)         & \(\times 1\) \\
      \hline
      \end{tabular}
    \caption{Tipos de Bakémon y sus ventajas y resistencias}
    \label{tab:tipos}
  \end{table}

  \subimport{Ataque_1/}{ataque_1.index.tex}
  \subimport{Ataque_2/}{_ataque_2.index.tex}
  \subimport{.}{Double_Dispatch.tex}
  \subimport{.}{Ataque_3.tex}
  % \subimport{.}{Ataque_4.tex}
  \subimport{.}{Ejercicios.tex}
  \printbibliography[keyword=double-dispatch]