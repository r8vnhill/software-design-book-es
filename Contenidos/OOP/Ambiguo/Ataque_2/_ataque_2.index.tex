\section{Ataques: segundo intento}
  Ahora intentemos agregar las fortalezas y debilidades de los tipos.
  Para esto comencemos por crear dos \textit{test factories}, una para los ataques muy efectivos y
  otra para los ataques poco efectivos:

  \begin{kotlin}
    fun `effective attack test`(
      factory: BakemonFactory,
      name: String,
      health: Int,
      level: Int,
      attack: Int,
      enemyFactory: BakemonFactory,
      enemyName: String,
      enemyHealth: Int,
      enemyLevel: Int,
      enemyAttack: Int,
    ) = funSpec {
      lateinit var bakemon: Bakemon
      lateinit var enemy: Bakemon

      beforeTest {
        bakemon = factory.createBakemon(name, health, level, attack)
        enemy = enemyFactory.createBakemon(enemyName, enemyHealth, enemyLevel, enemyAttack)
      }

      test("An effective attack should deal 1.5 times the damage") {
        bakemon.attackBakemon(enemy)
        enemy.health shouldBe enemyHealth - (attack * 1.5).toInt()
      }
    }
  \end{kotlin}

  \begin{kotlin}
    fun `ineffective attack test`(
      factory: BakemonFactory,
      name: String,
      health: Int,
      level: Int,
      attack: Int,
      enemyFactory: BakemonFactory,
      enemyName: String,
      enemyHealth: Int,
      enemyLevel: Int,
      enemyAttack: Int,
    ) = funSpec {
      lateinit var bakemon: Bakemon
      lateinit var enemy: Bakemon

      beforeTest {
        bakemon = factory.createBakemon(name, health, level, attack)
        enemy = enemyFactory.createBakemon(enemyName, enemyHealth, enemyLevel, enemyAttack)
      }

      test("An ineffective attack should deal 0.5 times the damage") {
        bakemon.attackBakemon(enemy)
        enemy.health shouldBe enemyHealth - attack / 2
      }
    }
  \end{kotlin}

  Ahora podemos usar estas funciones para definir los tests de los ataques entre \textit{Bakémon}
  de distintos tipos:

  \begin{kotlin}
    class FireBakemonTest : FunSpec({
      ...
      include(
        "Fire Bakemon -> Fire Bakemon:",
        `Bakemon can attack Bakemon`(
          FireBakemonFactory(), "Karmander", 25, 5, 4,
          FireBakemonFactory(), "Karmander", 25, 5, 4, 4
        )
      )
      include(
        "Fire Bakemon -> Water Bakemon:",
        `ineffective attack test`(
          FireBakemonFactory(), "Karmander", 25, 5, 4,
          WaterBakemonFactory(), "Kokodile", 25, 5, 4
        )
      )
      include(
        "Fire Bakemon -> Grass Bakemon:",
        `effective attack test`(
          FireBakemonFactory(), "Karmander", 25, 5, 4,
          GrassBakemonFactory(), "Kriko", 25, 5, 4
        )
      )
    })
  \end{kotlin}

  \begin{kotlin}
    class WaterBakemonTest : FunSpec({
      ...
      include(
        "Water Bakemon -> Water Bakemon:",
        `Bakemon can attack Bakemon`(
          WaterBakemonFactory(), "Kokodile", 25, 5, 4,
          WaterBakemonFactory(), "Kokodile", 25, 5, 4, 4
        )
      )
      include(
        "Water Bakemon -> Fire Bakemon:",
        `effective attack test`(
          WaterBakemonFactory(), "Kokodile", 25, 5, 4,
          FireBakemonFactory(), "Karmander", 25, 5, 4
        )
      )
      include("Water Bakemon -> Grass Bakemon:",
        `ineffective attack test`(
          WaterBakemonFactory(), "Kokodile", 25, 5, 4,
          GrassBakemonFactory(), "Kriko", 25, 5, 4
        )
      )
    })
  \end{kotlin}

  Corramos los tests para ver que fallan como esperamos.

  Ahora pasemos a implementar las fortalezas y debilidades de los \textit{Bakémon}.

  La implementación más simple que podríamos pensar es simplemente preguntar por el tipo del
  \textit{Bakémon} enemigo y actuar en consecuencia.
  Veamos cómo implementaríamos esto:

  \begin{kotlin}
    class FireBakemon(name: String, health: Int, level: Int, attack: Int) :
        AbstractBakemon(name, health, level, attack) {

      override fun attackBakemon(bakemon: Bakemon) = when(bakemon) {
        is WaterBakemon -> bakemon.takeDamage(attack / 2)
        is GrassBakemon -> bakemon.takeDamage((attack * 1.5).toInt())
        else -> bakemon.takeDamage(attack)
      }
      ...
    }
  \end{kotlin}

  \begin{kotlin}
    class WaterBakemon(name: String, health: Int, level: Int, attack: Int) :
        AbstractBakemon(name, health, level, attack) {

        override fun attackBakemon(bakemon: Bakemon) = when(bakemon) {
          is GrassBakemon -> bakemon.takeDamage(attack / 2)
          is FireBakemon -> bakemon.takeDamage((attack * 1.5).toInt())
          else -> bakemon.takeDamage(attack)
        }
      ...
    }
  \end{kotlin}

  Ahora corramos los tests y veamos que pasa.
  
  Lo último que nos queda es mejorar nuestro diseño.
  Partamos por revisar nuestras \textit{test factories}, debieran notar que hay mucho código
  repetido.
  El problema es que en realidad todas las \textit{test factories} de los ataques hacen las mismas
  comprobaciones y solo cambia la cantidad de daño que se hace.

  Para esto, primero agreguemos un nuevo parámetro a la función \mintinline{kotlin}|`Bakemon can attack 
  Bakemon`| para que reciba los nombres de los tests:

  \begin{kotlin}
    fun `Bakemon can attack Bakemon`(
      ...
      testNames: Pair<String, String> = "An attack should reduce the enemy's health" to
          "An attack should not set the enemy's health below 0"
    ) = funSpec {
      lateinit var bakemon: Bakemon
      lateinit var enemy: Bakemon

      beforeTest {
        bakemon = factory.createBakemon(name, health, level, attack)
        enemy = enemyFactory.createBakemon(enemyName, enemyHealth, enemyLevel, enemyAttack)
      }

      test(testNames.first) {
        bakemon.attackBakemon(enemy)
        enemy.health shouldBe enemyHealth - expectedDamage
      }

      test(testNames.second) {
        enemy.takeDamage(enemyHealth)
        bakemon.attackBakemon(enemy)
        enemy.health shouldBe 0
      }
    }
  \end{kotlin}

  Aquí tenemos un par (jaja) de cosas nuevas.

  Primero, el tipo \mintinline{kotlin}|Pair| es un tipo de dato que representa un par de elementos,
  en este caso, dos \mintinline{kotlin}|String|s.
  Luego tenemos la función \mintinline{kotlin}|to| que nos permite crear un par de elementos
  de cualquier tipo.
  Por último, notemos que le pasamos un valor directamente a un parámetro de la función, esto se 
  llama \textit{parámetro por defecto}\index{Parámetro por defecto}, y nos permite omitir el 
  parámetro si no queremos cambiar su valor.
  Para entender mejor esto veamos un ejemplo:

  \begin{kotlin}
    fun jojoReference(name: String = "Giorno Giovanna") =
      println("My name is $name and I have a dream")
    
    fun main() {
      jojoReference()
      jojoReference("Jotaro Kujo")
    }
  \end{kotlin}

  En este caso, el programa imprimiría:

  \begin{minted}{text}
    My name is Giorno Giovanna and I have a dream
    My name is Jotaro Kujo and I have a dream
  \end{minted}

  Volviendo a nuestro código, ahora podemos usar esta función para simplificar nuestras
  \textit{test factories}:

  \begin{kotlin}
    fun `effective attack test`(...) = `Bakemon can attack Bakemon`(
      factory, name, health, level, attack,
      enemyFactory, enemyName, enemyHealth, enemyLevel, enemyAttack,
      (attack * 1.5).toInt(),
      "An effective attack should deal the damage" to
          "An effective attack should not set the enemy's health below 0"
    )
  \end{kotlin}

  \begin{kotlin}
    fun `ineffective attack test`(...) = `Bakemon can attack Bakemon`(
      factory, name, health, level, attack,
      enemyFactory, enemyName, enemyHealth, enemyLevel, enemyAttack,
      attack / 2,
      "An ineffective attack should deal the damage" to
          "An ineffective attack should not set the enemy's health below 0"
    )
  \end{kotlin}

  Nos queda una última cosa que hacer, y es mejorar la forma en que decidimos si un ataque es
  efectivo o no.
  Aquí nos topamos con un problema que ya habíamos visto antes, estamos haciendo que cada 
  \textit{Bakémon} tenga que manejar el tipo de cada uno de sus enemigos.
  Esto es un problema porque si queremos agregar un nuevo tipo de \textit{Bakémon} tenemos que
  modificar el método \mintinline{kotlin}|attackBakemon| de todos los \textit{Bakémon} existentes.
  Volvemos a tener el problema de encapsular el cambio.

  Este problema es difícil de resolver, pero vamos a ver una solución que nos permitirá
  implementar nuestro sistema de tipos de una forma más flexible.

  Pero antes, hagamos \textit{commit} de nuestro código.

  \begin{minted}{text}
    git add .
    git commit -m "FEAT Adds effective and ineffective attacks"
  \end{minted}