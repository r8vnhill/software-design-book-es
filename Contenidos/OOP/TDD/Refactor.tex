\section{Mejorando nuestro código}
\label{sec:refactor}

  Hasta ahora hicimos los pasos 1 y 2 del TDD, nos falta el paso 3: \textbf{refactorizar}.
  \index{Refactor}
  Aquí nos centraremos en mejorar la calidad de nuestro código sin cambiar su comportamiento, esto
  significa que nuestros tests deben seguir pasando.

  Primero comenzaremos por cambiar nuestros tests porque, aunque no lo crean, tienen algunos 
  problemas.
  El primero es que estamos repitiendo las llamadas al constructor en cada test, sería ideal 
  hacer este llamado en un sólo lugar y luego usar el objeto que se crea en cada test (recordemos
  que queremos encapsular el cambio).
  Una primera idea sería definir una variable común para todos los tests e inicializarla una sola 
  vez en el cuerpo de la función lambda.

  \begin{kotlin}
    class BakemonTest : FunSpec({
      val bakemon = Bakemon("Kokodile", 25, 5, 4)

      test("Two objects created from the same class should be equal") {
        val b = Bakemon("Kokodile", 25, 5, 4)
        bakemon shouldBe b
      }
      test("Two objects created from the same class should have the same hashCode") {
        val b = Bakemon("Kokodile", 25, 5, 4)
        bakemon shouldHaveSameHashCodeAs b
      }
    })
  \end{kotlin}

  Pero esto tiene un gran problema, como la variable \texttt{bakemon} es común a todos los tests,
  si un test modifica el estado de \texttt{bakemon}, los otros tests pueden fallar.
  Una forma de solucionar eso sería resetear el estado de \texttt{bakemon} en cada test, pero
  con eso volvemos a nuestro problema inicial ya que no estamos encapsulando el cambio.

  Para solucionar esto, \textit{Kotest} provee una función llamada \idx{\texttt{beforeTest}}, esta 
  función se ejecuta antes de cada test y en general se usa para inicializar el estado de los 
  objetos que vamos a usar en todos los tests.
  Es importante notar que \texttt{beforeTest} se ejecuta antes de cada test, por lo que si tenemos
  10 tests, \texttt{beforeTest} se ejecutará 10 veces.
  Con esto, podemos reescribir nuestros tests de la siguiente forma:

  \begin{kotlin}
    class BakemonTest : FunSpec({
      lateinit var bakemon: Bakemon

      beforeTest {
        bakemon = Bakemon("Kokodile", 25, 5, 4)
      }

      test("Two objects created from the same class should be equal") {
        val b = Bakemon("Kokodile", 25, 5, 4)
        bakemon shouldBe b
      }
      test("Two objects created from the same class should have the same hashCode") {
        val b = Bakemon("Kokodile", 25, 5, 4)
        bakemon shouldHaveSameHashCodeAs b
      }
    })
  \end{kotlin}

  Pero notarán algo nuevo aquí, la palabra reservada \idx{\texttt{lateinit}}.
  Esta palabra reservada es una promesa que hacemos a Kotlin de que la variable \texttt{bakemon}
  va a ser inicializada más adelante, por lo que podemos definirla sin pasarle un valor inicial.

  Ahora, nos queda un último problema, estamos repitiendo mucha información entre los tests.
  En particular estamos repitiendo el nombre del Bakemon, su nivel, su ataque y su defensa.
  Esto podría no parecer un problema al principio, pero si tenemos 20 tests es fácil que nos 
  equivoquemos escribiendo alguno de los parámetros, basta que uno de sus dedos se mueva un poquito
  y estaremos comprobando si \textit{Kokodile} es igual \textit{Kolodile}.
  La forma más fácil de arreglar esto es definir los parámetros como valores en el cuerpo de la
  función lambda y luego usarlos en los tests.
  De esta forma, si queremos cambiar algún parámetro, lo hacemos en un sólo lugar y no tenemos
  que preocuparnos por cambiarlo en todos los tests, es decir, estamos encapsulando el cambio.

  \begin{kotlin}
    class BakemonTest : FunSpec({
      lateinit var bakemon: Bakemon

      val bakemonName = "Kokodile"
      val bakemonHealth = 25
      val bakemonLevel = 5
      val bakemonAttack = 4

      beforeTest {
        bakemon = Bakemon(bakemonName, bakemonHealth, bakemonLevel, bakemonAttack)
      }

      test("Two objects created from the same class should be equal") {
        val b = Bakemon(bakemonName, bakemonHealth, bakemonLevel, bakemonAttack)
        bakemon shouldBe b
      }
      test("Two objects created from the same class should have the same hashCode") {
        val b = Bakemon(bakemonName, bakemonHealth, bakemonLevel, bakemonAttack)
        bakemon shouldHaveSameHashCodeAs b
      }
    })
  \end{kotlin}

  Si corremos los tests, debieran seguir pasando.

  Ahora que tenemos nuestros tests más limpios, podemos pasar a refactorizar nuestro código.

  Lo primero que vamos a hacer es simplificar el constructor de \texttt{Bakemon}.
  Notemos que el constructor de \texttt{Bakemon} en realidad hace muy poco, solo inicializa las
  variables de instancia.
  Cuando lo único que hace el constructor primario\index{Constructor primario} de una clase es 
  inicializar las variables de instancia, \textit{Kotlin} provee una sintáxis especial para 
  simplificar el código.
  En lugar de definir un constructor primario, podemos definir en la lista de parámetros del
  constructor cuáles de los parámetros inicializarán variables de instancia anteponiendo 
  \texttt{var} o \texttt{val} a cada parámetro según corresponda, esto creará automáticamente
  un constructor primario que inicializará las variables de instancia con los valores de los
  parámetros.
  De esta forma, podemos reescribir el constructor de \texttt{Bakemon} de la siguiente forma:

  \begin{kotlin}
    class Bakemon(val name: String, val health: Int, val level: Int, val attack: Int) {...}
  \end{kotlin}

  Podemos ejecutar los tests y ver que siguen pasando.

  \begin{ejercicio}{Kygo}
    Refactorizar el código de \texttt{Kygo} para que el constructor primario inicialice las
    variables de instancia.

    {\footnotesize
      Solución: \url{https://github.com/r8vnhill/kygo/commit/eee0385b790366e892d2d47f04e1f1cba2bc8fc5}
    }
  \end{ejercicio}

  Para lo siguiente, vamos a aprovechar que las funciones en \textit{Kotlin} son valores\footnote{
    Esto significa que \textit{Kotlin} es un lenguaje de primera clase, esto es algo que veremos
    más adelante.
  }.
  Cuando una función en \textit{Kotlin} solo tiene una expresión, podemos utilizar una sintáxis más
  compacta para definirla, la idea es que podemos omitir el cuerpo de la función y escribir la expresión
  directamente después de los paréntesis de los parámetros como si estuvieramos asignando una 
  variable.
  Tomando como ejemplo la función \texttt{toString} de \texttt{Bakemon}, podemos reescribirla de la
  siguiente forma:

  \begin{kotlin}
    override fun toString(): String =
      "Bakemon(name='$name', health=$health, level=$level, attack=$attack)"
  \end{kotlin}

  Pero leyendo el código se puede inferir que la función \texttt{toString} devuelve un 
  \texttt{String}, por lo que podemos omitir el tipo de retorno de la función.
  Es importante que esto lo hagamos solo cuando el tipo de retorno de la función es obvio, ya que si
  no lo es, puede ser muy confuso para quien lea el código.

  Sabiendo esto, podemos reescribir la clase \texttt{Bakemon} de la siguiente forma:

  \begin{kotlin}
    class Bakemon(val name: String, val health: Int, val level: Int, val attack: Int) {

      override fun toString() =
        "Bakemon(name='$name', health=$health, level=$level, attack=$attack)"

      override fun equals(other: Any?) = when {
          this === other -> true
          other !is Bakemon -> false
          else -> name == other.name &&
              health == other.health &&
              level == other.level &&
              attack == other.attack
        }

      override fun hashCode() = Objects.hash(Bakemon::class, name, health, level, attack)
    }
  \end{kotlin}

  Nuevamente, podemos ejecutar los tests y ver que siguen pasando, y hacer \textit{commit} de 
  nuestros cambios.

  \begin{powershell}
    git add .
    git commit -m "REFACTOR Simplifies Bakemon class"
  \end{powershell}

  \begin{ejercicio}{Kygo}
    Simplifique el código de \textit{Kygo} convirtiendo las funciones que solo tienen una expresión
    en una asignación de valores.

    Por ejemplo, cambiar:
    \begin{kotlin}
      override fun toFile(filename: String) {
        serializer.toFile(this, filename)
      }
    \end{kotlin}

    Por:

    \begin{kotlin}
      override fun toFile(filename: String) = serializer.toFile(this, filename)
    \end{kotlin}

    { \footnotesize
      Solución: \url{https://github.com/r8vnhill/kygo/commit/276a0d1463e6325ae175d39db57c6968d28d3917}
    }
  \end{ejercicio}

  Y con esto ya completamos nuestra primera iteración de \textit{Test-Driven Development}. En este
  capítulo rompimos y reparamos nuestro código múltiples veces, en los capítulos siguientes 
  seguiremos utilizando esta metodología e iremos profundizando más en el tema a medida que nuestros 
  programas se vayan volviendo más complejos, siempre manteniendo la idea de que el código debe ser
  fácil de entender y de modificar, además de asegurarnos de ir encapsulando el cambio siempre que
  sea posible.