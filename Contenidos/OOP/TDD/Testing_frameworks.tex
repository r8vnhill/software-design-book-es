\section{Testing frameworks}
  \label{subsec:testing-frameworks}

  Un \textit{framework}\index{Framework} es una abstracción en la que un software provee una
  funcionalidad genérica que puede ser extendida/modificada por el usuario, obteniendo así un
  software específico para un problema.
  Podemos pensar en un framework como un esqueleto de un programa, que podemos completar con
  nuestras propias ideas.

  Los \textit{testing frameworks}\index{Testing framework} son frameworks que nos permiten 
  escribir casos de prueba para nuestros programas.
  Actualmente existen \textit{testing frameworks} para casi todos los lenguajes de programación
  utilizados en la industria.

  En este libro vamos a utilizar el framework \idxit{Kotest} para escribir los tests.
