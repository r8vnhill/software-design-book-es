\section{Ejercicios}
\label{sec:tdd:ejercicios}
  \begin{Exercise}[title={Listas}]
    Considere una clase \texttt{IntArrayList} que representa una lista basada \texttt{B)} en 
    arreglos que implementa la interfaz \texttt{IntList}.
    \begin{minted}{kotlin}
      /**
       * This interface represents a list of integers.
       */
      interface IntList {
        /**
         * Adds a new value to the list.
         */
        fun add(value: Int)
        /**
         * Gets the value at the given index.
         */
        fun get(index: Int): Int
        /**
         * Removes the value at the given index.
         */
        fun remove(index: Int): Int
        /**
         * Returns the size of the list.
         */
        fun size(): Int
        /**
         * Returns the capacity of the list.
         */
        fun capacity(): Int
      }
    \end{minted}

    Para su implementación considere que una lista basada en arreglos es una estructura de datos
    que tiene un arreglo de enteros y dos variables: una que representa el tamaño de la lista
    y otra que representa la capacidad del arreglo.
    Puede encontrar más información sobre listas basadas en arreglos en el siguiente video:
    \url{https://www.youtube.com/watch?v=pN2BHqWWGHU}

    Considere que el constructor de la clase \texttt{IntArrayList} se puede utilizar de la siguiente
    forma: 
    \begin{minted}{kotlin}
      val list = IntArrayList(intArrayOf(1, 2, 3))
    \end{minted}

    Recuerde hacer \textit{commit} de su trabajo después de cada pregunta.

    \Question Cree un proyecto \textit{lists} en \textit{IntelliJ} de la forma que se vió en 
      este capítulo utilizando \textit{Gradle} como sistema de construcción y agregue las 
      dependencias de \textit{Kotest}.
    \Question Cree una clase \texttt{IntArrayTest} que extienda de la clase \texttt{FunSpec}.
    \Question Escriba un test que verifique que el tamaño de una lista vacía es cero.
    \Question Escriba un test que verifique que el tamaño de una lista es el número de elementos
      con los que se inicializó.
    \Question Escriba un test que verifique que la capacidad de una lista vacía es 1.
    \Question Escriba un test que verifique que la capacidad de una lista es el doble del tamaño del
      arreglo con el que se inicializó.
    \Question Escriba un test que verifique que se pueden obtener elementos de una lista.
    \Question Escriba un test que verifique que el tamaño de listas creadas con 1 y 2 elementos
      es 1 y 2 respectivamente.
    \Question Escriba un test que verifique que se pueden agregar elementos a una lista.
      Para esto, utilice un ciclo \mintinline{kotlin}{for} para agregar elementos a la lista.
      Vea el código al final del ejercicio para obtener una idea de cómo hacerlo.
    \Question Escriba un test que verifique que se pueden remover elementos de una lista.
    \Question Implemente la clase \texttt{IntArrayList} de forma que pase todos los tests
      escritos en el ejercicio anterior.
      Puede utilizar el siguiente código como punto de partida:
      \begin{minted}{kotlin}
        class IntArrayList(elements: IntArray) : IntList {
          private var capacity = if (elements.isEmpty()) 1 else elements.size * 2
          var elements = IntArray(capacity)

          init {
            for (i in elements.indices) {
              this.elements[i] = elements[i]
            }
          }
        }
      \end{minted}

    \ExeText A continuación se presenta la estructura que debieran tener sus tests:
      \begin{minted}{kotlin}
        class IntArrayListTest : FunSpec({
          lateinit var list: IntList
          beforeTest {
            list = IntArrayList(intArrayOf())
          }
          test("The size of an empty list should be 0") {...}
          test("The capacity of an empty list should be 1") {...}
          test("The capacity of a list starts at double the size of the array") {
            list = IntArrayList(intArrayOf(1, 2, 3))
            // Verificar que la capacidad es 6
          }
          test("The size of the list is the number of elements it contains") {
            list = IntArrayList(intArrayOf(1, 2, 3))
            // Verificar que el tamaño es 3
          }
          test("An element can be added to the list") {
            for (i in 1..10) {
              list.add(i)
              // Verificar que el tamaño es i
              // Verificar que el último elemento es i
            }
          }
          test("The capacity of the list is doubled when it is full") {
            list = IntArrayList(intArrayOf(1, 2))
            // Verificar que la capacidad es 4
            list.add(3)
            list.add(4)
            // Verificar que la capacidad es 8
          }

          test("An element can be removed from the list") {
            list = IntArrayList(intArrayOf(1, 2, 3))
            list.remove(1) shouldBe 2
            // Verificar que el tamaño es 2
            // Verificar que los elementos son 1 y 3
          }
          test("The capacity of the list is reduced by half when it is 1/4 full") {
            list = IntArrayList(intArrayOf(1, 2, 3, 4))
            for (i in 1..3) {
              list.remove(0)
            }
            // Verificar que la capacidad es 2
          }

          test("Elements of the list can be obtained by index") {
            list = IntArrayList(intArrayOf(1, 2, 3))
            // Verificar que los elementos son 1, 2 y 3
          }
        })
      \end{minted}
  \end{Exercise}

  \begin{Exercise}[title={Bases de datos}]
    Para este ejercicio implementaremos una base de datos de videojuegos.

    \ExePart
      Partiremos modelando la clase \texttt{Game} que representa un videojuego.
      Considere que un videojuego tiene un nombre, un género, un año de lanzamiento y una
      calificación.

      \Question Cree un proyecto \textit{gamedb} en \textit{IntelliJ} de la forma que se vió en 
        este capítulo utilizando \textit{Gradle} como sistema de construcción y agregue las 
        dependencias de \textit{Kotest}.
      \Question Cree una clase \texttt{GameTest} que extienda de la clase \texttt{FunSpec}.
      \Question Escriba un test que verifique que se puede crear un videojuego con el nombre
        \texttt{"Super Mario Bros."}, el género \texttt{"Plataformas"}, el año de lanzamiento
        \texttt{1985} y la calificación \texttt{10.0}, para esto utilice el método 
        \texttt{beforeTest} y compruebe que dos objetos creados con los mismos parámetros son
        iguales.
        Recuerde definir los parámetros pasados a los tests como \mintinline{kotlin}{val} dentro 
        del cuerpo de la expresión lambda.
      \Question Escriba un test que verifique que dos videojuegos con los mismos atributos tienen
        el mismo hashcode.
      \Question Implemente los métodos \texttt{hashCode} y \texttt{equals} de la clase \texttt{Game}
        de forma que pase los tests escritos en el ejercicio anterior.
        Para esto considere que dos videojuegos son iguales si tienen el mismo nombre y año de
        lanzamiento.
      \Question Escriba un test que verifique que dos videojuegos con distinto nombre no son 
        iguales y que tienen distinto hashcode.
      \Question Escriba un test que verifique que dos videojuegos con distinto género no son
        iguales y que tienen distinto hashcode.
      \Question Escriba un test que verifique que dos videojuegos con distinto año de lanzamiento
        no son iguales y que tienen distinto hashcode.
      \Question Escriba un test que verifique que dos videojuegos con distinta calificación no son
        iguales y que tienen distinto hashcode.
    \ExePart
      Ahora implementaremos la base de datos de videojuegos, para esto crearemos un objeto 
      \texttt{GameDatabase} que tendrá un diccionario de videojuegos cuya clave será el código
        hash del videojuego y el valor será el videojuego en sí.
      \Question Cree un una clase \texttt{GameDatabaseTest} que extienda de la clase 
        \texttt{FunSpec}.
        A continuación se presenta un código que puede utilizar como punto de partida:
        \begin{minted}{kotlin}
          class GameDatabaseTest : FunSpec({
            val games = listOf(
              Game("The Legend of Zelda: Breath of the Wild", "Aventura", 2017, 10.0),
              Game("Dark Souls III", "RPG", 2016, 9.2),
              Game("Doki Doki Literature Club!", "Visual Novel", 2017, 9.1),
              Game("Final Fantasy X", "RPG", 2001, 8.0)
            )

            beforeTest {
              GameDatabase.init(games)
            }
          })
        \end{minted}
        
        \Question Considere los siguientes tests:
          \begin{minted}{kotlin}
            test("The database is empty when it is created with an empty list") {
              GameDatabase.init(listOf())
              GameDatabase.entries.size shouldBe 0
            }

            test("The database is initialized with the games in the list") {
              GameDatabase.entries.size shouldBe 4
              for (game in games) {
                GameDatabase.entries shouldContainKey game.hashCode()
              }
            }
          \end{minted}
          Implemente la funcionalidad mínima para que los tests pasen.
          Para esto puede utilizar el siguiente código como punto de partida:
          \begin{minted}{kotlin}
            object GameDatabase {
              val entries = mutableMapOf<Int, Game>()
            }
          \end{minted}
          \textit{Hint: Utilice los métodos \mintinline{kotlin}|MutableMap::clear()| y la expresión
          \mintinline{kotlin}|entries[game.hashCode()] = game| para implementar la inicialización de 
          la base de datos.}
        \Question Escriba un test que verifique que se puede agregar un videojuego a la base de 
          datos.
        \Question Implemente el método \texttt{add} del objeto \texttt{GameDatabase} de forma que
          pase el test.
        \Question Escriba un test que verifique que se puede eliminar un videojuego de la base de
          datos.
          Para esto puede utilizar el método \mintinline{kotlin}|shouldNotContainKey| para verificar
          que un elemento no está en la base de datos.
        \Question Implemente el método \texttt{remove} del objeto \texttt{GameDatabase} de forma
          que pase el test.
          \textit{Hint: Utilice el método \mintinline{kotlin}|MutableMap::remove(K)| para implementar
          el método.}
        \Question Considere la siguiente implementación del método \texttt{search}:
          \begin{minted}{kotlin}
            fun search(name: String, year: Int): Game? {
              return entries[Game(name, "", year, 0.0).hashCode()]
            }
          \end{minted}
          Escriba un test que verifique que se puede buscar un videojuego por nombre y año de
          lanzamiento.
  \end{Exercise}