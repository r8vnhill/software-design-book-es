\chapter{Cuando todo no funciona}
  \label{chap:cuando-todo-no-funciona}

  Como ya vimos, cuando desarrollamos aplicaciones, la complejidad de las mismas va creciendo a 
  medida que se van agregando nuevas funcionalidades. 
  Esto hace que el código se vuelva más difícil de entender y de mantener. 
  Además, cuando se agregan nuevas funcionalidades, es muy común que se rompan las funcionalidades 
  ya existentes D: 
  Esto es un problema, ya que no podemos estar seguros de que las funcionalidades que ya existen 
  siguen funcionando correctamente.

  En este capítulo vamos a ver cómo podemos pensar nuestras aplicaciones de una manera que nos
  permita encontrar estos problemas lo antes posible, y cómo podemos arreglarlos.\footnote{
    O al menos empezar a arreglarlos.
  }

  \section{Test-Driven Development}
    \label{sec:tdd}

    