\chapter{Cuando todo no funciona}
  \label{chap:cuando-todo-no-funciona}

  Como ya vimos, cuando desarrollamos aplicaciones, la complejidad de las mismas va creciendo a 
  medida que se van agregando nuevas funcionalidades. 
  Esto hace que el código se vuelva más difícil de entender y de mantener. 
  Además, cuando se agregan nuevas funcionalidades, es muy común que se rompan las funcionalidades 
  ya existentes D: 
  Esto es un problema, ya que no podemos estar seguros de que las funcionalidades que ya existen 
  siguen funcionando correctamente.

  En este capítulo vamos a ver cómo podemos pensar nuestras aplicaciones de una manera que nos
  permita encontrar estos problemas lo antes posible, y cómo podemos arreglarlos.\footnote{
    O al menos empezar a arreglarlos.
  }

  El \textit{Test-Driven Development}\index{Test-Driven Development} (TDD) es una técnica de 
  desarrollo de software en la que vamos a tratar de probar nuestros programas antes de escribirlos.
  
  ¿Pero a qué me refiero con \textit{probar} un programa?

  Cuando hablamos de probar un programa, nos referimos a que vamos a ejecutar el programa con
  distintos valores de entrada, y vamos a verificar que el programa nos devuelva los valores
  esperados.
  Por ejemplo, si tenemos una función que suma dos números, podemos probarla con distintos pares
  de números, y verificar que la suma de esos números sea la que esperamos.
  Aquí diremos que cada par de números es un \textit{caso de prueba}.\index{Caso de prueba}
  Más formalmente, un caso de prueba es una especificación de los valores de entrada, condiciones de
  ejecución y valores de salida esperados de un programa.

  En el caso de la función de suma, podríamos escribir los siguientes casos de prueba:

  \begin{itemize}
    \item Sumar 1 y 2, y verificar que el resultado sea 3.
    \item Sumar 2 y 2, y verificar que el resultado sea 4.
    \item Sumar 0 y 0, y verificar que el resultado sea 0.
    \item Sumar 1 y 0, y verificar que el resultado sea 1.
  \end{itemize}

  En el caso de la función de suma, los casos de prueba son muy simples, pero en la práctica, los
  casos de prueba pueden ser mucho más complejos.

  Entonces, cuando hablamos de probar un programa, nos referimos a escribir los casos de prueba y
  registrar el resultado de ejecutar estos casos.

  \begin{defaultbox}[Test-Driven Development]
    El \idxit{Test-Driven Development} (TDD) es una metodología de desarrollo de software en la que
    los requisitos del software son traducidos a casos de prueba antes de escribir el código del
    programa.
  \end{defaultbox}

  El TDD se separa en tres pasos:

  \begin{enumerate}
    \item Escribir unos pocos casos de prueba y verificar que todos fallen.\footnote{
        Si escribimos tests que pasan antes de implementar la funcionalidad, hay algo 
        \href{https://www.merriam-webster.com/words-at-play/what-does-sus-mean}{sus} en nuestros 
        tests.
      }
    \item Escribir el código mínimo necesario para que el caso de prueba pase.
    \item Refactorizar\footnote{
      Refactorizar es una técnica de desarrollo de software que consiste en reescribir el código sin 
      cambiar su comportamiento.
    } el código para que sea más legible y fácil de mantener.
  \end{enumerate}

  El TDD es difícil de explicar, así que en lugar de aburrirles con una explicación teórica extensa,
  vamos a ir desarrollando todas las soluciones de aquí en adelante usando TDD.
  Así, vamos a ir viendo cómo funciona el TDD en la práctica.

  ¿Pero por qué deberíamos usar TDD?
  Existen muchas razones para usar TDD, pero mencionemos tres que considero importante:

  \begin{itemize}
    \item Fallar rápido: Si escribimos los casos de prueba antes de escribir el código, vamos a
      encontrar los errores lo antes posible, y vamos a poder arreglarlos antes de que se vuelvan
      más difíciles de arreglar.
    \item Evitar imparcialidad en los tests: Si escribimos los tests después de escribir el código,
      es muy probable que los tests estén sesgados a favor del código que acabamos de escribir.
      Esto es un problema, ya que no podemos estar seguros de que los tests realmente estén
      verificando que el código funciona correctamente.
      Aquí nos importa el comportamiento esperado del código, no cómo está implementado.
    \item Documentación: Los tests son una forma de documentar el código.
      Si escribimos los tests antes de escribir el código, podemos usarlos como una forma de
      especificar el comportamiento del código.
  \end{itemize}

  \subimport{.}{Testing_frameworks.tex}
  \subimport{.}{Gradle.tex}
  \subimport{.}{Kotest.tex}
  \subimport{.}{Kotest_2.tex}
  \subimport{.}{Refactor.tex}

  \printbibliography[keyword=tdd]