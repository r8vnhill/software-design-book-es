\section{Conociendo \textit{Kotest}}
\label{sec:conociendo-kotest}

  Ahora veremos cómo utilizar \textit{Kotest} para testear nuestras aplicaciones siguiendo una
  metodología de desarrollo basada en pruebas.
  Para esto, comencemos por definir lo más básico de nuestro juego.

  Pero antes, instalaremos un nuevo plugin de \textit{IntelliJ} que nos permitirá sacarle más 
  provecho a \textit{Kotest}.
  Para esto, vayamos a \textit{File} $\rightarrow$ \textit{Settings} $\rightarrow$ \textit{Plugins},
  y busquemos \textit{Kotest} en el \textit{Marketplace}.
  Una vez instalado, \textit{IntelliJ} nos preguntará si queremos reiniciar el IDE, hagámosle caso.

  Cuando desarrollamos aplicaciones, las cosas se pueden complicar muy rápido.
  Es por esto que lo que es muy importante partir siempre por lo más simple.
  En este caso, vamos a comenzar con una clase que represente a un \textit{Bakémon}.
  ¿Pero qué es lo más simple?
  Simple, una clase que represente a un Bakémon sin nigún método ni atributo.
  Así que vamos a comenzar por ahí.

  Primero, busquemos 