\section{Principio de Liskov}
  \label{sec:principio-de-liskov}

  Veamos otro problema en nuestra implementación.
  Hasta ahora definimos 3 tipos de cartas: \textit{Monstruo}, \textit{Mágica} y \textit{Trampa}.
  Además definimos un método \texttt{Card::attack(Player)} que es invocado cuando una carta ataca a 
  un jugador.
  ¿No es raro que las cartas \textit{Mágica} y \textit{Trampa} también tengan un método 
  \texttt{attack}?

  En efecto, no tiene sentido que una carta \textit{Mágica} ataque a un jugador, aunque definamos
  un método \texttt{attack} que no haga nada.
  Una de las razones por las que no tiene sentido es porque estamos rompiendo el \textit{principio
  de Liskov}

  \begin{defaultbox}[Principio de sustitución de Liskov]
    \index{Principio de sustitución de Liskov}
    Sea \(\phi(x)\) una propiedad verificable de los objetos \(x\) de tipo \(\mathcal{T}\).
    Entonces, \(\phi(y)\) debe ser verificable para objetos \(y\) de tipo \(\mathcal{S}\) donde
    \(\mathcal{S}\) es un subtipo de \(\mathcal{T}\).
  \end{defaultbox}

  \begin{center}
    ¿Qué?
  \end{center}

  En nuestro caso, la propiedad \(\phi(x)\) es la propiedad de que las cartas pueden atacar a un
  jugador, y \(\mathcal{T}\) es el tipo \texttt{Card}.
  Pero la verdad es que no todas nuestras cartas pueden atacar a un jugador, por lo que no podemos
  decir que \(\phi(y)\) es verificable para todos los objetos de tipo \texttt{Card}.
  Por lo tanto, no podemos decir que el principio de Liskov se cumple.
  Noten que hay casos en los que el \textit{principio de Liskov} se cumplirá por restricciones
  impuestas por el lenguaje, pero hay otros casos (como en nuestro ejemplo) en los que el principio
  de Liskov no se cumple por restricciones lógicas que nosotros mismos impusimos.

  ¿Cómo podemos solucionar este problema?
  La solución en este caso es simple: quitar el método \texttt{attack} de la clase \texttt{Card} y
  definirlo solamente en la clase \texttt{MonsterCard}.
  De la misma forma, el parámetro \texttt{attack} de la clase \texttt{MonsterCard} no tiene
  sentido para las cartas \textit{Mágica} y \textit{Trampa}, por lo que también lo podemos quitar.

  Así, la clase \texttt{AbstractCard} quedaría de la siguiente forma:

  \begin{kotlin}
    abstract class AbstractCard(
      name: String,
      text: String,
      serializer: CardSerializer
    ) : Card {
      override val name = name
      override val text = text
      override var serializer = serializer

      override fun toFile(filename: String) {
        serializer.toFile(this, filename)
      }

      override fun serialize(): String {
        return serializer.serialize(this)
      }
    }
  \end{kotlin}

  Y la clase \texttt{MonsterCard} quedaría de la siguiente forma:

  \begin{kotlin}
    class MonsterCard(
      name: String, text: String, attack: Int, serializer: CardSerializer
    ) : AbstractCard(name, text, serializer) {
      val attack = attack
        
      fun attack(player: Player) {
        player.takeDamage(this.attack)
      }
    }
  \end{kotlin}

  Ahora podemos hacer \textit{commit} de nuestros cambios:

  \begin{powershell}
    git add .
    git commit -m "REFACTOR Remove attack method from Card class"
  \end{powershell}

  \begin{ejercicio}{Kygo}
    Implemente las clases \texttt{MagicCard} y \texttt{TrapCard} que hereden de 
    \texttt{AbstractCard} de modo que respeten el principio de Liskov.

    {\footnotesize 
      Solución: 
      \url{https://github.com/r8vnhill/kygo/commit/c3c436724b712e67ef096b0459fc40067319df8d}
    }
  \end{ejercicio}
  
  El \textit{principio de Liskov} es uno de los principios más importantes de la programación 
  orientada a objetos, y es muy importante que lo tengamos en cuenta a la hora de diseñar nuestras
  aplicaciones.
  En el siguiente capítulo veremos cómo podemos comenzar a diseñar aplicaciones pensando un poco en
  el futuro y en cómo todo se puede caer a pedazos de un momento a otro.
