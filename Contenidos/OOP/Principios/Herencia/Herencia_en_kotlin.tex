
\subsection{Herencia en Kotlin}
  En \textit{Kotlin} existen dos tipos de clases:

  \begin{itemize}
    \item \textbf{Clases finales}: estas clases no pueden ser heredadas.
    \item \textbf{Clases abiertas}: estas clases pueden ser heredadas.
  \end{itemize}

  Por defecto, todas las clases son finales, para hacer una clase abierta se utiliza la palabra
  reservada \textit{open}.
  Por ejemplo, la clase \textit{Card} que definimos antes podría ser abierta:

  \begin{kotlin}
    open class Card(name: String, text: String, attack: Int) {...} 
  \end{kotlin}

  Ahora, creemos una nueva clase \textit{MonsterCard} que herede de \textit{Card}, esto lo haremos
  de la siguiente forma:
                        
  \begin{kotlin}
    class MonsterCard(name: String, text: String, attack: Int) : Card(name, text, attack) {
      ...
    }
  \end{kotlin}

  Vamos viendo:
  \begin{enumerate}
    \item Definimos la clase \textit{MonsterCard} como haríamos normalmente.
    \item Luego de los parámetros del constructor, escribimos dos puntos (\texttt{:}) y el nombre
      de la clase de la que queremos heredar.
    \item Luego de los dos puntos, escribimos el nombre de la clase de la que queremos heredar
    \item Finalmente, entre paréntesis escribimos los parámetros del constructor de la clase
      de la que queremos heredar.
  \end{enumerate}

  \begin{important}
    Una clase puede tener una única superclase, pero una clase padre puede tener múltiples
    subclases.
  \end{important}

  Antes de seguir volvamos a la clase \textit{Card}.
  En la clase \textit{Card} no se declara explícitamente una clase padre, la keyword siendo 
  \textit{explícitamente}.
  Esto es porque en \textit{Kotlin} todas las clases heredan de la clase 
  \textit{Any}\autocite*{BuiltinTypesTheira}, que es la clase padre de todas las clases, y por lo 
  tanto, la clase padre de \textit{Card}.
  Esto significa que lo que hicimos antes es equivalente a:

  \begin{kotlin}
    open class Card(name: String, text: String, attack: Int) : Any() {...}
  \end{kotlin}

  Noten que agregamos paréntesis después del nombre de la clase padre, esto es porque \textit{Any}
  tiene un constructor vacío, y por lo tanto, debemos llamarlo explícitamente.
  Lo primero que se ejecuta en el constructor de una clase es el constructor de la clase padre,
  antes de ejecutar cualquier otra línea de código.

  \begin{important}
    Al no ser un método los constructores no se heredan, por lo que debemos llamar explícitamente
    al constructor de la clase padre.
  \end{important}

  Como vimos, cuando una clase hereda de otra, hereda todas las propiedades y métodos de la clase
  padre.
  Esto significa que no tenemos que volver a definir las propiedades \textit{name}, \textit{text}
  y \textit{attack} en la clase \textit{MonsterCard}, ya que estas propiedades ya están definidas
  en la clase \textit{Card}.
  Lo mismo ocurre con el método \texttt{attack}, ya que este método está definido en la clase
  \textit{Card} y por lo tanto, está disponible para la clase \textit{MonsterCard}.
  Así tendremos algo como:

  \begin{kotlin}
    open class Card(name: String, text: String, attack: Int) {
      val name: String = name
      val text: String = text
      val attack: Int = attack

      fun attack(player: Player) {...}
    }
  \end{kotlin}

  \begin{kotlin}
    class MonsterCard(name: String, text: String, attack: Int) : Card(name, text, attack)
  \end{kotlin}

  ¿Notaron que la clase \textit{MonsterCard} no tiene llaves?
  Esto es porque la clase \textit{MonsterCard} no tiene código adicional, por lo tanto, no
  necesitamos definir un cuerpo para la clase.

  Por último modifiquemos el método \texttt{attack} de la clase \textit{Card} para que sólo las
  cartas de monstruos puedan atacar.

  \begin{kotlin}
    fun attack(player: Player) {
      when(this) {
        is MonsterCard -> player.takeDamage(this.attack)
        else -> println("You can't attack with this card")
      }
    }
  \end{kotlin}

  Aquí tenemos una cosa nueva, la keyword \textit{is}.
  Esta keyword nos permite verificar si una variable es de un tipo determinado.

  Quizás ya vienen venir esto, pero tenemos un problema.
  ¿Qué sucede si queremos agregar nuevos tipos de cartas?
  Pues tendríamos que agregar nuevas condiciones al método \texttt{attack} de la clase 
  \textit{Card}.
  Es decir, seguimos sin encapsular el cambio.

  \begin{defaultbox}[Single-Responsibility Principle]
    Una clase debe tener \textbf{una y solamente} una razón para cambiar, por lo que toda clase 
    debe tener una sola responsabilidad
  \end{defaultbox}

  El problema aquí lo podemos resumir en que la clase \textit{Card} tiene la responsabilidad de
  conocer a todos los tipos de cartas que existen, esto es innecesario ya que cada clase conoce
  su propio tipo.
