\subsection{Method-lookup}
  \label{sec:method-lookup}
  Antes de continuar debemos entender algunos conceptos nuevos.

  El primer concepto nuevo es el de \textbf{firma}\index{Firma}.
  Todas las funciones tienen una firma, que es la combinación de la su nombre y sus parámetros.
  Por ejemplo, la función \texttt{attack} tiene la firma \texttt{attack(Int)}.
  Toda función debe tener una firma única dentro de una clase, es decir, no puede haber dos 
  funciones con el mismo nombre y los mismos parámetros en una misma clase, aunque estas funciones 
  tengan distinto tipo de retorno.
  Por ejemplo, no podemos tener dos funciones \texttt{attack(Int)} en la misma clase, aunque una
  tenga tipo de retorno \texttt{Int} y la otra \texttt{String}.
  Esto es porque la firma de una función es lo único que importa a la hora de llamar a una función,
  y no el tipo de retorno.

  Por ejemplo, si tenemos dos funciones \texttt{attack(Int)} y \texttt{attack(String)}, podemos
  llamar a ambas funciones de la siguiente manera:
  
  \begin{kotlin}
    attack(1)
    attack("1")
  \end{kotlin}

  Esto lo que se conoce como \textbf{sobrecarga de funciones}\index{Sobrecarga de funciones} (o 
  \textbf{overloading}).

  \begin{defaultbox}[Sobrecarga de funciones]
    La sobrecarga de funciones es una característica de los lenguajes de programación que permite
    definir varias funciones con el mismo nombre, pero con distinto número de parámetros o distinto
    tipo de parámetros.
    En Kotlin, la sobrecarga de funciones se realiza de forma automática, por lo que no es necesario
    declarar explícitamente que una función está sobrecargada.
    En otros lenguajes, como \textit{Pascal}, es necesario declarar explícitamente que una función
    está sobrecargada.
  \end{defaultbox}

  La sobrecarga de funciones debe evitarse dentro de lo posible, ya que el proceso de resolver la
  sobrecarga de funciones es complejo\autocite{OverloadResolutionKotlin} y puede llevar a errores 
  difíciles de detectar.
  También puede llevar a confusión, ya que no todos los lenguajes de programación resuelven la
  sobrecarga de funciones de la misma manera, o incluso pueden no resolverla en absoluto (como es el
  caso de \textit{Python}).
  Esto no quiere decir que no debamos usar la sobrecarga de funciones, sino que debemos usarla con
  cuidado y solo cuando sea necesario.

  \begin{note}
    A partir de ahora utilizaremos la siguiente sintaxis para referenciar miembros de una clase:
    
    \texttt{<Class>::<method>(<Parameter Types>): <Return Type>}.
    
    Por ejemplo, la función \texttt{attack(Int)} de la clase \texttt{Card} se representaría como
    \texttt{Card::attack(Int): Unit}.
  \end{note}

  Lo siguiente que debemos entender es el concepto de \textbf{mensaje}.\index{Mensaje}
  Por la propiedad de encapsulamiento\index{Encapsulamiento}, sabemos que no podemos acceder a los 
  atributos de una clase sin pasar por un \enquote{filtro}.
  Esto es, no podemos acceder a los atributos de una clase directamente, sino que debemos hacerlo
  a través de mensajes (message passing).

  \begin{defaultbox}[Method-lookup]
    El proceso de búsqueda de una función a la que enviar un mensaje se conoce como 
    \textbf{method-lookup}.\index{Method-lookup}

    El method-lookup se realiza de la siguiente manera:
    \begin{enumerate}
      \item Se busca la función en la clase actual.
      \item Si no se encuentra, se busca en la clase padre.
      \item Si no se encuentra, se busca en la clase padre de la clase padre.
      \item Y así sucesivamente hasta llegar a la clase \texttt{Any}.
      \item Si no se encuentra en ninguna de las clases, se produce un error.
    \end{enumerate}
    
    En los lenguajes de tipado estático, el method-lookup siempre se resuelve en tiempo de 
    compilación.
  \end{defaultbox}

  \begin{note}
    Un compilador\index{Compilador} es un programa que traduce un programa de un lenguaje a otro.
    En general cuando hablamos de compilación nos referimos a la traducción de un lenguaje de alto 
    nivel a un lenguaje de bajo nivel con el objetivo de que pueda ser ejecutado por una máquina.
    
    Por otro lado, un interprete\index{Interprete} es un programa que ejecuta directamente un 
    programa de un lenguaje de alto nivel.

    Ejemplos de lenguajes compilados son \textit{C}, \textit{C++}, \textit{Java}, \textit{C\#} y
    \textit{Kotlin}.
    Ejemplos de lenguajes interpretados son \textit{Python}, \textit{Ruby}, \textit{MATLAB} y
    \textit{Lisp}.
  \end{note}

  Una manera (no tan lejana de la realidad) de entender el method-lookup es pensar que cada objeto
  tiene una tabla en la que una columna son los mensajes que puede recibir y otra columna son las
  funciones que se ejecutan cuando se recibe ese mensaje.
  Cuando se envía un mensaje a un objeto, el objeto busca en su tabla la función que se debe
  ejecutar para ese mensaje y la ejecuta.
  Si no encuentra la función en su tabla, busca en la tabla de su padre y así sucesivamente hasta
  llegar a la tabla de \texttt{Any}.
  
  Ahora, antes dijimos que la firma de una función debe ser única dentro de una clase.
  ¿Pero qué pasa cuando una clase tiene un método con la misma firma que un método de su padre?

  \begin{defaultbox}[Sobrescritura de funciones]
    La sobrescritura (overriding) de métodos\index{Sobrescritura de funciones} es una característica 
    de los lenguajes de programación que permite definir una función con la misma firma que una 
    función de la clase padre.
    En Kotlin, la sobrecarga \textbf{no} se realiza de forma automática, por lo que es necesario
    declarar explícitamente que una función está sobrescrita con la palabra reservada
    \textit{override}.
  \end{defaultbox}

  \begin{note}
    Al igual que con las clases, existen las funciones abiertas\index{Funciones abiertas} y las
     y finales.\index{Funciones finales}
  \end{note}

  Ahora sí, volvamos a nuestro ejemplo.
  Teníamos que los ataques funcionaban de la siguiente manera:

  \begin{itemize}
    \item Si la carta es un \textit{Monstruo}, el ataque se realiza con el ataque del monstruo.
    \item Si no es un monstruo, entonces no se puede atacar.
  \end{itemize}

  Podemos pensar esto de la siguiente forma: ningúna carta puede atacar, excepto los monstruos.
  ¿Cómo podemos expresar esto en código?

  \begin{kotlin}
    open class Card(name: String, text: String, attack: Int) {
      val name: String = name
      val text: String = text
      val attack: Int = attack

      open fun attack(player: Player) {
        println("You can't attack with this card")
      }
    }
  \end{kotlin}

  \begin{kotlin}
    class MonsterCard(name: String, text: String, attack: Int) : Card(name, text, attack) {
      override fun attack(player: Player) {
        player.takeDamage(this.attack)
      }
    }
  \end{kotlin}

  Probemos nuestro código:

  \begin{kotlin}
    val dragon = MonsterCard(
      "White-eyes Blue Dragon", "Legendary dragon of destruction", 3000)
    val reflect = Card(
      "Reflect Strength", "When a Monster Card attacks you, it is destroyed", 0)
    val deck = mutableListOf(dragon, reflect)
    val player = Player("Mugi", 8000, deck)
    println("$player has ${player.healthPoints} health points")
    dragon.attack(player)
    println("$player has ${player.healthPoints} health points")
    reflect.attack(player)
    println("$player has ${player.healthPoints} health points")
  \end{kotlin}

  Si ejecutamos el código, obtenemos algo como lo siguiente:

  \begin{minted}{text}
    cl.ravenhill.kygo.Player@cc34f4d has 8000 health points
    cl.ravenhill.kygo.Player@cc34f4d has 5000 health points
    You can't attack with this card
    cl.ravenhill.kygo.Player@cc34f4d has 5000 health points
  \end{minted}

  \begin{center}
    ¿Qué?
  \end{center}

  Como ya hemos dicho, todos los objetos heredan de \textit{Any}, esto implica que heredan las 
  propiedades y métodos de \textit{Any}.
  En particular, heredan el método \idx{\texttt{Any::toString(): String}}.

  Cuando hacemos \texttt{println("\$player")}, en realidad lo que estamos haciendo es hacer 
  \texttt{println(player.toString())}.
  El método \texttt{toString()} es un método que poseen todos los objetos que devuelve un string
  que representa al objeto.
  Por defecto, el método \texttt{Any::toString()} devuelve el nombre de la clase seguido
  de \texttt{@} y un número que indica donde está ubicado el objeto en memoria.
  En nuestro caso, el objeto \texttt{player} está ubicado en la dirección de memoria 
  \texttt{cc34f4d}.

  \begin{defaultbox}[Sobrescritura de toString]
    Si queremos que nuestro objeto se imprima de una forma más amigable, podemos sobrescribir el
    método \texttt{toString()}.
  \end{defaultbox}

  ¿Cómo hacemos esto?
  Simplemente sobrescribiendo el método \texttt{toString()} de la siguiente forma:

  \begin{kotlin}
    override fun toString(): String {
      return name
    }  
  \end{kotlin}

  \begin{exercise}
    ¿Qué pasa si se ejecuta el siguiente código?
    ¿Por qué?

    \begin{kotlin}
      open class A {
        open fun f(): Unit {
          println("A.f()")
        }
      }

      class B : A() {
        override fun f(): String {
          return "B.f()"
        }
      }
      
      fun main() {
        val a: A = B()
        a.f()
      }
    \end{kotlin}
  \end{exercise}

  \begin{exercise}
    Implemente las clases \texttt{MagicCard} y \texttt{TrapCard} que hereden de \texttt{Card}.
  \end{exercise}
