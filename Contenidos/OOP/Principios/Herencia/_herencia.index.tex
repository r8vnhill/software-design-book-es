\section{Herencia}
  Consideremos ahora que el mazo puede tener cartas de diferentes tipos: cartas de monstruos, 
  magias y trampas.
  Por ahora, diremos que la única diferencia entre estas cartas es que las cartas de monstruos
  tienen un ataque, mientras que las cartas de magia y trampa no.

  Primero, simplifiquemos un poco nuestro constructor para ir reduciendo la cantidad de código que
  tenemos que escribir:

  \begin{kotlin}
    class Card(name: String, text: String, attack: Int) {
      val name: String = name
      val text: String = text
      val attack: Int = attack
      
      fun attack(player: Player) {
        player.takeDamage(attack)
      }
    }
  \end{kotlin}

  Noten que ahora nos deschicimos del bloque \texttt{init} y definimos los campos de la clase
  directamente en el cuerpo de la clase, esto es equivalente a lo que teníamos antes.

  Ahora sí, veamos cómo podemos definir los tipos de cartas que queremos.
  Una solución es agregar un campo \textit{type} a la clase \textit{Card}, esto podría ser un 
  \textit{String}, además podríamos definir que si intentamos atacar con una carta mágica o trampa
  el daño que se le hace al jugador es 0.

  \begin{kotlin}
    class Card(name: String, text: String, attack: Int, type: String) {
      val name: String = name
      val text: String = text
      val attack: Int = attack
      val type: String = type
      
      fun attack(player: Player) {
        if (type == "Monster") {
          player.takeDamage(attack)
        } else {
          println("You can't attack with a $type card")
        }
      }
    }
  \end{kotlin}

  Esta implementación tiene varios problemas, pero enfoquémosnos en dos:

  \begin{itemize}
    \item Utilizamos un \textit{String} para representar el tipo de carta, esto es un problema 
      porque no podemos asegurar que el valor que se le pase al constructor sea uno de los tipos
      válidos y tendríamos que agregar una condición para verificar que el tipo sea válido.
    \item Si queremos agregar un nuevo tipo de carta, tendríamos que modificar el código de la
      clase \textit{Card} y agregar una nueva condición en el método \texttt{attack}.
  \end{itemize}

  Pero ambos problemas se reducen a lo mismo: no encapsulamos el cambio.
  Si queremos agregar un nuevo tipo de carta, tenemos que modificar el código de todas las clases
  que dependen de la clase \textit{Card}.
  Esto claramente no escala bien.

  Una forma de enfrentar este problema es utilizar \textbf{herencia}.

  \begin{defaultbox}[Herencia]
    La herencia es la propiedad de los objetos de heredar características de otros objetos.
    
    Diremos que un objeto \textit{T} hereda de otro objeto \textit{S} si \textit{T} es una
    especialización de \textit{S}.
    A \textit{T} se le conoce como \textbf{subclase} de \textit{S} y a \textit{S} como
    \textbf{superclase} de \textit{T}.
  \end{defaultbox}

  Veamos un ejemplo de herencia en la vida real.
  Consideremos la familia de animales \textit{Caninae}, todos los miembros de esta familia pueden
  mover la cola (creo), pero puede que lo hagan por distintos motivos.
  
  Consideremos ahora a tres animales de esta familia: el perro, el lobo y el zorro.
  Todos ellos pueden mover la cola, pero el perro lo hace para saludar, el lobo para intimidar y
  el zorro para jugar.
  Podemos decir entonces que la propiedad de mover la cola es una propiedad de la familia 
  \textit{Caninae} que es heredada por todos los miembros de esta familia.

  \textit{Github Copilot} me sugiere escribir \enquote{En programación, la herencia es una forma de 
  reutilizar código}, pero esto es incorrecto.
  La herencia tiene como efecto secundario la reutilización de código, pero no es su objetivo.
  El objetivo de la herencia es especializar una clase, y para esto es necesario que la herencia 
  tenga coherencia lógica.

  Para ver un ejemplo de herencia sin coherencia lógica, pensemos que tanto el perro como el lobo
  pueden aullar.
  Si nuestro objetivo es sólo reutilizar código, podríamos decir que perro es una subclase de lobo,
  pero esto no tiene sentido, ya que los perros y los lobos son especies distintas.

  \subimport{.}{Herencia_en_kotlin.tex}
  \subimport{.}{Method_lookup.tex}