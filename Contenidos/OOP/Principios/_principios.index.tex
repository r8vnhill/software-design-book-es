\chapter{Programación orientada a objetos y amigos}
  \begin{refsection}
    Hasta ahora, gran parte de lo que hemos visto ha sido programación imperativa, es decir,
    programación basada en el concepto de \textit{procedimientos} o \textit{funciones}.
    Hemos implementado \textit{algoritmos} que nos permiten resolver problemas como series de
    instrucciones que se ejecutan de acuerdo a una lógica.
    Sin embargo, en la vida real, los problemas que debemos resolver no son tan simples como
    una serie de instrucciones, sino que son más complejos y requieren de una mayor abstracción
    para poder ser resueltos.

    En este capítulo veremos la programación orientada a objetos, un paradigma de programación
    que nos permite abstraer\footnote{
      Abstraer es el proceso de eliminar detalles innecesarios de un objeto o sistema para 
      enfocarnos en sus características esenciales (\cite{ObjectorientedProgramming2023}).
      Este será un concepto clave en todo lo que resta del libro.
    } problemas complejos en objetos que interactúan entre sí.

    \begin{defaultbox}[Programación orientada a objetos]
      La programación orientada a objetos es un paradigma de computación que se organiza en base a 
      objetos en vez de acciones y datos en vez de lógica.
      Aquí lo que realmente nos importa son los objetos que queremos manipular más que la lógica para
      manipularlos.
    \end{defaultbox}

    Este capítulo es largo, se los advierto.


    \subimport{.}{Objetos.tex}
    \subimport{.}{Clases.tex}
    \subimport{VCS/}{_vcs.index.tex}
    \subimport{Propiedades/}{_propiedades.index.tex}
    \subimport{Herencia/}{_herencia.index.tex}
    \subimport{.}{Clases_abstractas.tex}
    \subimport{.}{Liskov.tex}
    \subimport{.}{Ejercicios.tex}
    \nocite{*}
    \printbibliography[keyword=oop]
  \end{refsection}