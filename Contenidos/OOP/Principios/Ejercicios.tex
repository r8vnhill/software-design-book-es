\newpage
\section{Ejercicios}
  \label{sec:oop:principios:ejercicios}

  \begin{Exercise}[title={Puntos en el espacio}]
    \Question Defina una clase \texttt{Point2D} que represente un punto en el plano.
    \Question Defina el método \texttt{Point2D::toString(): String} que devuelva una representación del 
      punto en el formato \texttt{Point2D(x, y)}.
    \Question La distancia entre dos puntos \(p\) y \(q\) se define como:
      \[d(p, q) = \sqrt{(p_x - q_x)^2 + (p_y - q_y)^2}\]
      Defina un método \texttt{Point2D::distanceTo(Point2D): Double} que calcule la distancia entre dos 
      puntos.
      \textit{Hint: puede importar \texttt{kotlin.math.sqrt} para usar la función raíz cuadrada de 
      la librería estándar de \textit{Kotlin}.}
    \Question Defina la función \texttt{Point2D::distanceToOrigin(): Double} que calcule la 
      distancia entre un punto y el origen del sistema de coordenadas.
    \Question Considere la siguiente implementación de la interfaz \texttt{Point}:
      \begin{kotlin}
        interface Point {
          val coordinates: DoubleArray
          fun distanceTo(other: Point): Double
          fun distanceToOrigin(): Double
        }
      \end{kotlin}
      Modifique la clase \texttt{Point2D} para que implemente la interfaz \texttt{Point}.
    \Question Defina una clase \texttt{Point3D} que implemente la interfaz \texttt{Point}.
      Para esto considere que la distancia entre dos puntos en el espacio se define como:
      \[d(p, q) = \sqrt{(p_x - q_x)^2 + (p_y - q_y)^2 + (p_z - q_z)^2}\]
    \Question Defina una clase \texttt{PointND} que represente un punto en un espacio de 
      \textit{n} dimensiones.
      Para esto considere que la distancia entre dos puntos en el espacio se define como:
      \[d(p, q) = \sqrt{\sum_{i=1}^n (p_i - q_i)^2}\]
      y que puede calcularse de la siguiente forma:
      \begin{enumerate}
        \item Dados dos puntos \(p\) y \(q\), almacene las coordenadas del más corto (menos 
          dimensiones) en una variable \texttt{shorter} y las del más largo (más dimensiones) en una
          variable \texttt{longer}.
        \item Cree una variable \texttt{sum} que almacene el valor 0.
        \item Para cada coordenada \(i\) en \texttt{shorter}:
          \begin{enumerate}
            \item Calcule la diferencia entre la coordenada \(i\) de \texttt{shorter} y la 
              coordenada \(i\) de \texttt{longer}.
            \item Eleve el resultado al cuadrado.
            \item Sumelo a la variable \texttt{sum}.
          \end{enumerate}
        \item Para cada coordenada \(i\) en \texttt{longer} que no haya sido considerada en el paso 
          anterior:
          \begin{enumerate}
            \item Eleve la coordenada al cuadrado.
            \item Sumelo a la variable \texttt{sum}.
          \end{enumerate}
        \item Calcule la raíz cuadrada de \texttt{sum}.
      \end{enumerate}
    \Question Modifique su estructura de clases para representar \texttt{Point2D} y \texttt{Point3D} 
      como subclases de \texttt{PointND}.
    \Question Defina una clase \texttt{Line2D} que represente una línea a partir de dos puntos.
    \Question Cree un método \texttt{Line2D::getLength(): Double} que calcule la longitud de la 
      línea.
      \textit{Hint: puede usar el método \texttt{PointND::distanceTo(PointND)} para calcular la
      distancia entre dos puntos.}
    \Question Implemente el método \texttt{Line2D::distanceTo(Point2D): Double} que calcule la 
      distancia entre una línea y un punto.
      Para esto considere que para toda línea que pase por los puntos \(p\) y \(q\), la distancia
      entre la línea y el punto \(r\) se define como:
      \[
        d(p, q, r) = \frac{
          |(q_y - p_y)r_x - (q_x - p_x)r_y + q_xp_y - q_yp_x|
        }{
          \sqrt{(q_y - p_y)^2 + (q_x - p_x)^2}
        }
      \]
  \end{Exercise}

  \begin{Exercise}[title={Algebra Vectorial}]
    \Question Un vector euclideano es un objeto matemático que tiene una magnitud y una dirección.
      Otra forma de pensar en un vector es como una flecha que va desde el origen del sistema de
      coordenadas hasta un punto en el espacio.
      Defina una clase \texttt{Vector3D} que represente un vector en el plano.
    \Question Defina un método \texttt{Vector3D::toString(): String} que devuelva una representación
      del vector en el formato \texttt{Vector3D(x, y, z)}.
    \Question Dos vectores son iguales si sus coordenadas son iguales.
      Defina un método \texttt{Vector3D::equalTo(Vector3D): Boolean} que determine si dos vectores 
      son iguales.
    \Question La suma de dos vectores se define como:
      \[
        \mathbf{v} + \mathbf{w} = 
          (\mathbf{v}_x + \mathbf{w}_x, \mathbf{v}_y + \mathbf{w}_y, \mathbf{v}_z + \mathbf{w}_z)
      \]
      Defina un método \texttt{Vector3D::plus(Vector3D): Vector3D} que calcule la suma de dos 
      vectores.
    \Question La resta de dos vectores se define como:
      \[
        \mathbf{v} - \mathbf{w} = 
          (\mathbf{v}_x - \mathbf{w}_x, \mathbf{v}_y - \mathbf{w}_y, \mathbf{v}_z - \mathbf{w}_z)
      \]
      Defina un método \texttt{Vector3D::minus(Vector3D): Vector3D} que calcule la resta de dos 
      vectores.
    \Question El producto de un vector por un escalar se define como:
      \[
        \mathbf{v} \cdot \alpha = 
          (\alpha \mathbf{v}_x, \alpha \mathbf{v}_y, \alpha \mathbf{v}_z)
      \]
      Defina un método \texttt{Vector3D::times(Double): Vector3D} que calcule el producto de un 
      vector por un escalar.
    \Question El largo (magnitud o norma) de un vector se define como:
      \[
        ||\mathbf{v}|| = \sqrt{\mathbf{v}_x^2 + \mathbf{v}_y^2 + \mathbf{v}_z^2}
      \]
      Defina un método \texttt{Vector3D::length(): Double} que calcule el largo de un vector.
    \Question Normalizar un vector es el proceso de convertirlo en un vector unitario.
      Un vector unitario es aquel cuyo largo es 1.
      Para normalizar un vector, se divide cada coordenada por el largo del vector.
      Defina un método \texttt{Vector3D::normalize(): Vector3D} que devuelva un vector unitario 
      con la misma dirección que el vector original.
    \Question El producto punto entre dos vectores se define como:
      \[
        \mathbf{v} \cdot \mathbf{w} = 
          \mathbf{v}_x \mathbf{w}_x + \mathbf{v}_y \mathbf{w}_y + \mathbf{v}_z \mathbf{w}_z
      \]
      Defina un método \texttt{Vector3D::dot(Vector3D): Double} que calcule el producto punto entre 
      dos vectores.
    \Question El producto cruz entre dos vectores se define como:
      \[
        \mathbf{v} \times \mathbf{w} = 
          (\mathbf{v}_y \mathbf{w}_z - \mathbf{v}_z \mathbf{w}_y, 
           \mathbf{v}_z \mathbf{w}_x - \mathbf{v}_x \mathbf{w}_z, 
           \mathbf{v}_x \mathbf{w}_y - \mathbf{v}_y \mathbf{w}_x)
      \]
      Defina un método \texttt{Vector3D::cross(Vector3D): Vector3D} que calcule el producto cruz 
      entre dos vectores.
    \Question El ángulo entre dos vectores se define como:
      \[
        \theta = \arccos{\left(\frac{\mathbf{v} \cdot \mathbf{w}}{||\mathbf{v}|| ||\mathbf{w}||}\right)}
      \]
      Defina un método \texttt{Vector3D::angleTo(Vector3D): Double} que calcule el ángulo entre dos 
      vectores.
      \textit{Hint: puede importar \url{kotlin.math.acos} para calcular el arcocoseno.}
    \Question Defina un método \texttt{Vector3D::perpendicularTo(Vector3D): Boolean} que determine si dos 
      vectores son perpendiculares.
      Para esto considere que dos vectores son perpendiculares si y solo si su producto punto es 
      cero.
    \Question Defina un método \texttt{Vector3D::oppositeTo(Vector3D): Boolean} que determine si dos 
      vectores son opuestos.
      Para esto considere que dos vectores son opuestos si y solo si
      \[
        \mathbf{v_x} = -\mathbf{w_x} \wedge \mathbf{v_y} = -\mathbf{w_y} \wedge 
          \mathbf{v_z} = -\mathbf{w_z}
      \]
    \Question El vector cero es aquel que tiene todas sus coordenadas en cero.
      Defina un método \texttt{Vector3D::isZero(): Boolean} que determine si un vector es el vector 
      cero.
    \Question Defina un método \texttt{Vector3D::parallelTo(Vector3D): Boolean} que determine si dos 
      vectores son paralelos.
      Para esto considere que dos vectores son paralelos si y solo si su producto cruz es el vector 
      cero.
  \end{Exercise} 