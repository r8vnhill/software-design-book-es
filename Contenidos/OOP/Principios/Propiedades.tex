\section{Propiedades}
  Para las siguientes secciones utilizaremos como ejemplos un juego de cartas inspirado en
  \textit{Yu-Gi-Oh!}.

  \subsection{Composición}
    Comencemos modelando al jugador.
    Un jugador tiene un nombre, una cantidad de puntos de vida y un mazo de cartas.
    Un primer intento podría ser el siguiente:

    \begin{kotlin}
      class Player(
        name: String,
        healthPoints: Int,
        deck: MutableList<String>,
        cardsText: MutableMap<String, String>,
        cardsAttacks: MutableMap<String, Int>
      ) {
        val name: String
        var healthPoints: Int
        val deck: MutableList<String>
        val cardsText: MutableMap<String, String>
        val cardsAttacks: MutableMap<String, Int>

        init {
          this.name = name
          this.healthPoints = healthPoints
          this.deck = deck
          this.cardsText = cardsText
          this.cardsAttacks = cardsAttacks
        }
      }
    \end{kotlin}

    Aquí introducimos dos nuevos tipos de datos: \textit{MutableList} y \textit{MutableMap}.
    Un \textit{MutableList} es una lista que puede ser modificada, es decir, podemos agregar y
    eliminar elementos de la lista.
    Una lista, al igual que un arreglo, es homogénea, es decir, todos los elementos de la lista
    deben ser del mismo tipo.
    El tipo de elementos de la lista se indica entre \texttt{<>}.

    Un \textit{MutableMap} es un diccionario que puede ser modificado.
    Un diccionario es una estructura de datos que asocia una clave con un valor.
    En este caso, el tipo de la clave y el tipo del valor se indican entre \texttt{<>}.

    Ahora, podemos crear un jugador:

    \begin{kotlin}
      val deck = mutableListOf("White-eyes Blue Dragon", "Light Magician")
      val cardsText = mutableMapOf(
        "White-eyes Blue Dragon" to "Legendary dragon of destruction",
        "Light Magician" to "The ultimate sorcerer with a powerful attack"
      )
      val cardsAttacks = mutableMapOf(
        "White-eyes Blue Dragon" to 3000,
        "Light Magician" to 2500
      )
      val player = Player("Yulian", 8000, deck, cardsText, cardsAttacks)
      println(player.name)
      println(player.healthPoints)
      println(player.deck)
      println(player.cardsText)
    \end{kotlin}

    Sin embargo, este modelo tiene algunos problemas.
    Por ejemplo, cada vez que agregamos una carta al mazo, debemos agregar también su texto y su
    ataque a los diccionarios.
    ¿Qué pasa entonces si se nos olvida agregar una carta al diccionario de texto?

    La composicion es la propiedad de todos los objetos de contener a cualquier otro objeto.
    Podemos pensar la composición de objetos como la composición de funciones matemáticas.
    Por ejemplo, la función \texttt{f(x) = x + 1} es una composición de las funciones
    \texttt{g(x) = x + 1} y \texttt{h(x) = x}.

    Para solucionar este problema podemos crear una clase que represente a una carta y tenga campos
    para el texto y el ataque.
    De la misma forma que antes, podemos crear una nueva clase utilizando \textit{IntelliJ}, 
    nombremos a la clase \textit{Card} y agreguemos los campos \textit{name}, \textit{text} y
    \textit{attack}.

    \begin{kotlin}
      class Card(name: String, text: String, attack: Int) {
        val name: String
        val text: String
        val attack: Int

        init {
          this.name = name
          this.text = text
          this.attack = attack
        }
      }
    \end{kotlin}

    Una vez que tenemos la clase \textit{Card}, podemos modificar la clase \textit{Player} para que
    utilice la nueva clase.

    \begin{kotlin}
      class Player(
        name: String,
        healthPoints: Int,
        deck: MutableList<Card>
      ) {
        val name: String
        var healthPoints: Int
        val deck: MutableList<Card>

        init {
          this.name = name
          this.healthPoints = healthPoints
          this.deck = deck
        }
      }
    \end{kotlin}

    Note que ahora nuestro código es más conciso y fácil de leer, esto es una de las ventajas que
    nos ofrece la composición.
    La composición es una de las propiedades más importantes de la programación orientada a objetos,
    ya que nos permite separar problemas complejos en en problemas más simples.
    
    En realidad desde un comienzo estábamos usando composición sin saberlo, ya que la clase 
    \textit{Player} es una composición de las clases \textit{String}, \textit{Int}, 
    \textit{MutableList} y \textit{MutableMap}.
    El problema está en que no la estábamos aprovechando al máximo.
    
    \begin{center}
      ¿Ahora la estamos aprovechando al máximo :0?
    \end{center}

    No realmente, eso es algo que iremos desarrollando a lo largo del libro.

  \subsection{Encapsulación}
    Supongamos que queremos agregar un método para que una carta pueda atacar a un jugador.
    Esto es bastante simple de implementar, solo debemos restar los puntos de ataque de la carta
    al jugador.
    Definamos el método \texttt{attack} en la clase \textit{Card} de la siguiente forma:

    \begin{kotlin}
      fun attack(player: Player) {
        player.healthPoints -= this.attack
      }
    \end{kotlin}

    \begin{note}
      El operador \texttt{-=} (resta y asignación) es una abreviación de \texttt{a = a - b}.
    \end{note}

    Ahora podemos atacar a un jugador con una carta:

    \begin{kotlin}
      val deck = mutableListOf(
        Card("White-eyes Blue Dragon", "Legendary dragon of destruction", 3000),
        Card("Light Magician", "The ultimate sorcerer with a powerful attack", 2500)
      )
      val player = Player("Yulian", 8000, deck)
      player.deck[0].attack(player)
      println(player.healthPoints)
    \end{kotlin}

    Sin embargo, este código tiene un problema.
    ¿Qué pasa si el ataque de la carta es mayor que los puntos de vida del jugador?
    En este caso, el jugador tendrá puntos de vida negativos y eso podría causar problemas en el
    juego.

    Para solucionar este problema podemos agregar una condición en el método \texttt{attack}:

    \begin{kotlin}
      fun attack(player: Player) {
        if (attack < player.healthPoints) {
          player.healthPoints -= attack
        } else {
          player.healthPoints = 0
        }
      }
    \end{kotlin}

    Sin embargo, ahora tenemos un problema diferente.
    ¿Qué pasa si hay otras formas en las que el jugador puede perder puntos de vida?
    Entonces tendríamos que agregar una condición en cada método que modifique los puntos de vida
    del jugador.
    Esto haría nuestro código muchísimo más difícil de mantener.
    
    La encapsulación es la propiedad de los objetos de \enquote{empaquetar} sus datos y métodos, manejando
    y controlando el acceso a ellos.

    ¿Cómo podemos aplicar la encapsulación en este caso?
    Lo primero será definir un método para modificar los puntos de vida del jugador, esto podemos
    hacerlo creando un método \textit{takeDamage} en la clase \textit{Player}:

    \begin{kotlin}
      fun takeDamage(damage: Int) {
        if (damage < healthPoints) {
          healthPoints -= damage
        } else {
          healthPoints = 0
        }
      }
    \end{kotlin}

    Ahora podemos modificar el método \texttt{attack} para que utilice el nuevo método:

    \begin{kotlin}
      fun attack(player: Player) {
        player.takeDamage(this.attack)
      }
    \end{kotlin}

    Pero esto no es suficiente, ya que los puntos de vida del jugador pueden ser modificados desde
    cualquier parte del código sin utilizar el método \texttt{takeDamage}.
    Para solucionar esto podemos utilizar un modificador de acceso llamado \textit{private}.
    Este modificador de acceso nos permite definir que un campo o método solo puede ser accedido
    desde la \textbf{misma clase}.\footnote{Que un campo o método sea privado no significa que no
    pueda ser accedido desde otros objetos, solo significa que no puede ser accedido desde objetos 
    de otras clases.}
    
    Para hacer que un campo o método sea privado debemos agregar el modificador de acceso
    \texttt{private} antes de su definición.
    Por ejemplo, podemos hacer que el campo \textit{healthPoints} de la clase \textit{Player} sea
    privado de la siguiente forma:

    \begin{kotlin}
      class Player(
        name: String,
        healthPoints: Int,
        deck: MutableList<Card>
      ) {
        val name: String
        private var healthPoints: Int
        val deck: MutableList<Card>
        ...
      }
    \end{kotlin}

    Ahora el campo \textit{healthPoints} solo puede ser accedido desde la clase \textit{Player}.
    
    \begin{center}
      Yay
    \end{center}

    Pero introducimos otro problema.

    \begin{center}
      Bro
    \end{center}

    Hicimos que el campo \textit{healthPoints} sólo pueda ser asignado desde la clase 
    \textit{Player}, pero además hicimos que sólo pueda ser \enquote{visto} desde la misma clase.
    Esto significa que no podemos hacer cosas como \texttt{println(player.healthPoints)}.
    Podemos solucionar esto creando un método \textit{getHealthPoints} en la clase \textit{Player},
    pero \textit{Kotlin} nos ofrece una forma más simple de hacer esto.
    La idea es decir que una variable es pública, pero que su valor sólo puede ser asignado 
    (\textit{set}) desde la misma clase.

    \begin{kotlin}
      var healthPoints: Int private set
    \end{kotlin}
