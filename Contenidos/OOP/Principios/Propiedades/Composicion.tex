\section{Composición}

  De momento tenemos jugadores en nuestro juego, pero nuestros jugadores no tienen cartas.
  Un primer intento de solución podría ser el siguiente:

  \begin{kotlin}
    class Player(
      name: String,
      healthPoints: Int,
      deck: MutableList<String>,
      cardsText: MutableMap<String, String>,
      cardsAttacks: MutableMap<String, Int>
    ) {
      val name: String
      var healthPoints: Int
      val deck: MutableList<String>
      val cardsText: MutableMap<String, String>
      val cardsAttacks: MutableMap<String, Int>

      init {
        this.name = name
        this.healthPoints = healthPoints
        this.deck = deck
        this.cardsText = cardsText
        this.cardsAttacks = cardsAttacks
      }
    }
  \end{kotlin}

  Aquí introducimos dos nuevos tipos de datos: 
  \textit{MutableList}\autocite{MutableListKotlinProgramming} y 
  \textit{MutableMap}.\autocite{MutableMapKotlinProgramming}
  Un \textit{MutableList} es una lista que puede ser modificada, es decir, podemos agregar y
  eliminar elementos de la lista.
  Una lista, al igual que un arreglo, es homogénea, es decir, todos los elementos de la lista
  deben ser del mismo tipo.
  El tipo de elementos de la lista se indica entre \texttt{<>}.

  Un \textit{MutableMap} es un diccionario que puede ser modificado.
  Un diccionario es una estructura de datos que asocia una clave con un valor.
  En este caso, el tipo de la clave y el tipo del valor se indican entre \texttt{<>}.

  Ahora, podemos crear un jugador:

  \begin{kotlin}
    val deck = mutableListOf("White-eyes Blue Dragon", "Light Magician")
    val cardsText = mutableMapOf(
      "White-eyes Blue Dragon" to "Legendary dragon of destruction",
      "Light Magician" to "The ultimate sorcerer with a powerful attack"
    )
    val cardsAttacks = mutableMapOf(
      "White-eyes Blue Dragon" to 3000,
      "Light Magician" to 2500
    )
    val player = Player("Yulian", 8000, deck, cardsText, cardsAttacks)
    println(player.name)
    println(player.healthPoints)
    println(player.deck)
    println(player.cardsText)
  \end{kotlin}

  Sin embargo, este modelo tiene algunos problemas.
  Por ejemplo, cada vez que agregamos una carta al mazo, debemos agregar también su texto y su
  ataque a los diccionarios.
  ¿Qué pasa entonces si se nos olvida agregar una carta al diccionario de texto?

  \begin{defaultbox}[Composición]
    La \textbf{composición}\index{Composición} es la propiedad de todos los objetos de contener a 
    cualquier otro objeto.
    Podemos pensar la composición de objetos como la composición de funciones matemáticas.
    Por ejemplo, la función \(f(x) = x + 1\) es una composición de las funciones \(g(x) = x + 1\) y
    \(h(x) = x\).
  \end{defaultbox}

  Para solucionar este problema podemos crear una clase que represente a una carta y tenga campos
  para el texto y el ataque.
  De la misma forma que antes, podemos crear una nueva clase utilizando \textit{IntelliJ}, 
  nombremos a la clase \textit{Card} y agreguemos los campos \textit{name}, \textit{text} y
  \textit{attack}.

  \begin{kotlin}
    class Card(name: String, text: String, attack: Int) {
      val name: String
      val text: String
      val attack: Int

      init {
        this.name = name
        this.text = text
        this.attack = attack
      }
    }
  \end{kotlin}

  Una vez que tenemos la clase \textit{Card}, podemos modificar la clase \textit{Player} para que
  utilice la nueva clase.

  \begin{kotlin}
    class Player(
      name: String,
      healthPoints: Int,
      deck: MutableList<Card>
    ) {
      val name: String
      var healthPoints: Int
      val deck: MutableList<Card>

      init {
        this.name = name
        this.healthPoints = healthPoints
        this.deck = deck
      }
    }
  \end{kotlin}

  Note que ahora nuestro código es más conciso y fácil de leer, esto es una de las ventajas que
  nos ofrece la composición.
  La composición es una de las propiedades más importantes de la programación orientada a objetos,
  ya que nos permite separar problemas complejos en en problemas más simples.

  En realidad desde un comienzo estábamos usando composición sin saberlo, ya que la clase 
  \textit{Player} es una composición de las clases \textit{String}, \textit{Int}, 
  \textit{MutableList} y \textit{MutableMap}.
  El problema está en que no la estábamos aprovechando al máximo.

  \begin{center}
    ¿Ahora la estamos aprovechando al máximo :0?
  \end{center}

  No realmente, eso es algo que iremos desarrollando a lo largo del libro.
