
\section{Encapsulamiento}
  Supongamos que queremos agregar un método para que una carta pueda atacar a un jugador.
  Esto es bastante simple de implementar, solo debemos restar los puntos de ataque de la carta
  al jugador.
  Definamos el método \texttt{attack} en la clase \textit{Card} de la siguiente forma:

  \begin{kotlin}
    fun attack(player: Player) {
      player.health -= this.attack
    }
  \end{kotlin}

  \begin{note}
    El operador \texttt{-=} (resta y asignación) es una abreviación de \texttt{a = a - b}.
  \end{note}

  Ahora podemos atacar a un jugador con una carta:

  \begin{kotlin}
    val deck = mutableListOf(
      Card("White-eyes Blue Dragon", "Legendary dragon of destruction", 3000)
    )
    val player = Player("Mugi", 8000, deck)
    val attacker = 
      Card("Light Magician", "The ultimate sorcerer with a powerful attack", 2500)
    attacker.attack(player)
    println(player.health)
  \end{kotlin}

  Sin embargo, este código tiene un problema.
  ¿Qué pasa si el ataque de la carta es mayor que los puntos de vida del jugador?
  En este caso, el jugador tendrá puntos de vida negativos y eso podría causar problemas en el
  juego.

  Para solucionar este problema podemos agregar una condición en el método \texttt{attack}:

  \begin{kotlin}
    fun attack(player: Player) {
      if (attack < player.health) {
        player.health -= attack
      } else {
        player.health = 0
      }
    }
  \end{kotlin}

  Sin embargo, ahora tenemos un problema diferente.
  ¿Qué pasa si hay otras formas en las que el jugador puede perder puntos de vida?
  Entonces tendríamos que agregar una condición en cada método que modifique los puntos de vida
  del jugador.
  Esto haría nuestro código muchísimo más difícil de mantener.

  \begin{defaultbox}[Encapsulamiento]
    El \textbf{encapsulamiento} es la propiedad de los objetos de \enquote{empaquetar} sus datos y 
    métodos, manejando y controlando el acceso a ellos.
  \end{defaultbox}

  ¿Cómo podemos aplicar la encapsulación en este caso?
  Lo primero será definir un método para modificar los puntos de vida del jugador, esto podemos
  hacerlo creando un método \textit{takeDamage} en la clase \textit{Player}:

  \begin{kotlin}
    fun takeDamage(damage: Int) {
      if (damage < health) {
        health -= damage
      } else {
        health = 0
      }
    }
  \end{kotlin}

  Ahora podemos modificar el método \texttt{attack} para que utilice el nuevo método:

  \begin{kotlin}
    fun attack(player: Player) {
      player.takeDamage(this.attack)
    }
  \end{kotlin}

  Pero esto no es suficiente, ya que los puntos de vida del jugador pueden ser modificados desde
  cualquier parte del código sin utilizar el método \texttt{takeDamage}.
  Para solucionar esto podemos utilizar un modificador de acceso llamado \textit{private}.
  Este modificador de acceso nos permite definir que un campo o método solo puede ser accedido
  desde la \textbf{misma clase}.\footnote{Que un campo o método sea privado no significa que no
  pueda ser accedido desde otros objetos, solo significa que no puede ser accedido desde objetos 
  de otras clases.}

  Para hacer que un campo o método sea privado debemos agregar el modificador de acceso
  \texttt{private} antes de su definición.
  Por ejemplo, podemos hacer que el campo \textit{health} de la clase \textit{Player} sea
  privado de la siguiente forma:

  \begin{kotlin}
    class Player(
      name: String,
      health: Int,
      deck: MutableList<Card>
    ) {
      val name: String
      private var health: Int
      val deck: MutableList<Card>
      ...
    }
  \end{kotlin}

  Ahora el campo \textit{health} solo puede ser accedido desde la clase \textit{Player}.

  \begin{center}
    Yay
  \end{center}

  Pero introducimos otro problema.

  \begin{center}
    Bro
  \end{center}

  Hicimos que el campo \textit{health} sólo pueda ser asignado desde la clase 
  \textit{Player}, pero además hicimos que sólo pueda ser \enquote{visto} desde la misma clase.
  Esto significa que no podemos hacer cosas como \texttt{println(player.health)}.
  Podemos solucionar esto creando un método \textit{gethealth} en la clase \textit{Player},
  pero \textit{Kotlin} nos ofrece una forma más simple de hacer esto.
  La idea es decir que una variable es pública, pero que su valor sólo puede ser asignado 
  (\textit{set}) desde la misma clase.

  \begin{kotlin}
    var health: Int private set
  \end{kotlin}

  El encapsulamiento es un concepto clave en el desarrollo de software, porque (como veremos en este 
  libro) siempre vamos a querer \textit{encapsular el cambio}.
  En este caso, si ahora quisieramos hacer que los puntos de vida del jugador sí puedan ser 
  negativos, solo tendríamos que modificar el método \texttt{takeDamage} y no tendríamos que
  modificar los múltiples métodos que podrían querer modificar los puntos de vida del jugador.
  