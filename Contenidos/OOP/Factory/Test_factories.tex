\section{Test factories}
\label{sec:test-factories}
  Ahora veamos cómo podemos utilizar el patrón de diseño \textit{Abstract Factory} para reducir la
  repetición de código en los tests.

  \textit{Kotest} provee una funcionalidad llamada \textit{test factories} que nos permite
  reusar tests de una manera muy sencilla.
  Para usarlas, crearemos una función que va a utilizar un método de \textit{Kotest} llamado
  \texttt{funSpec}.
  Se entenderá mejor poniéndolo en práctica.

  Primero, creemos un archivo en el paquete \url{cl.ravevnhill.bakemon} del directorio de tests 
  haciendo click derecho en el paquete y seleccionando \textit{New} \(\rightarrow\) 
  \textit{Kotlin File/Class}, y en la ventana que se abre, seleccionaremos \textit{File} y
  lo llamaremos \textit{TestFactories}.
  Luego, agregaremos el siguiente código:

  \begin{kotlin}
    fun `Bakemon equality and hashcode test`(
      factory: BakemonFactory,
      name: String,
      health: Int,
      level: Int,
      attack: Int
    ) = funSpec {
      lateinit var bakemon: Bakemon

      beforeTest {
        bakemon = factory.createBakemon(name, health, level, attack)
      }

      test("Two objects created from the same class should be equal") {
        val b = factory.createBakemon(name, health, level, attack)
        bakemon shouldBe b
      }
      test("Two objects created from the same class should have the same hashCode") {
        val b = factory.createBakemon(name, health, level, attack)
        bakemon shouldHaveSameHashCodeAs b
      }
    }
  \end{kotlin}

  Probablemente les llame la atención un detalle: el nombre de la función tiene espacios.
  Esto es porque \textit{Kotlin} permite definir nombres de variables y funciones 
  \enquote{literales}, esto se hace encerrando el nombre entre \textit{backticks} (\texttt{\`{}}).
  Estos nombres sólo deben usarse en los tests\autocite{CodingConventionsKotlin} ya que abusar de
  ellos puede hacer que el código sea difícil de leer.

  Ahora, veamos cómo podemos utilizar esta función para reducir la repetición de código en los 
  tests.
  Para esto, utilizaremos la función \texttt{include} de \textit{Kotest} que nos permite incluir
  tests definidos en otras funciones.

  \begin{kotlin}
    class FireBakemonTest : FunSpec({
      include(
        `Bakemon equality and hashcode test`(FireBakemonFactory(), "Karmander", 25, 5, 4))
    })
  \end{kotlin}

  \begin{kotlin}
    class WaterBakemonTest : FunSpec({
      include(
        `Bakemon equality and hashcode test`(WaterBakemonFactory(), "Kokodile", 25, 5, 4))
    })
  \end{kotlin}

  \begin{important}
    Cada \textit{test factory} tiene su propio contexto, por lo que no puede acceder a las variables
    definidas fuera de ella, y no se puede acceder a las variables definidas dentro de ella desde
    fuera.

    Esto implica que el método \texttt{beforeTest} es propio de cada \textit{test factory}, por lo
    que no podemos \enquote{reusarlo} entre distintas \textit{test factories}.
  \end{important}