\newpage
\section{Ejercicios}
\label{sec:factory_ejercicios}
  \begin{important}
    Recuerde hacer \textit{commit} después de cada pregunta.
  \end{important}

  \begin{Exercise}[title={Lo que faltó}]
    \Question Implemente la clase \texttt{GrassBakemonFactoryTest} que pruebe la clase 
      \texttt{GrassBakemonFactory}.
    \Question Implemente la clase \texttt{GrassBakemonFactory} para que pase las pruebas.
    \Question Implememte la clase \texttt{GrassBakemonTest} que pruebe la clase 
      \texttt{GrassBakemon} y utilice la fábrica \\\texttt{GrassBakemonFactory}.
    \Question Implemente la clase \texttt{GrassBakemon} para que pase las pruebas.
  \end{Exercise}

  \begin{Exercise}[title={La Pizzería}]
    La pizzería \textit{Imma Cooka} necesita un sistema para manejar sus pedidos.
    Cada pedido tiene un nombre, una dirección y una lista de pizzas.
    Cada pizza tiene una lista de ingredientes y tipo de masa.
    Cada ingrediente tiene un nombre y un precio base.
    
    El precio de una pizza se calcula como el precio base de la masa más la suma de los precios de
    los ingredientes.
    El precio de un pedido se calcula como la suma de los precios de cada pizza.
    El sistema debe permitir agregar pizzas a un pedido, agregar ingredientes a una pizza y
    calcular el precio de un pedido.

    \Question Cree una clase \texttt{IngredientTest} que verifique que un ingrediente se pueda crear 
      con un nombre y un precio base.
      Para esto utilice correctamente el método \texttt{beforeTest}.

      Asuma que tiene acceso a los métodos 
      \mintinline{kotlin}|Ingredient::equals(Any?): Boolean| e 
      \mintinline{kotlin}|Ingredient::hashCode(): Int|.
    \Question Implemente la clase \texttt{Ingredient} para que pase las pruebas.
    \Question Cree una clase \texttt{PizzaTest} que verifique que una pizza se pueda crear con un 
      nombre, una lista de ingredientes y un precio base.
      Para esto utilice correctamente el método \texttt{beforeTest}.

      Asuma que tiene acceso a los métodos 
      \mintinline{kotlin}|Pizza::equals(Any?): Boolean| e 
      \mintinline{kotlin}|Pizza::hashCode(): Int|.
    \Question Implemente la clase \texttt{Pizza} para que pase las pruebas.
    \Question Cree un test que verifique que el precio de una pizza se calcula correctamente.
    \Question Implemente el método \texttt{Pizza::price()} para que pase las pruebas.
    \Question Cree una clase \texttt{OrderTest} que verifique que un pedido se pueda crear con un 
      nombre, una dirección y una lista de pizzas.
      Para esto utilice correctamente el método \texttt{beforeTest}.

      Asuma que tiene acceso a los métodos 
      \mintinline{kotlin}|Order::equals(Any?): Boolean| y
      \mintinline{kotlin}|Order::hashCode(): Int|.
    \Question Implemente la clase \texttt{Order} para que pase las pruebas.
    \Question Cree un test que verifique que el precio de un pedido se calcula correctamente.
    \Question Implemente el método \texttt{Order::price()} para que pase las pruebas.

    \ExeText Además de permitirle a sus clientes elegir los ingredientes que quieren en sus pizzas,
      la pizzería quiere ofrecerles una variedad de pizzas predefinidas, donde cada una tiene un
      porcentaje de descuento sobre el precio final de la pizza.
      Existen dos tipos de pizzas predefinidas: \textit{Margarita} y \textit{Napolitana}
      Los descuentos son los siguientes:

      \begin{itemize}
        \item \textit{Margarita}: 10\%.
        \item \textit{Napolitana}: 15\%.
      \end{itemize}

      Los ingredientes de cada una de estas pizzas son los siguientes:

      \begin{itemize}
        \item \textit{Margarita}: \textit{Tomate}, \textit{Queso} y \textit{Albahaca}.
        \item \textit{Napolitana}: \textit{Tomate}, \textit{Queso} y \textit{Aceitunas}.
      \end{itemize}

      Los precios de cada ingrediente son los siguientes:

      \begin{itemize}
        \item \textit{Tomate}: \$400.
        \item \textit{Queso}: \$500.
        \item \textit{Albahaca}: \$100.
        \item \textit{Aceitunas}: \$200.
      \end{itemize}
    
    \Question Cree la clase \texttt{MargaritaPizzaTest} que verifique que una pizza \textit{Margarita}
      se pueda crear correctamente.
    \Question Considere la siguiente implementación de la clase \texttt{MargaritaPizza}:
      \begin{minted}{kotlin}
        class MargaritaPizza(basePrice: Int): Pizza(basePrice, listOf(
          Ingredient("Tomato", 100),
          Ingredient("Cheese", 200),
          Ingredient("Basil", 300)
        )) {
          override fun price() = (super.price() * 0.9).toInt()
        }
      \end{minted}
      ¿Qué significa la llamada a \texttt{super} en el método \texttt{price()}?
      ¿A qué objeto se está referenciando?
    \Question Considere la interafaz \texttt{PizzaFactory}:
      \begin{minted}{kotlin}
        interface PizzaFactory {
          fun create(basePrice: Int): Pizza
        }
      \end{minted}
      Escriba una clase \texttt{MargaritaPizzaFactoryTest} que verifique que la fábrica de pizzas
      \textit{Margarita} crean pizzas \textit{Margarita} correctamente.
    \Question Utilice \textit{test factories} para reescribir la clase \texttt{MargaritaPizzaTest}
      utilizando la fábrica \texttt{MargaritaPizzaFactory}.
    \Question Cree la clase \texttt{NapolitanaPizzaFactoryTest} que verifique que la fábrica de
      pizzas \textit{Napolitana} crean pizzas \textit{Napolitana} correctamente.
    \Question Cree la clase \texttt{NapolitanaPizzaTest} que verifique que una pizza
      \textit{Napolitana} se pueda crear correctamente utilizando \textit{test factories}.
  \end{Exercise}
