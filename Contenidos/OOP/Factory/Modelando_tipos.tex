\section{Modelando tipos}
  Consideremos que en el juego existen tres tipos de \textit{Bakémon}: \textit{Fuego},
  \textit{Agua} y \textit{Planta}.
  Por ahora pensemos que cada \textit{Bakémon} tiene un solo ataque de su mismo tipos.
  Esto significa que un \textit{Bakémon} de tipo \textit{Fuego} tiene un ataque de tipo
  \textit{Fuego}, un \textit{Bakémon} de tipo \textit{Agua} tiene un ataque de tipo
  \textit{Agua} y un \textit{Bakémon} de tipo \textit{Planta} tiene un ataque de tipo
  \textit{Planta}.

  Comencemos pensando en cómo modelar los tipos de \textit{Bakémon}.
  Una idea es hacer que nuestra clase \texttt{Bakemon} sea abierta para extensión y así poder
  crear subclases de \texttt{Bakemon} para cada tipo de \textit{Bakémon}.
  Esta parece una buena idea, así que comencemos por ahí.

  Pero antes de implementar, debemos escribir tests para seguir el proceso TDD.

  Crearemos tres clases: \texttt{FireBakemonTest}, \texttt{WaterBakemonTest} y
  \texttt{GrassBakemonTest}.

  \begin{kotlin}
    class FireBakemonTest : FunSpec({
      lateinit var bakemon: FireBakemon
      val bakemonName = "Karmander"
      val bakemonHealth = 25
      val bakemonLevel = 5
      val bakemonAttack = 4
      beforeTest {
        bakemon = FireBakemon(bakemonName, bakemonHealth, bakemonLevel, bakemonAttack)
      }
      test("Two objects created from the same class should be equal") {
        val b = FireBakemon(bakemonName, bakemonHealth, bakemonLevel, bakemonAttack)
        bakemon shouldBe b
      }
      test("Two objects created from the same class should have the same hashCode") {
        val b = FireBakemon(bakemonName, bakemonHealth, bakemonLevel, bakemonAttack)
        bakemon shouldHaveSameHashCodeAs b
      }
    })
  \end{kotlin}

  \begin{kotlin}
    class WaterBakemonTest : FunSpec({
      lateinit var bakemon: WaterBakemon
      val bakemonName = "Kokodile"
      val bakemonHealth = 25
      val bakemonLevel = 5
      val bakemonAttack = 4
      beforeTest {
        bakemon = WaterBakemon(bakemonName, bakemonHealth, bakemonLevel, bakemonAttack)
      }
      test("Two objects created from the same class should be equal") {
        val b = WaterBakemon(bakemonName, bakemonHealth, bakemonLevel, bakemonAttack)
        bakemon shouldBe b
      }
      test("Two objects created from the same class should have the same hashCode") {
        val b = WaterBakemon(bakemonName, bakemonHealth, bakemonLevel, bakemonAttack)
        bakemon shouldHaveSameHashCodeAs b
      }
  })
  \end{kotlin}

  \begin{note}
    Omitiremos la implementación de algunas clases y métodos de la clase \texttt{GrassBakemon}
    para no extender demasiado el código.
  \end{note}

  Ahora que tenemos los tests, podemos implementar las clases de \textit{Bakémon}:

  \begin{kotlin}
    class FireBakemon(name: String, health: Int, level: Int, attack: Int) :
        Bakemon(name, health, level, attack) {

      override fun equals(other: Any?) = when {
        this === other -> true
        other !is FireBakemon -> false
        else -> name == other.name &&
          health == other.health &&
          level == other.level &&
          attack == other.attack
      }

      override fun hashCode() = 
        Objects.hash(FireBakemon::class, name, health, level, attack)

      override fun toString() =
        "FireBakemon(name='$name', health=$health, level=$level, attack=$attack)"
    }
  \end{kotlin}

  \begin{kotlin}
    class WaterBakemon(name: String, health: Int, level: Int, attack: Int) :
        Bakemon(name, health, level, attack) {

      override fun equals(other: Any?) = when {
        this === other -> true
        other !is WaterBakemon -> false
        else -> name == other.name &&
          health == other.health &&
          level == other.level &&
          attack == other.attack
      }
      
      override fun hashCode() = 
        Objects.hash(WaterBakemon::class, name, health, level, attack)

      override fun toString() =
        "WaterBakemon(name='$name', health=$health, level=$level, attack=$attack)"
    }
  \end{kotlin}

  Ahora podemos comprobar que los tests pasan, y hacer \textit{commit} de los cambios.

  \begin{powershell}
    git add .
    git commit -m "FEAT Adds FireBakemon and WaterBakemon classes"
  \end{powershell}