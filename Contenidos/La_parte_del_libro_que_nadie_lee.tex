\chapter*{La parte del libro que nadie lee}
  La idea de este \enquote{apunte} nació como una \textit{wiki} de \textit{Github} creada por 
  Juan-Pablo Silva como apoyo para el curso de \textit{Metodologías de Diseño y Programación} 
  dictado por el profesor Alexandre Bergel del Departamento de Ciencias de la Computación, Facultad
  de Ciencias Físicas y Matemáticas de la Universidad de Chile.

  Lo que comenzó como unas notas para complementar las clases del profesor lentamente fue creciendo,
  motivado por esxs alumnxs que buscaban dónde encontrar soluciones para esas pequeñas dudas que no
  les dejaban avanzar.

  El objetivo principal del texto sigue siendo el mismo, plantear explicaciones más detalladas, 
  ejemplos alternativos a los vistos en clases y para dejar un documento al que lxs alumnxs puedan
  recurrir en cualquier momento.

  Este libro no busca ser un reemplazo para las clases del curso, es y será siempre un complemento.

  Esta obra va dirigida a los estudiantes de la facultad así como para cualquier persona que esté
  dando sus primeros pasos en programación.
  El libro presenta una introducción al diseño de software, la programación orientada a objetos y lo
  básico de los lenguajes de programación \textit{Java} y \textit{Kotlin}.
  Se asume que los lectores tienen nociones básicas de programación, conocimiento básico de 
  \textit{Python} y, en menor medida, de \textit{C}.

  Antes de comenzar, debo agradecer a las personas que hicieron posible y motivaron la escritura de
  esto: Beatríz Graboloza, Dimitri Svandich, Nancy Hitschfeld y, por supuesto, Alexandre Bergel y 
  Juan-Pablo Silva.

  \begin{center}
    \today, Santiago, Chile
  \end{center}