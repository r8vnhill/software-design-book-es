\title{EL TIEMPO PASA, Y EL SOFTWARE MUERE}

\author{Ignacio Slater}
\newcommand{\subtitle}{Cómo diseñar programas resistentes al cambio}
\affil{Departamento de Ciencias de la Computación, Universidad de Chile}

\date{\today}


\newtheorem{theorem}{Theorem}
\newtheorem*{note}{Nota}
\newtheorem*{important}{Importante}
\theoremstyle{definition}
\newtheorem{definition}{Definition}[chapter]
\newtheorem{exercise}[definition]{Ejercicio}
\definecolor{LightGray}{gray}{0.9}
\definecolor{Transparent}{gray}{1.0}

\makeatletter
   \let\@epipart\@endpart
   \renewcommand{\@endpart}{\thispagestyle{epigraph}\@epipart}
\makeatother        
 
\setlength\epigraphwidth{\textwidth}

%%
 % Java inline source code
 %%
\newcommand{\java}[1]{\mintinline{java}{#1}}

\newtcblisting{powershell}{
   enhanced,
   breakable,
   listing engine=minted,
   minted style=emacs,
   minted language=powershell,
   minted options={autogobble},
   colback=blue!5!white,
   colframe=Cerulean!75!black,
   listing only,
   left=5mm,enhanced,
   overlay={
      \begin{tcbclipinterior}
         \fill[Cerulean!20!white] (frame.south west) rectangle ([xshift=5mm]frame.north west);
      \end{tcbclipinterior}
   }
}

